\section{Broad Absorption Line Quasars}

19 quasars in our catalogue are classified as broad absorption line (BAL) quasars, using the either the \ac{SDSS} classification flags or the \citet{allen11} catalogue. 
We find that the BAL quasars have typically broader [\ion{O}{III}] than the rest of the sample. 
Note that in the \citet{zakamska16} sample of very red quasars, the incidence of BALs is very high, and these objects have extremely broad [\ion{O}{III}] profiles. 
A two-sided Kolmogorov-Smirnov statistic on the $w_{80}$ distributions returned a p-value of 0.10. 
What does this mean?
Try with different parameters?
Histograms look rubbish so maybe just give the numbers. 


\subsection{Type II quasars}

Implications of our findings on searches for high-redshift type II quasars. 
It could be that type II quasars exist. 
If you look at CIV/MgII the narrow line components are very weak. 
So the contribution from the \ac{BLR} is very weak in luminous quasars, and you just won't see it even if the broad line region is obscured.
Findings in this paper seem to suggest that the static \ac{NLR} is very weak in luminous quasars. 

\todo{Wasn't too sure about what this section was trying to say... Have you considered the \ac{SDSS} Type 2 samples from e.g. Alexandroff et al. ? (http://adsabs.harvard.edu/abs/2013MNRAS.435.3306A). I thought those were pretty luminous, narrow-line objects?}

