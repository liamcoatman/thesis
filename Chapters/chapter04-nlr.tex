% !TEX root = ../main.tex

%************************************************
\chapter{Narrow line region properties}
\label{ch:nlr} 
%************************************************

\section{Introduction}

X-ray and \ac{UV} spectroscopy reveal high velocity outflows to be nearly ubiquitous on sub-parsec scales in high accretion rate \ac{AGN}.
Models of galaxy evolution that invoke \ac{AGN} feedback require these outflows to reach galactic scales and quench star formation in the \ac{AGN} host galaxy. 
In recent years, a huge amount of resources have been devoted to searching for observational evidence of galaxy-wide, \ac{AGN}-driven outflows. 
This has resulted in recent detections of winds in \ac{AGN}-host galaxies using tracers of atomic, molecular, and ionised gas \citep[e.g.][]{nesvadba06,arav08,nesvadba08,moe09,dunn10,alexander10,harrison12,harrison14,nesvadba10,rupke13,veilleux13,nardini15,feruglio10,alatalo11,cimatti13,cicone14}.  

One particularly successful technique has been observing forbidden emission lines, which trace warm (T$\sim$$10^4$K) ionised gas in the \ac{NLR}. 
Because of its high equivalent width, [\ion{O}{III}]\l5008 is the most studied of the narrow \ac{AGN} emission lines. 
In general, the [\ion{O}{III}] emission consists of two components: a narrow, `core' component, with a velocity close to the systemic redshift of the host galaxy, and a broader `wing' component, which is normally blueshifted. 
The general consensus is that the core component traces the gravitational potential of the host galaxy, as the width correlates well with the stellar velocity dispersion. 
On the other hand, the broad, blueshifted wing is tracing outflowing gas. 
This emission appears blueshifted because the far-side of the outflow - that is, the side which is moving away from the line of sight - is obscured \citep[e.g.][]{heckman81,vrtilek85}. 

Observations of broad velocity-widths and blueshifts in narrow emission lines stretch back several decades \citep[e.g.][]{weedman70,stockton76,heckman81,veron81,feldman82,heckman84,vrtilek85,whittle85,boroson92}. 
However, these studies rely on small samples, which are often unrepresentative of the properties of the population. 
More recently, the advent of large optical spectroscopic surveys (e.g. \ac{SDSS}) have facilitated studies of the \ac{NLR} in tens of thousands of \ac{AGN} \citep[e.g.][]{boroson05,greene05a,zhang11,mullaney13,zakamska14,shen14}. 
This has provided constraints on the prevalence and drivers of ionised outflows.   
At the same time, there is strong evidence from spatially resolved spectroscopic observations that these outflows are extended over galaxy scales \citep[e.g.][]{greene09,greene11,hainline13,harrison12,harrison14}. 

However, these studies do not cover the redshift range when star formation and \ac{BH} accretion peaked, and consequently when feedback is predicted to be strongest. 
At these redshifts the bright optical emission lines are redshifted to near-infrared wavelengths, where observations are much more challenging. 
As a consequence, studies at high redshifts have typically relied on relatively small numbers of objects \citep[e.g.][]{netzer04,sulentic04,shen16a}.
These studies find [\ion{O}{III}] to be broader in more luminous \ac{AGN}, suggesting that \ac{AGN} efficiency in driving galaxy-wide outflows increases with luminosity \citep[e.g.][]{netzer04,nesvadba08,kim13,brusa15,carniani15,perna15,bischetti16}. 
In addition, [\ion{O}{III}] is often very weak, or is missing entirely in luminous \ac{AGN} \citep[e.g.][]{netzer04}. 

Other recent studies have looked at the [\ion{O}{III}] emission properties of extreme objects - e.g. heavily obscured quasars \citep{zakamska16} and the most luminous quasars \citep{bischetti16} - at redshifts $z\sim2$. 
The [\ion{O}{III}] emission in these objects is very broad and strongly blueshifted. 
These observations are consistent with galaxy formation models that predict \ac{AGN} feedback to be strongest in luminous, dust-obscured quasars.

In this chapter we analyse the [\ion{O}{III}] properties of a sample of 356 high-luminosity, redshift $1.5 < z < 4$ quasars.
This is the largest study of the narrow line region properties of high redshift quasars ever undertaken. 
The large sample size will help to put observations of extreme objects in context of the \ac{AGN} population as a whole.
We will analyse the [\ion{O}{III}] emission properties as a function of key properties of the quasar, e.g. \ac{BH} mass, luminosity, and accretion rate. 

\section{Quasar Sample}

From our \ac{NIR} spectroscopic catalogue (Chapter~\ref{ch:nirsample}), we have selected 356 quasars which have spectra covering the strong, narrow [\ion{O}{III}] doublet. 
The broad Balmer \hb line is also observed for all but two of the sample. 
In 165 the spectra extend to the broad \ha emission at 6565\AA, and in 260 optical spectra including \ion{C}{IV} are also available (mostly from \ac{SDSS}/\ac{BOSS}). 
The sample, which has a redshift range $1.5 < z < 4$, is summarised in Table~\ref{tab:specnums_ch4}.

\begin{table}
  \centering
  \small 
  \caption{The numbers of quasars with [\ion{O}{III}] line measurements and the spectrographs and telescopes used to obtain the near-infrared spectra.}
  \label{tab:specnums_ch4}
    \begin{tabular}{ccc} 
    \hline
    Spectrograph & Telescope & Number \\
                 &           & \\
    \hline
    FIRE         & MAGELLAN  & 31 \\
    GNIRS        & GEMINI-N  & 28 \\
    ISAAC        & VLT       & 9 \\
    LIRIS        & WHT       & 7 \\
    NIRI         & GEMINI-N  & 29 \\
    NIRSPEC      & Keck II   & 3 \\
    SINFONI      & VLT       & 80 \\
    SOFI         & NTT       & 76 \\
    TRIPLESPEC   & ARC-3.5m  & 27 \\
    TRIPLESPEC   & P200      & 45 \\
    XSHOOTER     & VLT       & 21 \\
    \hline
    \multicolumn{2}{c}{Total} & 356 \\
    \hline
    \end{tabular}
\end{table} 

\section{Parametric Model Fits}

In this section we describe how parameters of the [\ion{O}{III}] emission are derived. 
Our approach is to model the spectra using a power-law continuum, an empirical \ion{Fe}{II} template and multiple Gaussian components to model the emission from the broad and narrow components of \hb and the [\ion{O}{III}] doublet.
This is a model which is commonly adopted in the literature \citep[e.g.][]{shen11}. 
We then derive emission line parameters from the best-fitting models. 
The model-fitting is procedure is more robust when analysing spectra with limited \ac{S/N} (in comparison to measuring line properties directly from the data) and allows the emission from different transitions to be isolated. 

\subsection{Description of model}

Before a spectrum can be modelled, it must first be transformed to the quasar rest-frame.  
The redshift used in this transformation is either derived from the peak of the broad \ha emission ($\sim$ 40 per cent of our sample), from the peak of the broad \hb emission ($\sim$ 40 per cent) or from the peak of the narrow [\ion{O}{III}] emission (20 per cent).
The rest-frame transformation is only required to be accurate to within $\sim$1000\kms\, for our fitting procedure to work. 
In later sections, more precise estimates of the systemic redshift will be calculated using our parametric model fits to the [\ion{O}{III}] emission. 

The continuum and \ion{Fe}{II} emission is first modelled and subtracted using the procedure described in Chapter~\ref{ch:bhmass}. 
The \hb and [\ion{O}{III}] emission is then fit, again using the procedure described in Chapter~\ref{ch:bhmass}. 
However, we make a number of modifications to the parametric model employed in the fit, which we now describe. 

In general, \hb is modelled by two Gaussians with non-negative amplitudes and \ac{FWHM} greater than 1200\kms.
In 10 objects \hb is modelled with a single Gaussian and in 41 objects \hb is modelled with two Gaussians, but the velocity centroids of the two Gaussians are constrained to be equal. 
These spectra generally have low \ac{S/N}, and adding extra freedom to the model does not significantly decrease the minimised reduced $\chi^2$.
In addition there are cases where the blue wing of the \hb emission is below the lower wavelength limit of the spectrograph; in these cases models with more freedom are insufficiently constrained by the data.    

Contributions to the \hb emission from the narrow-line region is weak in the vast majority of our sample, and in general we do not include an additional Gaussian component to model this emission. 
In 9 objects features in the model - data residuals suggest that a narrow emission component is significant, and an additional narrow Gaussian is included for these quasars. 
It is likely that there is some not insignificant contribution from the narrow line region in other quasars in our sample. 
If this is the case then measures of the \hb velocity width will be biased to lower values on average. 
However, measurements of the [\ion{O}{III}] emission (the focus of this chapter) will not be affected by not decomposing \hb in separate broad and narrow components.  

Each component of the [\ion{O}{III}] doublet is fit with one or two Gaussians, depending on the fractional reduced $\chi^2$ difference between the one- and two-component models. 
Concretely, if the addition of the second Gaussian decreases the reduced $\chi^2$ by more than 5 per cent then the double-Gaussian model is accepted.
One hundred and thirty-one are fit with a single Gaussian and 154 with two Gaussians. 
When a single Gaussian is used to model each line, the peak flux ratio of the [\ion{O}{III}] 4960\,\AA\, and 5008\AA\, components are fixed at the expected 1:3 ratio and the width and velocity offsets are set to be equal\footnote{For QSL176, a significantly better fit ($\Delta \chi^2_{\nu} \sim 25\%$) is obtained when the peak flux ratio constraint relaxed; the peak ratio of the best-fitting model is 0.47.}.
In the double Gaussian model, the peak flux ratio of the additional components is again fixed at 1:3, and the width and velocity offsets are again set to be equal. 
Some example fits are shown in Figure~\ref{fig:example_spectrum_grid}

\subsection{Deriving upper limits on the [\ion{O}{III}] \ac{EQW}}
% oiii_upper_limits.py

In 71 objects [\ion{O}{III}] is undetected, or is detected with very low \ac{S/N}. 

Firstly, the best-fitting model comprising the continuum, \ion{Fe}{II}, and \hb emission is subtracted from the spectra, leaving behind only emission due to [\ion{O}{III}].
These spectra are then smoothed by convolving with a Gaussian of width 200\kms. 
From each of these spectra we generate 100 mock spectra, with the flux at each wavelength randomly drawn from a Normal distribution with a mean equal to the flux and a width equal to the known error. 

We then perform an error-weighted linear least-squares regression with a [\ion{O}{III}] template.
To generate this template we generate a median composite spectrum from the continuum- and \ion{Fe}{II}-subtracted spectra of the XX quasars with reliable [\ion{O}{III}] line measurements (see below).
We then run our line-fitting routine on the composite spectrum, and use the best-fitting [\ion{O}{III}] model as a template. 
The equivalent width of the best-fitting model is recorded for each of the 100 realisations of the spectra. 
The error in the equivalent width is defined as the root-mean-square of these values.
We define the upper limit as the mean plus the standard deviation of these trials. 
\todo{Paul: does this way of deriving upper limits make sense?}

\subsection{Modelling \hans}

There are 217 quasars in our sample with spectra covering the \ha emission line. 
Below, we adopt the peak of the [\ion{O}{III}] emission to measure the quasar systemic redshift. 
However, as a test of the reliability of these estimates, systemic redshift estimates are also derived from \ha. 
In this section we describe how the \ha emission was modelled. 

We fit a parametric model, which is very similar to the model described in Chapter~\ref{ch:bhmass}. 
The continuum emission is first modeled and subtracted using the procedure described in Chapter~\ref{ch:bhmass}. 
We then test five different models with increasing degrees of freedom to model the \ha emission. 
The models are summarised in Table~\ref{tab:hamod}. 
They are (1) a single broad Gaussian; (2) two broad Gaussians with identical velocity centroids; (3) two broad Gaussians with different velocity centroids; (4) two broad Gaussians with identical velocity centroids, and additional narrower Gaussians to model narrow \ha emission, and the narrow components of [\ion{N}{II}]\ll6548,6584 and [\ion{S}{II}]\ll6717,6731; (5) two broad Gaussians with different velocity centroids, and additional narrower Gaussians. 
If used, the width and velocity of all narrow components are set to be equal in the fit, and the relative flux ratio of the two [\ion{N}{II}] components is fixed at the expected value of 2.96.
The model we select is the simplest model for which the fractional change in the reduced $\chi^2$ from the model with the lowest reduced $\chi^2$ is less than ten per cent. 

\begin{table}
  \centering
  \small 
  \caption{Summary of models used to fit the \ha emission, and the number of quasars each model is applied to.}
  \label{tab:hamod}
    \begin{tabular}{cccc} 
    \hline
    Model     & Components & Fix centroids? & Number \\
    \hline
    1        & 1 broad Gaussian  & N/A &  10 \\
    2        & 2 broad Gaussians & Yes &  71 \\
    3        & 2 broad Gaussians & No  &  32 \\
    4        & 2 broad Gaussians + narrow Gaussians & Yes & 51 \\
    5        & 2 broad Gaussians + narrow Gaussians & No  & 53 \\
    \hline
    \end{tabular}
\end{table} 

\begin{figure}
    \centering
    \includegraphics[width=\textwidth]{figures/chapter04/example_spectrum_grid.pdf} 
    \caption[{Model fits to the continuum- and \ion{Fe}{II}-subtracted \hbns/[\ion{O}{III}] emission in 15 quasars, chosen at random.}]{Model fits to the continuum- and \ion{Fe}{II}-subtracted \hbns/[\ion{O}{III}] emission in 15 quasars, chosen at random. The data is shown in grey, the best-fitting model in black, and the individual model components in orange. The peak of the [\ion{O}{III}] emission is used to set the redshift, and $\Delta{v}$ is the velocity shift from the rest-frame transition wavelength of \hb. Below each spectrum we plot the data minus model residuals, scaled by the errors on the fluxes.}     
    \label{fig:example_spectrum_grid}
\end{figure}

\subsection{Derived parameters}

All [\ion{O}{III}] line properties are derived from the [\ion{O}{III}]\l5008 emission, but, as described above, the kinematics of [\ion{O}{III}]\l4960 are constrained by our fitting routine to be identical.

We do not attach any physical meaning to the individual Gaussian components used in the model. 
Decomposing the [\ion{O}{III}] emission in to a narrow component component at the systemic redshift and a lower-amplitude, blueshifted broad component is subject to large uncertainties and is highly dependent on the spectral \ac{S/N} and resolution. 
Furthermore, there is no theoretical justification that the broad component should have a Gaussian profile.  

We therefore choose to characterize the [\ion{O}{III}] line profile using a number of non-parametric measures, which are commonly used in the literature \citep[e.g.][]{zakamska14,zakamska16}. 
A normalised cumulative velocity distribution is constructed from the best-fitting model, from which the velocities below which 5, 10, 25, 50, 75, 90, and 95 per cent of the total flux accumulates can be calculated. 
The width of the emission line can then be defined using either $w_{50}$ ($\equiv v_{75} - v_{25}$), $w_{80}$ ($\equiv v_{90} - v_{10}$) or $w_{90}$ ($\equiv v_{95} - v_{5}$). 
In terms of the \ac{FWHM}, $w_{50} \simeq {\mathrm FWHM} / 1.746$, $w_{80} \simeq {\mathrm FWHM} / 0.919$, $w_{90} \simeq {\mathrm FWHM} / 0.716$.  
$w_{90}$ is relatively more sensitive to the wings of the line profile, whereas $w_{50}$ is relatively more sensitive to the core.

Line-width measures are corrected for instrumental broadening by subtracting the resolution of the spectrograph (Table~\ref{tab:specres}) in quadrature.
Because the line profiles are typically non-Gaussian, this deconvolution procedure is only approximate. 
All of the derived parameters are summarised in Table~\ref{tab:specmeasure}. 

\subsection{Reliability of derived parameters}

Our method to estimate realistic uncertainties on emission line properties derived from the best-fitting model is very similar to the one describe in Chapter~\ref{ch:bhmass}. 
Very briefly, random simulations of each spectrum are generated.
Our fitting-procedure is run on each simulated spectrum, and the errors on the line parameters are estimated by looking at the distribution of values from the ensemble of simulations. 
In a slight modification of the procedure in Chapter~\ref{ch:bhmass}, the error is defined using the 68 (84 - 16) percentile spread in the parameter values. 

\subsubsection{Removal of \ion{Fe}{II} emission}

\begin{figure}
    \centering
    \includegraphics[width=\columnwidth]{figures/chapter04/example_spectrum_grid_extreme_fe.pdf} 
    \caption[{Spectra of the 23 objects for which significant \ion{Fe}{II} emission is still visible following our \ion{Fe}{II}-subtraction procedure.}]{Spectra of the 23 objects for which significant \ion{Fe}{II} emission is still visible following our \ion{Fe}{II}-subtraction procedure. The vertical lines indicate the expected positions of the [\ion{O}{III}] doublet (which is generally very weak) with the systemic redshift defined using the peak of the broad \hb emission. \todoinline{Make bigger and split in two?}}     
    \label{fig:bad_fe}
\end{figure}

We encountered 23 cases where the relative strengths of the \ion{Fe}{II} lines appear to differ significantly from those of I Zw 1 on which the \ion{Fe}{II} template we use is based. 
As a result, significant \ion{Fe}{II} flux remained in the spectra after the removal process. 
This emission is at rest-frame wavelengths very close to the [\ion{O}{III}] emission, and so could potentially lead to large errors in the inferred [\ion{O}{III}] line parameters. 
In Figure~\ref{fig:bad_fe} we plot the spectral region around [\ion{O}{III}] for these 23 objects.
The vertical lines indicate the expected positions of the [\ion{O}{III}] doublet, with zero velocity defined using the peak of the broad \hb emission. 
[\ion{O}{III}] is generally extremely weak in these objects. 
As a result, the \ion{Fe}{II} emission could be misinterpreted as broad, shifted [\ion{O}{III}]. 
For example, J125141+080718 was studied by \citet{shen16a}, and assigned an extremely large [\ion{O}{III}] blueshift. 
Our analysis suggests that emission which was modelled by \citet{shen16a} as [\ion{O}{III}] is more likely to be \ion{Fe}{II} which is blueward of the [\ion{O}{III}] laboratory wavelength. 
Because of the difficulty measuring the [\ion{O}{III}] properties of these objects, they are excluded from subsequent analysis.  

\subsubsection{Low \ac{EQW} [\ion{O}{III}]}

\begin{figure}
    \centering
    \includegraphics[width=0.8\textwidth]{figures/chapter04/eqw_cut.pdf} 
    \caption[{Uncertainty in $v_{10}$ as a function of the \ac{EQW}, for [\ion{O}{III}].}]{Uncertainty in $v_{10}$ as a function of the \ac{EQW}, for [\ion{O}{III}]. Uncertainties in $v_{10}$ are large to the left of the vertical line, at 8\AA. These objects are ignored in our subsequent analysis of the [\ion{O}{III}] line shape.}     
    \label{fig:eqw_cut}
\end{figure}

In Figure~\ref{fig:eqw_cut} we show how the uncertainty in $v_{10}$ depends on the \ac{EQW}. 
As the strength of [\ion{O}{III}] decreases, the average uncertainty in $v_{10}$ increases. 
As the \ac{EQW} drops below 8\AA\,, typical uncertainties in $v_{10}$ become very large (exceeding 1000\kms in many objects). 
Clearly, the emission line is too weak for it's shape to be reliably characterised in many of these objects. 
Therefore, when the [\ion{O}{III}] line shape is analysed in later sections, these objects with \ac{EQW} $<$ 8\AA will be excluded.

\section{Reliability of redshift estimates}

\begin{figure}
    \centering
    \includegraphics[width=0.8\linewidth]{figures/chapter04/redshift_comparison.pdf} 
    \caption[{Comparison of systemic redshift estimates using [\ion{O}{III}], broad \hb and broad \hans.}]{Comparison of systemic redshift estimates using [\ion{O}{III}], broad \hb and broad \hans. The probability density functions are generated using a Gaussian kernel density estimator with a $\simeq150$\kms kernel width. The short black lines show the locations of the individual points.}       
    \label{fig:redshift_comparison}
\end{figure}

In this section we do a comparison of systemic redshift estimates from [\ion{O}{III}], broad \hb and \hans. 
The wavelength of each of these lines is measured at the peak of the emission. 
This measurement is done on the best-fitting parameter model. 
In the case of the Balmer lines, this model includes both broad and narrow emission features. 

The redshift comparison is shown in Figure~\ref{fig:redshift_comparison}. 
We compare systemic redshift estimates based on [\ion{O}{III}] and \hb (Figure~\ref{fig:redshift_comparison}a), [\ion{O}{III}] and \ha (Figure~\ref{fig:redshift_comparison}b) and \hb and \ha (Figure~\ref{fig:redshift_comparison}c). 
In (a) and (b) we consider only the subset of objects with [\ion{O}{III}] detections that do not suffer from poor \ion{Fe}{II} subtraction or have extreme [\ion{O}{III}] profiles (210 quasars). 
The available spectroscopic data extends to \hb and \ha for 204 and 99 of these objects respectively. 
Spectroscopic data that covers both \hb and \ha is available for 219 quasars. 
We also exclude objects with extremely large peak wavelength uncertainties due to poor spectra \ac{S/N}. 
We choose cut-offs of 750, 400 and 400 for errors on the \hbns, \ha and [\ion{O}{III}] peaks respectively. 
This leaves 187, 85 and 142 objects in samples (a), (b) and (c) respectively. 

We generate probability density functions using a Gaussian kernel density estimator.
The bandwidth, which is optimised using leave-one-out cross-validation, is 170, 140 and 140 \kms for (a), (b) and (c) respectively. 
The means (medians) of the distributions shown in (a), (b) and (c) are -120 (-90), -90 (-40) and 20 (40) \kms. 
The standard deviations are 360, 300 and 250 \kms. 
The scatter in these distributions is consistent with previous studies of redshift uncertainties from broad emission lines \citep[e.g.][]{shen16b}. 
\todo{Paul: what can I say about these results?}

\section{Results}

\begin{figure}
    \includegraphics[width=0.8\columnwidth]{figures/chapter04/parameter_hists.pdf} 
    \caption[{Correlations between the line width $w_{80}$, asymmetry $R$ and \ac{EQW} of [\ion{O}{III}].}]{Correlations between the line width $w_{80}$, asymmetry $R$ and \ac{EQW} of [\ion{O}{III}].}     
    \label{fig:parameter_hists}
\end{figure}

In our sample of 356 quasars, there is a huge diversity in [\ion{O}{III}] emission properties (Fig.~\ref{fig:example_spectrum_grid}). 
Properties of the [\ion{O}{III}] emission (\ac{EQW}, $w_{80}$ and asymmetry) are shown in Figure~\ref{fig:parameter_hists}. 

The [\ion{O}{III}] \ac{EQW} follows an approximately log-normal distribution, peaking at 17\AA. 
In XX per cent of our sample [\ion{O}{III}] is very weak, with \ac{EQW} $<$ XX \AA. 
\todo{Paul: How to define undetected?}

The width of [\ion{O}{III}], here characterised by $w_{80}$, has a very broad distribution, with extremes at 300 and 3000 \kms. 
The 1200\kms upper limit on the velocity width of the Gaussian functions used to model [\ion{O}{III}] is responsible for the peak at 1200\kms. 

The [\ion{O}{III}] asymmetry is shown in Figure~\ref{fig:parameter_hists}c. 
In 40 per cent of the sample [\ion{O}{III}] is fit with a single Gaussian. 
The asymmetry is zero in this model and so these objects are excluded. 
[\ion{O}{III}] is blue-asymmetric in all but a handful of objects. 

\subsection{Luminosity/redshift-evolution of [\ion{O}{III}] properties}

In this section we compare the [\ion{O}{III}] properties of our quasar sample with a sample of \ac{AGN} at lower redshifts with lower luminosities. 
[\ion{O}{III}] is broader, which is consistent with these quasars having more massive \ac{BH}s. 
[\ion{O}{III}] also shows stronger blue asymmetries, suggesting that outflows are stronger/more prevalent at these higher luminosities/redshifts. 
The luminous blueshifted broad wing and the extremely broad profile reveals high-velocity outflowing ionized gas. 
Our results therefore suggest that kilo-parsec-scale outflows in ionized gas are common in this sample of high-luminosity, high-redshift quasars.

\subsubsection{Equivalent width}

\begin{figure}
    \includegraphics[width=\columnwidth]{figures/chapter04/eqw_lum.pdf} 
    \caption[{The [\ion{O}{III}] \ac{EQW} as a function of the quasar bolometric luminosity for the sample presented in this chapter (blue circles) and the low-$z$ \ac{SDSS} sample (grey points and contours).}]{The [\ion{O}{III}] \ac{EQW} as a function of the quasar bolometric luminosity for the sample presented in this chapter (blue circles) and the low-$z$ \ac{SDSS} sample (grey points and contours). Upper limits are denoted by the downward arrows. \todoinline{Paul: When should upper limits be used? At the moment I'm using $<8$ \AA, but this was defined as the cut-off to measure $w_{80}$.}}     
    \label{fig:eqw_lum}
\end{figure}

In Fig.~\ref{fig:eqw_lum} we show the [\ion{O}{III}] \ac{EQW} as a function of the quasar bolometric luminosity. 
Bolometric luminosity is estimated from the monochromatic continuum luminosity at 5100\AA, using the correction factor given by \citet{richards06}. 
For comparison, we also show the low-$z$ sample from \citet{shen11}.  

We find that [\ion{O}{III}] \ac{EQW} is fairly constant as a function of quasar luminosity in the objects with prominent [\ion{O}{III}] emission. 
However, the proportion of objects in which [\ion{O}{III}] is undetected is much larger in the higher luminosity sample \citep[e.g.][]{netzer04}. 
\citet{netzer04} found 1/3 of their high luminosity sample had very weak [\ion{O}{III}], whereas quasars with weak [\ion{O}{III}] are very rare for nearby \ac{AGN}. 
We find that [\ion{O}{III}] is undetected/very weak in XX per cent of our sample, which is very similar to the fraction reported by \citet{netzer04}.  
As a result, the mean [\ion{O}{III}] \ac{EQW} decreases as a function of luminosity \citep[e.g.][]{brotherton96,netzer04,sulentic04,baskin05b}. 

\subsection{Velocity width}

\begin{figure}
    \includegraphics[width=\columnwidth]{figures/chapter04/oiii_luminosity_z_w80.pdf} 
    \caption[{The [\ion{O}{III}] velocity-width, characterised by $w_{80}$, as a function the [\ion{O}{III}] luminosity and the quasar redshift.}]{The [\ion{O}{III}] velocity-width, characterised by $w_{80}$, as a function the [\ion{O}{III}] luminosity and the quasar redshift. The colour of each hexagon denotes the mean $w_{80}$ for the objects in that luminosity-redshift bin. We have supplemented our sample with low-$z$ objects from \citet{zakamska14} and medium ($z\sim1.5$) redshift objects from \citet{harrison16}.}       
    \label{fig:oiii_luminosity_z_w80}
\end{figure}

In this section we look for any luminosity/redshift dependent changes in the [\ion{O}{III}] line properties. 
To do this we extend the dynamic range of our samples in terms of both luminosity and redshift by supplementing our sample with quasars presented by \citet{mullaney13} and \citet{harrison16}. 
They are selected to have [\ion{O}{III}] luminosities above $10^{8.5}{\rm L}\odot$ and have a median redshift $z=0.397$. 
The \citet{mullaney13} catalogue contains [\ion{O}{III}] line measurements for $\sim$25\,000 \ac{SDSS} spectra. 
We select only the Type I \ac{AGN}. 
The \citet{harrison16} sample contains 40 quasars at intermediate redshifts ($1.1 \leq z \leq 1.7$).

In Figure~\ref{fig:oiii_luminosity_z_w80} we show the [\ion{O}{III}] velocity width as a function of the [\ion{O}{III}] luminosity and the quasar redshift. 
The lack of any redshift-evolution between $z=0$ and $z=1.5$ was reported by \citet{harrison16}.
On the other hand, at fixed redshift, we see a significant correlation between the [\ion{O}{III}] velocity width and the luminosity. 

\subsection{Eigenvector 1 correlations}

\begin{figure}
    \includegraphics[width=\columnwidth]{figures/chapter04/ev1_lowz.pdf} 
    \caption[{\ac{EV1} parameter space.}]{\ac{EV1} parameter space. The contours and shading show low-redshift, low-luminosity SDSS \ac{AGN} (with measurements taken from \citet{shen11}) and the red circles show the high-redshift, high-luminosity objects presented in this chapter.}      
    \label{fig:ev1_lowz}
\end{figure}

The \ac{FWHM} of the broad \hb emission line and the relative strengths of optical \ion{Fe}{II} and \hb have been identified as the features responsible for the largest variance in the spectra of \ac{AGN}. 
These parameters form part of \ac{EV1}, the first eigenvector in a \ac{PCA} which originated from the work of \citet{boroson92}.   
The underlying driver behind EV1 is thought to be the Eddington ratio \citep[e.g.][]{sulentic00b,shen14}. 

In Figure~\ref{fig:ev1_lowz} we show the [\ion{O}{III}] \ac{EQW} as a function of the \hb \ac{FWHM} and the optical \ion{Fe}{II} strength. 
The optical \ion{Fe}{II} strength is defined as the ratio of the \ion{Fe}{II} and \hb \ac{EQW}, where the \ion{Fe}{II} \ac{EQW} is measured between 4434 and 4684\AA.
Measurements of the \hb line properties are taken from Chapter~\ref{ch:bhmass}. 
In our sample, these parameters follow very similar correlations to what is observed at low-$z$ \citep[see also][]{sulentic04, shen16a}.
In particular, we observer a strong anti-correlation between the [\ion{O}{III}] and \ion{Fe}{II} \ac{EQW}.  
The \hb \ac{FWHM} are displaced to higher values, which is consistent with the high-redshift, high-luminosity sample having larger \ac{BH} masses. 

These emission line trends in the optical (for low-$z$ quasars) can be extended to UV emission lines observed at higher redshifts. 
The \ion{C}{IV} blueshift and \ac{EQW} is a diagnostic that similarly spans the diversity of broad emission line properties in high redshift quasars \citep[dominated by a virialized component at one extreme and a wind driven component at the other][]{richards11,sulentic07}. 
The similarity of the \ion{C}{IV} \ac{EQW}-blueshift parameter space at high redshift to \ac{EV1} parameter space at low redshift suggests that these trends are connected. 

Can we calculate a mapping between the two parameter spaces? 
As a first step we show how the \ac{EV1} parameters change as a function of position in the \ion{C}{IV} \ac{EQW}-blueshift parameter space in Figure~\ref{fig:ev1}.
Optical spectra are available for XXX quasars in our catalogue, and cover the broad \ion{C}{IV} doublet. 
As we described in Chapter~\ref{ch:bhmass}, \ion{C}{IV} is often blueshifted, which almost certainly signals the presence of strong outflows, most likely originating in a disc wind.
In Chapter~\ref{ch:bhmass} we demonstrated that the quasars in our sample cover the full range of \ion{C}{IV} blueshifts seen in the \ac{SDSS} quasar population, which makes our sample unique in that it allows us to study properties of the quasar across the full parameter range. 
The \ion{C}{IV} blueshift is measured relative to the redshift determined from the peak of [\ion{O}{III}], \hb or \hans. 
Two hundred and fourteen objects are shown in Figure~\ref{fig:ev1}.
Objects flagged as having significant \ion{Fe}{II} residual emission have been removed.  
Objects for which the \hb or \ion{C}{IV} line properties could not be measured reliably (see Section~\ref{sec:flagged_spectra}) have also been removed. 
Finally, we consider only objects for which the \ion{C}{IV} EQW exceeds 15\AA. 

Most of the diversity in \ion{C}{IV} properties is correlated with the [\ion{O}{III}] \ac{EQW}. 
This is seen more clearly in Fig.~\ref{fig:civ_blueshift_oiii_eqw}, in which we plot the [\ion{O}{III}] \ac{EQW} as a function of the \ion{C}{IV} blueshift. 

\begin{figure}
    \centering
    \includegraphics[width=\columnwidth]{figures/chapter04/civ_blueshift_oiii_eqw.pdf} 
    \caption[{[\ion{O}{III}] \ac{EQW} as a function of the \ion{C}{IV} blueshift.}]{[\ion{O}{III}] \ac{EQW} as a function of the \ion{C}{IV} blueshift.}     
    \label{fig:civ_blueshift_oiii_eqw}
\end{figure}


On the other hand, the \ion{C}{IV} blueshift and \ac{EQW} cannot be used to predict the \hb \ac{FWHM}. 
This is consistent with what we found in Chapter~\ref{ch:bhmass}: objects with large \ion{C}{IV} blueshifts have narrow Balmer emission lines, but objects with modest \ion{C}{IV} blueshifts have a wide range of Balmer line widths. 

We see a correlation between the [\ion{O}{III}] velocity width and asymmetry. 
As the line gets broader it gets more blue-asymmetric. 
One interpretation of this is that the strength of the narrow core is decreasing, leading to a broader and more blueshifted profile \citep[e.g.][]{shen14}. 

\begin{figure}
    \includegraphics[width=\columnwidth]{figures/chapter04/ev1.pdf} 
    \caption[{The high-redshift \ac{EV1} parameter space of \ion{C}{IV} blueshift and \ac{EQW}.}]{The high-redshift \ac{EV1} parameter space of \ion{C}{IV} blueshift and \ac{EQW}. Our sample is shown with points, and quasars from the full \ac{SDSS} catalogue are shown with grey contours. The [\ion{O}{III}] EQW varies systematically with position in the \ion{C}{IV} blueshift-\ac{EQW} parameter space (a) but the \hb \ac{FWHM} shows significantly less systematic variation (b).}      
    \label{fig:ev1}
\end{figure}

\subsubsection{Extreme [\ion{O}{III}] emitters}

\begin{figure}
    \centering
    \includegraphics[width=\columnwidth]{figures/chapter04/example_spectrum_grid_extreme_oiii.pdf} 
    \caption[{Model fits to the continuum- and \ion{Fe}{II}-subtracted \hbns/[\ion{O}{III}] emission in 18 quasars with extreme [\ion{O}{III}] emission profiles.}]{Model fits to the continuum- and \ion{Fe}{II}-subtracted \hbns/[\ion{O}{III}] emission in 18 quasars with extreme [\ion{O}{III}] emission profiles. The data is shown in grey, the best-fitting model in black, and the individual model components in orange. The peak of the [\ion{O}{III}] emission is used to set the redshift, and $\Delta{v}$ is the velocity shift from the rest-frame transition wavelength of \hb. Below each spectrum we plot the data minus model residuals, scaled by the errors on the fluxes.}     
    \label{fig:example_spectrum_grid_extreme_oiii}
\end{figure}

\begin{figure}
    \centering
    \includegraphics[width=\columnwidth]{figures/chapter04/lum_w80.pdf} 
    \caption[{}]{[\ion{O}{III}] velocity width as a function of quasar bolometric luminosity. Objects with extreme [\ion{O}{III}] profiles are shown in red.}     
    \label{fig:lum_w80}
\end{figure}

Extreme [\ion{O}{III}] velocity widths have been extremely red quasars (e.g. Zakamska et al. 2016; Hamman et al. 2016b) and obscured quasars (Brusa et al. 2015). 

Figure~\ref{fig:example_spectrum_grid_extreme_oiii} shows the spectra of 18 objects which we visually identified as having exceptionally broad [\ion{O}{III}] emission profiles. 
These objects are defined as having very broad [\ion{O}{III}] emission (although not necessarily the broadest in our sample) and heavily blended emission in between the zero-velocity wavelengths of \hb and [\ion{O}{III}]. 
In Figure~\ref{fig:lum_w80} we show that these objects all have high luminosities. 
\todo{Paul/Manda: Is this interesting? Just following trend of non-extreme objects?}

These [\ion{O}{III}] emission lines are similar to the lines observed in a sample of four extremely dust-reddened quasars at $z\sim2$ recently identified by \citet{zakamska16}. 
The extreme nature of the [\ion{O}{III}] emission in these objects led \citet{zakamska16} to propose that these objects are being observed transitioning from a dust-obscured, star-burst phase to a luminous, blue quasar. 
A similar [\ion{O}{III}] emission was also observed in J1201+1206 in a sample of five of the most luminous quasars at redshifts $2.3 \lesssim z \lesssim 3.5$ observed by \citet{bischetti16}.

We matched our catalogue to the AllWISE release from WISE with a 5$''$ matching radius. 
Out of 259 quasars, matches were found for 249. 
We did a linear interpolation through the WISE SED to find the flux at rest-frame 5$\mu$m, which we then convert in to a monochromatic luminosity. 
The four \citet{zakamska16} quasars have 5$\mu$m luminosities of $\sim10^{47}$ erg/s, which is comparable to maximum luminosity of our sample.
The [\ion{O}{III}] velocity widths of the \citet{zakamska16} objects are extreme in relation to our sample, matched in 5 micron luminosity. 
The typical dust reddening in our sample is small ($\sim$0.03 by fitting Maddox et al. SED model). 

It is impossible to determine unambiguously what combination of \hb, [\ion{O}{III}] and \ion{Fe}{II} is responsible for the unusual plateau-like emission observed in these objects. 

\subsection{[\ion{O}{III}] and \ion{C}{IV} outflows are linked}

\begin{figure}
    \includegraphics[width=\columnwidth]{figures/chapter04/civ_blueshift_oiii_blueshift.pdf} 
    \caption[{The relation between the blueshifts of \ion{C}{IV} and [\ion{O}{III}].}]{The relation between the blueshifts of \ion{C}{IV} and [\ion{O}{III}]. \todoinline{Move colorbar to bottom.}}     
    \label{fig:oiii_civ_blueshifts}
\end{figure}

We have already seen how [\ion{O}{III}] is broader and more blueshifted in more luminous quasars. 
However, at a given luminosity, what else controls the [\ion{O}{III}] line properties? 
It has been known for some time that the [\ion{O}{III}] \ac{EQW} is anti-correlated with the strength of optical \ion{Fe}{II}, and this trend is thought to be driven by the Eddington ratio. 
\citet{shen14} showed that the amplitude of the core [\ion{O}{III}] emission decreases faster than the wing component as the Eddington ratio increases. 
Therefore, the [\ion{O}{III}] emission is weaker and more blueshifted in high accretion rate quasars.  
In Chapter~\ref{ch:bhmass} we found that all quasars with strong \ac{BLR} outflows have high Eddington ratios. 
In this section, we show that the \ion{C}{IV} and [\ion{O}{III}] blueshifts are directly linked. 
This suggests a direct connection between the gas kinematics in the broad and narrow line regions. 


As described above, \ion{C}{IV} emission properties are available for XX quasars in our sample. 
We use the \ion{C}{IV} velocity centroid measurements we derived in Chapter~\ref{ch:bhmass}.
We take a subset of quasars with [\ion{O}{III}] \ac{EQW} $>8$\AA. 
\todo{Other flags - fe, bad fit - should just be assumed at this point}. 
We also remove objects where the fractional uncertainty on $v_{10}$ exceeds 50 per cent (XX quasars). 

In Figure~\ref{fig:oiii_civ_blueshifts} we show the \ion{C}{IV} blueshifts against the [\ion{O}{III}] blueshifts.
This comparison is done for a sub-sample of 146 objects where we have good measurements of the \ion{C}{IV}, [\ion{O}{III}] profiles. 
\todo{What is good?}

There is a clear and strong correlation. 
Our EQW cut removes most of the quasars with large \ion{C}{IV} blueshifts, since [\ion{O}{III}] is on average very weak in these quasars. 
Similar correlations have been tentatively found in lower redshift quasars and \ac{AGN} \citep{zamanov02}. 

The blueshifting of \ion{C}{IV} is known to correlate with luminosity \citep{richards11}.
In [\ion{O}{III}], the blueshifted wing becomes relatively more prominent as the luminosity of the quasar increases \citep{shen14}. 
Therefore, it is plausible that the correlation between the \ion{C}{IV} and [\ion{O}{III}] blueshifts is a secondary effect that is driven by the correlation of each with the luminosity. 
However, no strong luminosity-dependent trends are apparent in Figure~\ref{fig:oiii_civ_blueshifts}. 
We find that both the [\ion{O}{III}] and \ion{C}{IV} blueshifts are correlated with the luminosity, but that these correlations are much weaker than the correlation between the [\ion{O}{III}] and \ion{C}{IV} blueshifts. 

\todoinline{Also, you could put some more text and maybe a figure to explicitly demonstrate that the trend remains even after you have accounted for the trends with luminosity. This is the highlight result for me and personally I think needs a little more fleshing out.}


The [\ion{O}{III}] `blueshift' is parametrized using as $v_{10} - v_{\mathrm peak}$.
For the \ion{C}{IV} blueshift we use $v_{50}$ as a measure of the line location, and again use the peak of the [\ion{O}{III}] emission to define the systemic redshift. 
This is consistent with the definition we use in Chapter~\ref{ch:bhmass}. 
We considered a number of of alternative approaches to parametrising both the [\ion{O}{III}] line shape and the systemic redshift. 
As expected, very similar trends are observed when the [\ion{O}{III}] line shape is parametrized using $v_{25} - v_{\mathrm peak}$, $v_{50} - v_{\mathrm peak}$, $w_{80} = v_{90} - v_{10}$, or the relative asymmetry (skewness), which we define as:

\begin{equation}
  R = \frac{(v_{90} - v_{\mathrm peak}) - (v_{\mathrm peak} - v_{10})}{(v_{90} - v_{10})}.     
\end{equation} 

The same trend is also observed when the systemic redshift is defined using the peak of the \hb emission. 

Discuss cuts
We don't show extreme objects


\section{Broad Absorption Line Quasars}

\todo{Check all of this}
19 quasars in our catalogue are classified as broad absorption line (BAL) quasars, using the either the \ac{SDSS} classification flags or the \citet{allen11} catalogue. 
We find that the BAL quasars have typically broader [\ion{O}{III}] than the rest of the sample. 
Note that in the \citet{zakamska16} sample of very red quasars, the incidence of BALs is very high, and these objects have extremely broad [\ion{O}{III}] profiles. 
A two-sided Kolmogorov-Smirnov statistic on the $w_{80}$ distributions returned a p-value of 0.10. 
What does this mean?
Try with different parameters?
Histograms look rubbish so maybe just give the numbers. 

\section{Discussion}

Looking at the [\ion{O}{III}] velocity width as a function of luminosity tells us about the physical drivers of the outflows observed in [\ion{O}{III}]. 
The correlation with luminosity suggests that the highest velocity outflows are associated with the most luminous \ac{AGN}. 
This has been reported for low-redshift \ac{AGN}, for both ionized and molecular outflows (e.g. Westmoquette et al. 2012; Veilleux et al. 2013; Arribas et al. 2014; Cicone et al. 2014; Hill \& Zakamska 2014).

This suggests that the outflows are driven by radiative forces. 
On the other hand, \citet{mullaney13} find that once the correlation between the [\ion{O}{III}] luminosity and the radio luminosity has been taken in to account, the [\ion{O}{III}] velocity width is more strongly related to the radio luminosity of the \ac{AGN}. 

Is the \ac{AGN} \ac{NLR} absent in objects where outflows have reached kilo-parsec scales, sweeping up the low-density material responsible for the [\ion{O}{III}]-emission?
If the \ac{BLR} outflows can escape, they are very fast and wouldn't need long to clear out the \ac{NLR} gas. 
\todo{Might be useful to estimate a time-scale for how long the NLR would take to be cleared given typical size of galaxy and velocity of outflow}. 

\subsection{Type II quasars}

Implications of our findings on searches for high-redshift type II quasars. 
It could be that type II quasars exist. 
If you look at CIV/MgII the narrow line components are very weak. 
So the contribution from the \ac{BLR} is very weak in luminous quasars, and you just won't see it even if the broad line region is obscured.
Findings in this paper seem to suggest that the static \ac{NLR} is very weak in luminous quasars. 

\todo{Wasn't too sure about what this section was trying to say... Have you considered the \ac{SDSS} Type 2 samples from e.g. Alexandroff et al. ? (http://adsabs.harvard.edu/abs/2013MNRAS.435.3306A). I thought those were pretty luminous, narrow-line objects?}

\section{ICA}

Then having presented the main results, I would go on to discuss the limitations of the Gaussian approach - e.g. FeII can't be properly subtracted in many cases and sensitive to S/N - and use this as an intro to the much more flexible ICA method. You could then have a much briefer description of the ICA reconstructions and present this more as work in progress. You could show that your main results (as above) still hold with the ICA (e.g. Figs 1.15, 1.16, 1.17) and that this allows you to solve the FeII problem and push to lower SNR. Finally, you could discuss some of the potential improvements to the ICA components that would allow the derived line properties from the ICA to become even more robust. ICA works better at low S/N because we are effectively putting priors on the model parameters. 


The second model consists of six spectral components derived from an \ac{ICA} of a large sample of low-redshift \ac{AGN} with \ac{SDSS} spectra covering the same spectral region.
As we will demonstrate, a linear combination of these spectral components is able to reproduce the spectra around \hbns/[\ion{O}{III}] to a high degree of precision.  

\subsection{Model Two: Independent Component Analysis}

\ac{ICA} is a blind source separation technique for separating a signal in to linearly mixed statistically independent subcomponents. 
Unlike the more widely-used principle component analysis technique, \ac{ICA} produces non-negative components which allows for a physical interpretation of the components and weights.  
\ac{ICA} has been succesfully applied to model the spectra of emission-line galaxies \citep{allen13} and BAL quasars \citep{allen11}. 
The quasar spectra can be thought of as a set of observations, $\bm{x}$, which are made up of statistically independent components, $\bm{c}$, that are combined by some mixing matrix, $\bm{W}$:

\begin{equation}
    \bm{x} = \bm{W}\bm{c}
\end{equation}

\ac{ICA} reverses this process and describes how the observed data are generated. 
Both the independent components and the mixing matrix are unknown, but can be found by solving:

\begin{equation}
    \bm{c} = \bm{W}^{-1}\bm{x}.
\end{equation}

The components were solved for using a sample of 2,154 \ac{SDSS} quasars at redshifts XX. 
\todo{Ask Paul for details.}
At these redshifts the \ac{SDSS} spectrograph covers the rest-frame region XX-XX\AA\, where \hb and [\ion{O}{III}] lie. 
The individual spectra were first adjusted to give the same overall shape as a model quasar template spectrum.
Six positive independent components and four additional components that could be negative were found to be sufficient to reconstruct the spectrum, without over-fitting. 
Each quasar spectrum $x_j$ can then be represented as a linear combination of the independent components: 

\begin{equation}
    x_j = \sum_{i=1}^{10} c_{ij}W_{ij}
\end{equation}

\subsubsection{Fitting procedure}

Each of the individual \ac{ICA} components has been adjusted to give the same overall shape as a quasar template spectrum. 
We approximate the overall shape of this template by fitting a single power-law to emission line free windows at 4200-4230, 4435-4700 and 5100-5535 \AA. 
We then flatten each of the \ac{ICA} components by dividing by this power-law. 
An identical process is performed on each spectrum we fit, so that both the components and the spectrum to be fitted have essentially zero shape. 
For each quasar in our sample we perform a variance-weighted least-squares minimisation to determine the optimum value of the components weights.
The first six component weights are constrained to be non-negative, and the fit is done in logarithmic wavelength space, so that each pixel corresponds to a fixed velocity width.   
The relative shift of the \ac{ICA} components is also allowed to vary in the optimisation procedure, to account for errors in the systemic redshifts used to transform the spectra in to rest-frame wavelengths. 

\subsubsection{Quality of fits}

In general, the \ac{ICA} components do a remarkably good job at reconstructing the spectra of the objects in our sample. 
\todo{Is there some way to demonstrate/quantify this?}
For example, in J125141+080718 (discussed above), it does much better job at modelling the \ion{Fe}{II} emission than the \citet{boroson92} template. 
It is less sensitive to the spectral \ac{S/N}, and the component weights do not need to be constrained. 
It is therefore much simpler to apply than fitting multiple Gaussians. 

However, it does have its limitations. 
The components were calculated using a set of lower-redshift, lower-luminosity \ac{AGN}, and quasar spectra are known to vary systematically as a function of luminosity. 
For example, the [\ion{O}{III}] line is typically broader in more luminous quasars. 
Because there are so few objects with very broad [\ion{O}{III}] in the low-redshift sample, the \ac{ICA} reconstruction fails to reproduce the broadest [\ion{O}{III}] profiles in our sample. 

\subsection{Physical interpretation of \ac{ICA} components}

\begin{figure}
    \centering
    \includegraphics[width=0.8\textwidth]{figures/chapter04/mfica_components.pdf} 
    \caption{\hbns/[\ion{O}{III}] emission J002952+020607. The \ac{ICA} reconstruction is shown in black, and the spectrum in grey. The first three components, and the sum of components four, five and six are shown individually.}     
    \label{fig:mfica_components}
\end{figure}

Although the \ac{ICA} is analysis is not based on any physics,  there appears to be a direct correspondence between the individual components and the different emission features which contribute to the spectra (Fig.~\ref{fig:mfica_components}). 
This correspondence is summarised in Table~\ref{tab:icacomps}. 
The component $w_1$ seems to correspond to \ion{Fe}{II} emission, the components $w_2$ and $w_3$ to broad \hb emission, the components $w_4$ and $w_5$ to narrow [\ion{O}{III}] emission at the systemic redshift, and the component $w_6$ to broad, blueshifted [\ion{O}{III}] emission. 

\begin{table}
  \centering
  \small
  \caption{Physical interpretation of the \ac{ICA} components.}
  \label{tab:icacomps}
    \begin{tabular}{cc} 
    \hline
    Component & Origin \\
    \hline
    $w_1$& \ion{Fe}{II} \\
    $w_2$& \hbns \\
    $w_3$& \hbns \\
    $w_4$& [\ion{O}{III}] core \\
    $w_5$& [\ion{O}{III}] core \\
    $w_6$& [\ion{O}{III}] wing \\
    \hline
    \end{tabular}
\end{table} 

\subsubsection{Reconstructing the [\ion{O}{III}] profile}

\begin{figure}
    \centering
    \includegraphics[width=0.8\textwidth]{figures/chapter04/oiii_reconstruction.pdf} 
    \caption[{[\ion{O}{III}] emission in J002952+020607.}]{[\ion{O}{III}] emission in J002952+020607. The data is shown in blue, and the \ac{ICA} spectrum in grey. The first three \ac{ICA} components have been subtracted from both the \ac{ICA} composite and the data. The black curve shows the reconstructed [\ion{O}{III}] profile.}     
    \label{fig:oiii_reconstruction}
\end{figure}

In order to measure non-parametric line parameters, e.g. $v_{10}$, we must first reconstruct the [\ion{O}{III}] emission. 
It is fortunate that most of the [\ion{O}{III}] emission is in just three of the \ac{ICA} components; the remaining three contribute very little. 
Therefore, we can set the first three weights to zero to leave only the [\ion{O}{III}] emission. 
The four correction components are also included. 

We define the boundaries of [\ion{O}{III}]\l5008 as being between 4950 and 5500\AA. 
The blue limit is close to the peak of the [\ion{O}{III}]\l4960 line, and so to recover the intrinsic profile we instead use the blue wing of [\ion{O}{III}]\l4960. 
We use the emission from 4980-5050\AA, and from 4900-(4980-(5008.2-4960.3)). 
The blue window is then shifted by (5008.2-4960.3) to reconstruct the blue wing of the [\ion{O}{III}]\l5008 line. 
We then subtract a constant, because the flux does not always go to zero (suggests that there is probably flux which is not due to [\ion{O}{III}] emission in components four to six). 

An examples of a reconstructed [\ion{O}{III}] emission line is shown in Figure~\ref{fig:oiii_reconstruction}. 
\todoinline{At present I am summing the flux all the way from 4950\AA. However, this is quite a lot of flux to sum up, and we can't ascribe this flux to the wing of the [\ion{O}{III}] emission with any certainty. This is borne out by the fact that there are quite large differences between, for example, $v_{10}$ measured from the Gaussian fit and $v_{10}$ measured from the \ac{ICA} fit.} 

Unfortunately, there are systematic differences between the line-width estimates from the Gaussian reconstructions and the ICA reconstructions, particularly for broad-line objects.
The current way of doing the ICA reconstruction of the [\ion{O}{III}] line ignores any cross-talk between the components and there is potentially flux being ascribed to the line that could be coming from some other component. 
We can solve this by finding some more representative broad [\ion{O}{III}] lines in \ac{SDSS} from which to derive the components as well as producing a set of components for [\ion{O}{III}] only.
Therefore we don't use these reconstructions and leave this for future work. 

\subsection{\ac{ICA} fits}

\begin{figure}
    \includegraphics[width=\textwidth]{figures/chapter04/mfica_component_weights.pdf} 
    \caption[{The relative weight in each of the six positive \ac{ICA} components for the high-luminosity and low luminosity samples.}]{The relative weight in each of the six positive \ac{ICA} components for the high-luminosity (blue) and low luminosity samples (grey). In the high-luminosity sample \ion{Fe}{II} emission is stronger (component $w_1$). The core [\ion{O}{III}] emission (components $w_4$, $w_5$) is weaker but the strength of the blueshifted wing ($w_6$) is the same.}     
    \label{fig:mfica_component_weights}
\end{figure}

\begin{figure}
    \centering
    \includegraphics[width=\textwidth]{figures/chapter04/mfica_oiii_weight.pdf} 
    \caption[{he relative weight in the three \ac{ICA} components corresponding to [\ion{O}{III}] emission and the relative weight of the component most closely related to blueshifted [\ion{O}{III}] emission relative to all three [\ion{O}{III}] components.}]{The relative weight in the three \ac{ICA} components corresponding to [\ion{O}{III}] emission ({\em left}) and the relative weight of the component most closely related to blueshifted [\ion{O}{III}] emission relative to all three [\ion{O}{III}] components ({\em right}). [\ion{O}{III}] emission is weaker in the high-luminosity sample, but the relative contribution from the blueshifted component to the total [\ion{O}{III}] emission is higher.}     
    \label{fig:mfica_oiii_weight}
\end{figure}

\begin{figure}
    \centering
    \includegraphics[width=\columnwidth]{figures/chapter04/oiii_core_strength_blueshift.pdf} 
    \caption[{Weight in the [\ion{O}{III}] wing relative to the weight in the [\ion{O}{III}] core emission versus the strength of the core [\ion{O}{III}] emission.}]{Weight in the [\ion{O}{III}] wing relative to the weight in the [\ion{O}{III}] core emission versus the strength of the core [\ion{O}{III}] emission. The blue-asymmetry of the [\ion{O}{III}] emission increases as the strength of the core component decreases.}     
    \label{fig:oiii_core_strength_blueshift}
\end{figure}

In Figure~\ref{fig:mfica_component_weights} we show the relative weights of each of the six positive \ac{ICA} components. 
Also shown are the same measurements for a sample of low-redshift, low-luminosity AGN. 
We want to examine whether or not there are systematic differences between these two samples. 

We see that [\ion{O}{III}] core emission is weaker in the more luminous sample, but the strength of the wing component is similar. 
\citet{shen14} showed that the strength of the core [\ion{O}{III}] component decreases with quasar luminosity and optical \ion{Fe}{II} strength faster than the wing component, leading to overall broader and more blueshifted profiles as luminosity and \ion{Fe}{II} strength (or \ion{C}{IV} blueshift) increases. 
\citet{shen14} suggested that a stable \ac{NLR} is being removed by the outflowing material. 
Similarly, \citet{zhang11} found that the more the peak of the [\ion{O}{III}] line is blueshifted, the more the core component decreases dramatically, while the blue wing changes much less. 
Therefore, there is an anti-correlation between the strength of the core component and the relative strength of the wing component (Figure~\ref{fig:oiii_core_strength_blueshift}). 

To show this phenomenon more clearly, we plot the relative [\ion{O}{III}] strength and the [\ion{O}{III}] wing/core ratio in the high/low luminosity samples (Figure~\ref{fig:oiii_core_strength_blueshift}). 
We see that [\ion{O}{III}] is weaker in the high luminosity sample, but that the wing component is much stronger relative to the core component. 
\todo{Similar to behaviour of \ion{C}{IV}? Would suggests that the mechanism producing the two correlations is the same}. 

\subsubsection{\ac{EV1} correlations}

\begin{figure}
    \centering
    \includegraphics[width=\textwidth]{figures/chapter04/civ_blueshift_oiii_strength.pdf} 
    \caption[{The \ac{ICA} component weight $w_4$, which is a proxy for the strength of core [\ion{O}{III}], as a function of the \ion{C}{IV} blueshift.}]{The \ac{ICA} component weight $w_4$, which is a proxy for the strength of core [\ion{O}{III}], as a function of the \ion{C}{IV} blueshift. The \ion{C}{IV} blueshift is measured relative to the NIR \ac{ICA} redshift.}     
    \label{fig:civ_blueshift_oiii_strength}
\end{figure}

\begin{figure}
    \centering
    \includegraphics[width=\columnwidth]{figures/chapter04/mfica_composites.pdf} 
    \caption[{Median \ac{ICA}-reconstructed spectra as a function of the \ion{C}{IV} blueshift.}]{Median \ac{ICA}-reconstructed spectra as a function of the \ion{C}{IV} blueshift.}     
    \label{fig:mfica_composites}
\end{figure}

In Figure~\ref{fig:civ_blueshift_oiii_strength} we show how the [\ion{O}{III}] strength varies as a function of the \ion{C}{IV} blueshift. 
There is a very well defined relation: when \ion{C}{IV} is strongly blueshifted [\ion{O}{III}] is very weak. 
This is very similar to what we found when we used Gaussian functions to model the emission. 
The correlation between \ion{C}{IV} blueshift and [\ion{O}{III}] EQW is shown in a different way in Figure~\ref{fig:mfica_composites}. 
Here we divide our sample in to four bins according to the \ion{C}{IV} blueshift. 
From the quasars in each \ion{C}{IV} blueshift bin we then find then generate an \ac{ICA} spectrum using the median weights from each quasar. 
The differences in the spectra as a function of the \ion{C}{IV} blueshift are dramatic. 
[\ion{O}{III}] becomes progressively weaker and more blueshifted.
The anti-correlation with \ion{Fe}{III} and the blue-ward \ion{Fe}{II} also clear, but there is no change in the redward \ion{Fe}{II}. 

\subsubsection{Updating \ac{EV1}}

The \ac{ICA} can be thought of as update on \ac{EV1}. 
The spectral diversity is encapsulated in the \ac{EV1} components. 
Most of the variance in \ac{EV1} is the anti-correlation between the strengths of [\ion{O}{III}] and \ion{Fe}{II}. 
So at one end we have objects with strong \ion{Fe}{II} and weak [\ion{O}{III}], and at the other end objects with weak \ion{Fe}{II} and strong [\ion{O}{III}]. 
Other properties, including the \ion{C}{IV} blueshift and the \hb \ac{FWHM}, also change systematically. 
Our work shows that the \ac{ICA} component weights change systematically along the \ac{EV1} sequence. 

\todo{Just present this as an idea for future work right at the end rather than having this sandwiched in the middle.} 

Accurate systemic redshift estimates are essential in a number of applications, and researchers have devoted a large amount of telescope time to obtaining near-infrared spectra to access [\ion{O}{III}] for this purpose. 
HI, CO and absorption line measures of the host galaxy rest frame suggest that [\ion{O}{III}] usually gives consistent results within 200\kms (de Robertis 1985; Whittle 1985; Wilson \& Heckman 1985; Condon et al. 1985; Stripe 1990; Alloin et al. 1992; Evans et al. 2001).  
However, our work shows that at high luminosities this can result in large errors (profile can be dominated by blueshifted component, \ion{Fe}{II} emission can be improperly subtracted, or [\ion{O}{III}] might not be detected at all. 



The size of the narrow line region is roughly expected to scale as $L^{0.5}$ \citep[e.g.][]{netzer04}. 
However, for high luminosity quasars with strong [\ion{O}{III}] this gives \ac{NLR} sizes which are unreasonably large \citep[$\sim$100 kpc;][]{netzer04}. 

\todoinline{See extra text from Brotherton paper. I could be confused here, but I think the Netzer argument goes that the nlr size increase with luminosity because there are more ionising photons. but then you run out of nlr to ionise. the luminosity of the quasar keeps increasing but the luminosity of the nlr flattens out. so the eqw starts to decrease. but we see a huge scatter in eqw at high luminosities. we can relate this to the \ion{C}{IV} blueshift, which I don't think Netzer will have been able to.}



\section{Description of catalogue}

\begin{table}
  \centering
  \small
  \caption{The format of the table containing the emission line properties from our parametric model fits. \todoinline{Finish table}}
  \label{tab:specmeasure}
  \centering
    \begin{tabular}{cccc} 
    \hline
    Column & Name & Units & Description \\ 
    \hline
    1 & NAME & & Catalogue name \\
    2 & OIII\_V$10$ & \kms & $v_{10}$ \\
    \hline
    \end{tabular}
\end{table}


\todoinline{Move table code from OIIIProperties.ipynb to code/. Make sure I'm happy with flags.}

\todoinline{Make table of OIII line properties and provide code to turn these in to spectrum. When OIII\_Broad\_OFF just set amplitude of second component to zero. Check none with FIX\_OIII\_PEAK\_RATIO=True.}

\begin{itemize}
    
  \item[2] Unique ID: QSOXXX.
  
  \item[3-16] $v_{5}$, $v_{10}$, $v_{25}$, $v_{50}$, $v_{75}$, $v_{90}$ and $v_{95}$ velocity of [\ion{O}{III}], relative to [\ion{O}{III}] peak, and their errors, in \kms.  

  \item[17-18] Systemic redshift measured at [\ion{O}{III}] peak wavelength, and its error. 

  \item[19-24] $w_{50}$ ($\equiv v_{75} - v_{25}$), $w_{80}$ ($\equiv v_{90} - v_{10}$) and $w_{90}$ ($\equiv v_{95} - v_{5}$) velocity width of [\ion{O}{III}], and their errors, in \kms.

  \item[25-26] Dimensionless asymmetry of [\ion{O}{III}] ($A = (v_{90} + v_{10}) / (v_{90} - v_{10})$), and its error.  

  \item[27-28] Rest-frame [\ion{O}{III}] EQW, and its error, in \AA.

  \item[29-30] [\ion{O}{III}] luminosity, and its error, in \ergs. 

  \item[31-32] 4434 - 4684 \AA rest-frame \ac{EQW} of \ion{Fe}{II}, and its error, in \AA. % Not the same as used in EV1 plots 

  \item[33-34] Velocity of \hb peak, relative to [\ion{O}{III}] peak, in \kms, and its error. 

  \item[35-36] Velocity of \ha peak, relative to [\ion{O}{III}] peak, in \kms, and its error. 

  \item[37-39] Redshift of \hb, and its error.

  \item[40-41] Redshift of \ha, and its error.

  \item[37] Reduced $\chi$-squared from [\ion{O}{III}]/\hb fit. 

  \item[38] 5$\mu$m luminosity, in \ergs. Luminosity is derived by linearly interpolating between WISE magnitudes. 

  \item[39] \ion{Fe}{II} flag. 

  \item[40] [\ion{O}{III}] \ac{EQW} flag. 

  \item[41] [\ion{O}{III}] bad fit flag (mention in text)

  \item[42] Extreme [\ion{O}{III}] flag  

  \item[43] Reduced $\chi$-squared in fit to \ha. 

  \item[43] CIV v50 relative to OIII peak velocity.

  \item[] OIII\_5007\_EQW\_MEAN, OIII\_5007\_EQW\_STD

  \item[] EBV\_SDSS\_SPEC, EBV\_MAG 

  \item[] kMag, kSDSS, k




\end{itemize}





