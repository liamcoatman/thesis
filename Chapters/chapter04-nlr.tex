% !TEX root = ../main.tex

%************************************************
\chapter{Quasar-driven outflows of ionised gas}
\label{ch:nlr} 
%************************************************

\section{[\ion{O}{III}] as a probe of quasar-driven outflows}

X-ray and ultra-violet spectroscopy has revealed high-velocity outflows to be nearly ubiquitous in high accretion rate quasars.
Strong evidence for high-velocity outflows in the vicinity of quasars include BALs, NALs and blueshifted emission-lines. 
These observations suggest that the energy released by quasars can have a dramatic effect on their immediate environments. 

Quasars driving powerful outflows over galactic scales is a central tenet of galaxy evolution models involving `quasar' feedback \citep[e.g.][]{silk98,king03,dimatteo05,king15}.
In recent years, a huge amount of resources have been devoted to searching for observational evidence of this phenomenon.  
This has resulted in recent detections of outflows in quasar-host galaxies using tracers of atomic, molecular, and ionised gas \citep[e.g.][]{nesvadba06,arav08,nesvadba08,moe09,alexander10,dunn10,feruglio10,nesvadba10,alatalo11,harrison12,harrison14,cimatti13,rupke13,veilleux13,cicone14,nardini15}.  

One particularly successful technique has been to use forbidden quasar emission-lines to probe the dynamical state of the gas in the quasar host-galaxy. 
Forbidden lines are suppressed by collisional de-excitation in higher-density environments and are only produced in low-density gas.
Therefore, even without spatial information, [\ion{O}{III}] gas is generally assumed to be extended over kilo-parsec scales. 
Because of its high equivalent width, [\ion{O}{III}]\l$5008$ is the most studied of the narrow quasar emission-lines. 
The [\ion{O}{III}] emission is found to consist of two distinct components: a narrow, `core' component, with a velocity close to the systemic redshift of the host-galaxy, and a broader `wing' component, which is normally blueshifted. 
The core component is dominated by the gravitational potential of the host-galaxy.
However, the velocity-width of the wing is much too broad for the gas to be in dynamical equilibrium with the host-galaxy \citep[e.g.][]{liu13} and the general consensus is that it is tracing high-velocity outflowing gas. The receding side of the outflow may be obscured due to the presence of dust, either in the outflow or elsewhere.
Hence only the near-side of the outflow, which is blueshifted, is generally observed. 
The relative balance between the core and wing components varies significantly from object to object, and governs the width and asymmetry of the overall [\ion{O}{III}] emission profile \citep[e.g.][]{shen14}. 

Observations of broad velocity-widths and blueshifts in narrow emission-lines stretch back several decades \citep[e.g.][]{weedman70,stockton76,heckman81,veron81,feldman82,heckman84,vrtilek85,whittle85,boroson92}. 
However, these studies rely on small samples, which are often unrepresentative of the properties of the quasar population. 
More recently, the advent of large optical spectroscopic surveys (e.g. SDSS) have facilitated studies of the NLR in tens of thousands of quasars \citep[e.g.][]{boroson05,greene05a,zhang11,mullaney13,zakamska14,shen14}. 
This has provided constraints on the prevalence and drivers of ionised outflows.   
At the same time, there is strong evidence from spatially resolved spectroscopy that these outflows are extended over galaxy scales \citep[e.g.][]{greene09,greene11,harrison12,hainline13,harrison14}. 

However, these studies do not cover the redshift range when star formation and BH accretion peaked ($2 \lesssim z \lesssim 4$), which is when the $M_{\mathrm BH}-\sigma$ relation is thought to have been established. 
At these redshifts bright optical emission-lines including the [\ion{O}{III}] doublet are redshifted to near-infrared wavelengths, where observations are far more challenging. 
As a consequence, studies at high redshifts have typically relied on relatively small numbers of objects \citep[e.g.][]{netzer04,sulentic04,shen16a}.
These studies find [\ion{O}{III}] to be broader in more luminous quasars, with velocity-widths $\gtrsim1000$\,\kms\, common \citep[e.g.][]{netzer04,kim13,brusa15,shen16a}.  
These findings suggest that quasar efficiency in driving galaxy-wide outflows increases with luminosity \citep[e.g.][]{netzer04,nesvadba08,kim13,brusa15,carniani15,perna15,bischetti16}. 
The fraction of objects with very weak [\ion{O}{III}] emission also appears to increase with redshift and/or luminosity \citep[e.g.][]{netzer04}. 

In this chapter, we analyse the [\ion{O}{III}] properties of a sample of $354$ high-luminosity, redshift $1.5 < z < 4$ quasars.
To date, this is the largest study of the NLR properties of high redshift quasars. 

\section{Constructing a sample with [\ion{O}{III}] spectra}

From our near-infrared spectroscopic catalogue (Chapter~\ref{ch:nirsample}), we have selected $354$ quasars which have spectra covering the [\ion{O}{III}] doublet. 
The broad Balmer \hb line has also been observed for all but two of the sample. 
For $165$ quasars, the spectra extend to the broad \ha emission-line at $6565$\,\AA, and in $260$ objects optical spectra, including \ion{C}{IV}, are also available (mostly from SDSS/BOSS). 
The sample covers a wide range in redshifts ($1.5 \lesssim z \lesssim 4$) and luminosities ($45.5 \lesssim \log L_{\mathrm Bol} \lesssim 49$\,\ergs). 
The spectrographs and telescopes used to obtain the near-infrared spectra are summarised in Table~\ref{tab:specnums_ch4}.

\begin{table}
  \centering
  \footnotesize 
  \caption{The numbers of quasars with [\ion{O}{III}] line measurements and the spectrographs and telescopes used to obtain the near-infrared spectra.}
  \label{tab:specnums_ch4}
    \begin{tabular}{ccc} 
    \hline
    Spectrograph & Telescope & Number \\
                 &           & \\
    \hline
    FIRE         & MAGELLAN  & $31$ \\
    GNIRS        & GEMINI-N  & $28$ \\
    ISAAC        & VLT       & $7$ \\
    LIRIS        & WHT       & $7$ \\
    NIRI         & GEMINI-N  & $29$ \\
    NIRSPEC      & Keck II   & $3$ \\
    SINFONI      & VLT       & $80$ \\
    SOFI         & NTT       & $76$ \\
    TRIPLESPEC   & ARC-$3.5$m  & $27$ \\
    TRIPLESPEC   & P$200$      & $45$ \\
    XSHOOTER     & VLT       & $21$ \\
    \hline
    \multicolumn{2}{c}{Total} & $354$ \\
    \hline
    \end{tabular}
\end{table} 

\section{Spectral measurements}

In this section, we describe how emission-line parameters are derived. 
Our approach is to model the spectra using a power-law continuum, an empirical \ion{Fe}{II} template (taken from \citealt{boroson92}) and multiple Gaussian emission components.
Non-parametric properties are then derived from the best-fitting model. 
This approach, which is commonly adopted in the literature \citep[e.g.][]{shen11,shen12,shen16a}, is more robust when analysing spectra with limited S/N (in comparison to measuring line properties directly from the data) and allows adjacent emission-lines to be de-blended.

The same approach was used to model the \hbns/[\ion{O}{III}] complex in Chapter~\ref{ch:bhmass}. 
However, a number of small adjustments have been made to the model. 
\ha emission-line properties (used to estimate the quasar systemic redshift) are also re-derived in this chapter using a slightly modified model to the one adopted in Chapter~\ref{ch:bhmass}. 
\ion{C}{IV} emission-line properties (used to infer the strength of BLR outflows) are taken directly from Chapter~\ref{ch:bhmass}. 

\subsection{Transforming spectra to rest-frame wavelengths}

Before a spectrum can be modelled, it must first be transformed to the rest-frame of the quasar.  
The redshift used in this transformation is either derived from the peak of the broad \ha emission ($\sim40$ per cent of our sample), from the peak of the broad \hb emission ($\sim40$ per cent) or from the peak of the narrow [\ion{O}{III}] emission ($20$ per cent).
The rest-frame transformation is only required to be accurate to within $\sim1000$\,\kms\, of the true systemic redshift for our fitting procedure to function. 
In later sections, more precise estimates of the systemic redshift will be calculated using our parametric model fits. 

\subsection{Removing \ion{Fe}{II} emission}
\label{sec:ch4-fe-removal}

\begin{figure}
    \centering
    \includegraphics[width=\columnwidth]{figures/chapter04/example_spectrum_grid_extreme_fe_1.pdf} 
    \caption[{Spectra of the $24$ objects for which significant \ion{Fe}{II} emission is still present following our \ion{Fe}{II}-subtraction procedure.}]{Spectra of the $24$ objects for which significant \ion{Fe}{II} emission is still present following our \ion{Fe}{II}-subtraction procedure. Spectra have been smoothed via convolution with a $100$\,\kms\, Gaussian kernel. The vertical lines indicate the expected positions of the [\ion{O}{III}] doublet (which is generally very weak in these objects) with the systemic redshift defined using the peak of the broad \hb emission.}     
    \label{fig:bad_fe}
\end{figure}

\begin{figure}
\ContinuedFloat
    \centering
    \includegraphics[width=\columnwidth]{figures/chapter04/example_spectrum_grid_extreme_fe_2.pdf} 
    \caption[]{Continued.}     
\end{figure}

Before \hbns/[\ion{O}{III}] is modelled, we first model and subtract the nearby continuum and \ion{Fe}{II} emission using the procedure described in Chapter~\ref{ch:bhmass}. 
We encountered $24$ objects for which the procedure failed to adequately remove the \ion{Fe}{II} emission from the spectra (Figure~\ref{fig:bad_fe}).  
In these objects the relative strengths of the \ion{Fe}{II} lines differ significantly from those of I Zw $1$, on which the \citet{boroson92} \ion{Fe}{II} template we use is based. 
The residual \ion{Fe}{II} emission is at rest-frame wavelengths very close to the laboratory wavelengths of the [\ion{O}{III}] doublet, which is generally very weak in these objects. 
As a result, the [\ion{O}{III}] line parameters we derive for these objects are unreliable. 
These objects are therefore flagged and excluded from our analysis in the remainder of this chapter (leaving $330$ objects in our sample). 

To illustrate the importance of the \ion{Fe}{II} subtraction procedure in reliably measuring [\ion{O}{III}] emission properties, we consider the object J$223819$-$092106$ (shown bottom row, centre in Figure~\ref{fig:bad_fe}). 
This object was also analysed by \citet{shen16a}, who reported the [\ion{O}{III}] emission to have an extreme redshift ($\sim7500$\,\kms) relative to the \citet{hewett10} systemic redshift.
However, our analysis suggests that the emission which was modelled by \citet{shen16a} as [\ion{O}{III}] is more likely to be poorly-subtracted \ion{Fe}{II}.  

\subsection{Modelling \hbns/[\ion{O}{III}]}
\label{sec:oiiimodel}

\begin{table}
  \centering
  \footnotesize 
  \caption{Summary of models used to fit the \hb emission, and the number of quasars each model is applied to.}
  \label{tab:hbmod}
    \begin{tabular}{ccc} 
    \hline
    Model & Fix centroids? & Number \\
    \hline
    $2$ broad Gaussians + $1$ narrow Gaussian & No & $9$ \\
    $2$ broad Gaussians & No  &  $295$ \\
    $2$ broad Gaussians & Yes &  $39$ \\
    $1$ broad Gaussian  & N/A &  $9$ \\
    \hline
    \end{tabular}
\end{table} 

In general, \hb is modelled with two Gaussians with non-negative amplitudes and FWHM greater than $1200$\,\kms.
In nine objects \hb is modelled with a single Gaussian and in $39$ objects \hb is modelled with two Gaussians, but the velocity centroids of the two Gaussians are constrained to be equal. 
These spectra generally have low S/N, and adding extra freedom to the model does not significantly decrease the  reduced-$\chi^2$.
In addition there are cases where the blue wing of the \hb emission is below the lower wavelength limit of the spectra; in these cases models with more freedom are insufficiently constrained by the data.    

Contributions to the \hb emission from the NLR is generally weak in our sample, and an additional Gaussian component to model this emission is not required for the vast majority of objects. 
In nine objects features in the model - data residuals suggest that a narrow emission component is significant, and an additional narrow Gaussian is included in the model for these quasars. 
If the NLR contribution to the \hb emission is significant in more of our sample, then measures of the \hb velocity-width will be biased to lower values. 
However, our systemic redshift estimates that use the peak of the \hb emission (Section~\ref{sec:ch4_redshifts}) will not be affected. 
The \hb models, and the numbers of quasars each model is applied to, are summarised in Table~\ref{tab:hbmod}. 

Each component of the [\ion{O}{III}] doublet is fit with one or two Gaussians, depending on the fractional reduced-$\chi^2$ difference between the one- and two-component models. 
Concretely, if the addition of the second Gaussian decreases the reduced-$\chi^2$ by more than 5 per cent then the double-Gaussian model is accepted.
One hundred and twenty-eight spectra are fit with a single Gaussian and $140$ with two Gaussians. 
The peak flux ratio of the [\ion{O}{III}] $4960$\,\AA\, and $5008$\,\AA\, components are fixed at the expected $1$:$3$ ratio and the width and velocity offsets are set to be equal\footnote{For J$003136$+$003421$, a significantly better fit ($\Delta \chi^2_{\nu} \sim 25\%$) is obtained when the peak flux ratio constraint relaxed; the peak ratio of the best-fitting model is $1$:$2.13$.}.

In $62$ objects with very weak [\ion{O}{III}] (mean ${\mathrm EQW}\sim2$\,\AA) we find that the Gaussian model has a tendency to fit features to the noise. 
In some cases this can lead to large errors on the [\ion{O}{III}] line properties. 
To avoid this problem, we instead fit a fixed [\ion{O}{III}] template to the spectra, with the normalisation of this template the only free-parameter in the fit.
This template is generated by running our line-fitting routine on a median composite spectrum that we have constructed from the $268$ quasars with reliable [\ion{O}{III}] line measurements.  
The spectra used to construct the composite were first de-redshifted and continuum- and \ion{Fe}{II}-subtracted.  

The models we use to fit [\ion{O}{III}], and the numbers of quasars each model is applied to, are summarised in Table~\ref{tab:oiiimod}.

\begin{table}
  \centering
  \footnotesize 
  \caption{Summary of models used to fit the [\ion{O}{III}] emission, and the number of quasars each model is applied to.}
  \label{tab:oiiimod}
    \begin{tabular}{cc} 
    \hline
    Model & Number \\
    \hline
    $2$ Gaussians &  $140$ \\
    $1$ Gaussian  &  $128$ \\
    Template &  $62$ \\
    \hline
    \end{tabular}
\end{table} 

In Figure~\ref{fig:example_spectrum_grid} we show example fits to eight objects. 
The median reduced-$\chi^2$ value in the whole sample is $1.31$ and, in general, there are no strong features observable in the spectrum-minus-model residuals.

\begin{figure}
    \centering
    \includegraphics[width=\textwidth]{figures/chapter04/example_spectrum_grid.pdf} 
    \caption[{Model fits to the continuum- and \ion{Fe}{II}-subtracted \hbns/[\ion{O}{III}] emission in eight quasars.}]{Example model fits to the continuum- and \ion{Fe}{II}-subtracted \hbns/[\ion{O}{III}] emission in eight quasars. The data is shown in grey, the best-fitting model in black, and the individual model components in orange. The peak of the [\ion{O}{III}] emission is used to set the redshift, and $\Delta{v}$ is the velocity shift from the rest-frame transition wavelength of \hbns. Below each spectrum we plot the data-minus-model residuals, scaled by the errors on the fluxes.}     
    \label{fig:example_spectrum_grid}
\end{figure}

\subsection{Modelling \hans}
\label{sec:hamodel}

\begin{table}
  \centering
  \footnotesize 
  \caption{Summary of models used to fit the \ha emission, and the number of quasars each model is applied to.}
  \label{tab:hamod}
    \begin{tabular}{cccc} 
    \hline
    Model     & Components & Fix centroids? & Number \\
    \hline
    1        & $1$ broad Gaussian  & N/A &  $8$ \\
    2        & $2$ broad Gaussians & Yes &  $47$ \\
    3        & $2$ broad Gaussians & No  &  $20$ \\
    4        & $2$ broad Gaussians + narrow Gaussians & Yes & $42$ \\
    5        & $2$ broad Gaussians + narrow Gaussians & No  & $48$ \\
    \hline
    \end{tabular}
\end{table} 

There are $165$ quasars in our sample with spectra covering the \ha emission-line. 
In Section~\ref{sec:ch4_redshifts}, we use the peak of the \ha emission as one estimate of the quasar systemic redshift. 
In this section, we describe how the \ha emission was modelled. 

The continuum emission is first modeled and subtracted using the procedure described in Section~\ref{sec:ha}. 
We then test five different models with increasing degrees of freedom to model the \ha emission. 
The models are summarised in Table~\ref{tab:hamod}. 
They are (1) a single broad Gaussian; (2) two broad Gaussians with identical velocity centroids; (3) two broad Gaussians with different velocity centroids; (4) two broad Gaussians with identical velocity centroids, and additional narrower Gaussians to model narrow \ha emission, and the narrow components of [\ion{N}{II}]\ll$6548,6584$ and [\ion{S}{II}]\ll$6717,6731$; (5) two broad Gaussians with different velocity centroids, and additional narrower Gaussians. 
If used, the width and velocity of all narrow components are set to be equal in the fit, and the relative flux ratio of the two [\ion{N}{II}] components is fixed at the expected value of $2.96$.

In order to determine which model is selected for each spectrum we use the following procedure.  
Each of the five models are fit to every spectrum and the reduced-$\chi^2$ recorded.
Initially, the model with the smallest reduced-$\chi^2$ is selected. 
We then measure how the reduced-$\chi^2$ changes as the complexity of the model is decreased (i.e. considering the models in Table~\ref{tab:hamod} in descending order). 
If it results in an increase in the reduced-$\chi^2$ which is less than $10$ per cent relative to the best fitting model, then the simpler model is selected.  

\subsection{Deriving emission-line properties from the best-fitting models}

All [\ion{O}{III}] line properties are derived from the [\ion{O}{III}]\l$5008$ peak, but, as described above, the kinematics of [\ion{O}{III}]\l$4960$ are constrained to be identical in our fitting routine. 

We do not attach any physical meaning to the individual Gaussian components used in the model. 
Decomposing the [\ion{O}{III}] emission into a narrow component at the systemic redshift and a lower-amplitude, blueshifted broad component is often highly degenerate and dependent on the spectral S/N and resolution. 
Furthermore, there is no theoretical justification that the broad component should have a Gaussian profile.  

We therefore choose to characterize the [\ion{O}{III}] line profile using a number of non-parametric measures, which are commonly used in the literature \citep[e.g.][]{whittle85,zakamska14,zakamska16}. 
A normalised cumulative velocity distribution is constructed from the best-fitting model, from which the velocities below which $5$, $10$, $25$, $50$, $75$, $90$, and $95$ per cent of the total flux accumulates can be calculated. 
These velocities are then adjusted so that the peak of the [\ion{O}{III}] emission is at $0$\,\kms. 

We calculate the velocity-width containing $90$ per cent of the flux $w_{90}$ by rejecting $5$ per cent of the flux in the blue and red wings of the profile ($w_{90}\equiv v_{95} - v_{5}$).
We also calculate $w_{80}$ ($\equiv v_{90} - v_{10}$) and $w_{50}$ ($\equiv v_{75} - v_{25}$).
$w_{90}$ is relatively most sensitive to the flux in the wings of the line, whereas $w_{50}$ is relatively most sensitive to the flux in the core.  
In terms of the FWHM, $w_{50} \simeq {\mathrm FWHM} / 1.746$, $w_{80} \simeq {\mathrm FWHM} / 0.919$, $w_{90} \simeq {\mathrm FWHM} / 0.716$, assuming a Gaussian line profile.  

All of the derived parameters we have calculated are summarised in Table~\ref{tab:nlr-specmeasure}. 

% remove page number from this page only
\afterpage{%
\thispagestyle{empty}
\begin{table}
  \centering
  \footnotesize
  \caption{The format of the table containing the emission-line properties from our parametric model fits. \todoinline{Available online.}}
  \label{tab:nlr-specmeasure}
  \centering
    \begin{tabular}{cccc} 
    \hline
    Column & Name & Units & Description \\ 
    \hline
    1 & UID & & Catalogue name \\
    2 & OIII\_V$5$ & \kms & [\ion{O}{III}] $v_{5}$ \\
    3 & OIII\_V$5$\_ERR & \kms & \\
    4 & OIII\_V$10$ & \kms & [\ion{O}{III}] $v_{10}$ \\
    5 & OIII\_V$10$\_ERR & \kms &  \\
    6 & OIII\_V$25$ & \kms & [\ion{O}{III}] $v_{25}$ \\
    7 & OIII\_V$25$\_ERR & \kms &  \\
    8 & OIII\_V$50$ & \kms & [\ion{O}{III}] $v_{50}$ \\
    9 & OIII\_V$50$\_ERR & \kms &  \\
    10 & OIII\_V$75$ & \kms & [\ion{O}{III}] $v_{75}$ \\
    11 & OIII\_V$75$\_ERR & \kms &  \\
    12 & OIII\_V$90$ & \kms & [\ion{O}{III}] $v_{90}$ \\
    13 & OIII\_V$90$\_ERR & \kms &  \\
    14 & OIII\_V$95$ & \kms & [\ion{O}{III}] $v_{95}$ \\
    15 & OIII\_V$95$\_ERR & \kms &  \\
    16 & OIII\_Z & & [\ion{O}{III}] redshift \\
    17 & OIII\_Z\_ERR & &  \\
    18 & OIII\_W$50$ & \kms & [\ion{O}{III}] $w_{50}$ \\
    19 & OIII\_W$50$\_ERR & \kms &  \\
    20 & OIII\_W$80$ & \kms & [\ion{O}{III}] $w_{80}$ \\
    21 & OIII\_W$80$\_ERR & \kms & \\
    22 & OIII\_W$90$ & \kms & [\ion{O}{III}] $w_{90}$ \\
    23 & OIII\_W$90$\_ERR & \kms & \\
    24 & OIII\_A & & [\ion{O}{III}] asymmetry \\
    25 & OIII\_A\_ERR & & \\
    26 & OIII\_EQW & \AA & [\ion{O}{III}] EQW \\
    27 & OIII\_EQW\_ERR & \AA & \\
    28 & OIII\_LUM & \ergs & [\ion{O}{III}] luminosity \\
    29 & OIII\_LUM\_ERR & \ergs & \\
    30 & EQW\_FE\_$4434$\_$4684$ & \AA & \ion{Fe}{II} EQW \\
    31 & EQW\_FE\_$4434$\_$4684$\_ERR & \AA & \\
    32 & HB\_VPEAK & \kms & \hb peak velocity \\
    33 & HB\_VPEAK\_ERR & \kms & \\
    34 & HA\_VPEAK & \kms & \ha peak velocity \\
    35 & HA\_VPEAK\_ERR & \kms & \\
    36 & HB\_Z & & \hb redshift \\
    37 & HB\_Z\_ERR & & \\
    38 & HA\_Z & & \ha redshift \\
    39 & HA\_Z\_ERR & & \\
    40 & OIII\_FE\_FLAG & & Bad \ion{Fe}{II} subtraction \\
    41 & OIII\_EXTREM\_FLAG & & Extreme [\ion{O}{III}] emission \\
    42 & BLUESHIFT\_CIV\_OIII & \kms & \ion{C}{IV} blueshift, relative to [\ion{O}{III}] \\
    43 & BLUESHIFT\_CIV\_OIII\_ERR & \kms &  \\
    \hline
    \end{tabular}
\end{table}
\clearpage
}
The columns are as follows: 

\begin{itemize}
    
  \item[1] Catalogue name. 

  \item[2-15] $v_{5}$, $v_{10}$, $v_{25}$, $v_{50}$, $v_{75}$, $v_{90}$ and $v_{95}$ velocity of [\ion{O}{III}], relative to [\ion{O}{III}] peak, $v_{\mathrm peak}$, and their errors, in \kms.  

  \item[16-17] Systemic redshift measured at [\ion{O}{III}] peak wavelength, and its error. 

  \item[18-23] $w_{50}$ ($\equiv v_{75} - v_{25}$), $w_{80}$ ($\equiv v_{90} - v_{10}$) and $w_{90}$ ($\equiv v_{95} - v_{5}$) velocity-width of [\ion{O}{III}], and their errors, in \kms.

  \item[24-25] Dimensionless [\ion{O}{III}] asymmetry $A$, and its error. The assymetry is define as 

  \begingroup\makeatletter\def\f@size{11}\check@mathfonts
   \begin{eqnarray}
    A = \frac{(v_{90} - v_{\mathrm peak}) - (v_{\mathrm peak} - v_{10})}{(v_{90} - v_{10})}.     
    \end{eqnarray}  
  \endgroup

  \item[26-27] Rest-frame [\ion{O}{III}] EQW, and its error, in \AA.

  \item[28-29] [\ion{O}{III}] luminosity, and its error, in \ergs. 

  \item[30-31] $4434$-$4684$\,\AA\, rest-frame \ion{Fe}{II} EQW, and its error, in \AA.  

  \item[32-33] Velocity of \hb peak, relative to [\ion{O}{III}] peak, and its error, in \kms. 

  \item[34-35] Velocity of \ha peak, relative to [\ion{O}{III}] peak, and its error, in \kms. 

  \item[36-37] Redshift of \hb peak, and its error.

  \item[38-39] Redshift of \ha peak, and its error.

  \item[40] \ion{Fe}{II} flag. When flag is $1$ \ion{Fe}{II}-subtraction procedure has been unsuccessful (Section~\ref{sec:ch4-fe-removal}).  

  \item[41] Extreme [\ion{O}{III}] flag. When flag is $1$ [\ion{O}{III}] emission is extremely broad and blueshifted (Section~\ref{sec:extreme_oiii}). 

  \item[42-43] \ion{C}{IV} $v_{50}$, relative to [\ion{O}{III}] peak, and its error, in \kms.

\end{itemize}

\subsection{Deriving uncertainties on parameters}

To estimate realistic uncertainties on emission-line parameters derived from the best-fitting model we use the same Monte Carlo approach described in Section~\ref{sec:ch3_param_errors}. 
Very briefly, random simulations of each spectrum are generated.
Our fitting-procedure is run on each simulated spectrum, and the errors on the line parameters are estimated by measuring the spread of the parameter distribution from the ensemble of simulations. 
In a slight modification of the procedure in Section~\ref{sec:ch3_param_errors}, the error is defined as half the $68$ ($84$ - $16$) percentile spread in the parameter values. 

\subsection{Flagging low EQW [\ion{O}{III}]}
\label{sec:ch4-loweqw}

\begin{figure}
    \centering
    \includegraphics[width=0.8\textwidth]{figures/chapter04/eqw_cut.pdf} 
    \caption[{Uncertainty in $v_{10}$ as a function of the EQW, for [\ion{O}{III}].}]{Uncertainty in $v_{10}$ as a function of the EQW, for [\ion{O}{III}]. Uncertainties in $v_{10}$ are large to the left of the vertical line, at $8$\,\AA. These objects are ignored in our subsequent analysis of the [\ion{O}{III}] line shape.}     
    \label{fig:eqw_cut}
\end{figure}

In Figure~\ref{fig:eqw_cut} we show how the uncertainty in [\ion{O}{III}] $v_{10}$ depends on the EQW. 
As the strength of [\ion{O}{III}] decreases, the average uncertainty in $v_{10}$ increases.
When the [\ion{O}{III}] ${\mathrm EQW} > 80$\,\AA, the mean uncertainty in $v_{10}$ is $50$\,\kms; this increases to $450$\,\kms\, when $10 < {\mathrm EQW} < 20$\,\AA. 
As the EQW drops below $8$\,\AA, uncertainties in $v_{10}$ become very large (exceeding $1000$\,\kms\, in many objects). 
Clearly, the emission-line is too weak for its shape to be reliably measured in many of these objects. 
Therefore, when the [\ion{O}{III}] line properties (e.g. velocity-width, centroid) are analysed in later sections, objects with ${\mathrm EQW} < 8$\,\AA\, will be excluded. 
This leaves $226$ quasars in the sample. 

\subsection{Reliability of systemic redshift estimates}
\label{sec:ch4_redshifts}

\begin{figure}
   \captionsetup[subfigure]{labelformat=empty}
    \centering
    \subfloat[\label{fig:redshift_comparison_a}]{}
    \subfloat[\label{fig:redshift_comparison_b}]{}
    \subfloat[\label{fig:redshift_comparison_c}]{}
    \subfloat[]{{\includegraphics[width=0.8\linewidth]{figures/chapter04/redshift_comparison.pdf} }}
    \caption[{Comparison of systemic redshift estimates using [\ion{O}{III}], \hb and \hans.}]{Comparison of systemic redshift estimates using [\ion{O}{III}], \hb and \hans. The probability density distributions are generated using a Gaussian kernel density estimator with $170$, $120$ and $140$\,\kms\, kernel widths for (a), (b) and (c) respectively. The short black lines show the locations of the individual points.}       
    \label{fig:redshift_comparison}
\end{figure}

In this section, we compare systemic redshift estimates based on [\ion{O}{III}], \hb and \hans. 
The wavelength of each of these lines is measured at the peak of the emission and this measurement is made using the best-fitting parametric model. 
In the case of the Balmer lines, this model includes both broad and (if present) narrow emission features. 

We compare systemic redshift estimates based on [\ion{O}{III}] and \hb (Figure~\ref{fig:redshift_comparison_a}), [\ion{O}{III}] and \ha (Figure~\ref{fig:redshift_comparison_b}) and \hb and \ha (Figure~\ref{fig:redshift_comparison_c}). 
We generate probability density distributions using a Gaussian kernel density estimator.
The bandwidth, which is optimised using leave-one-out cross-validation, is $170$, $120$ and $140$\,\kms\, for samples (a), (b) and (c) respectively. 

[\ion{O}{III}], \hb and \ha measurements are available for $226$, $418$ and $226$ objects respectively. 
We exclude [\ion{O}{III}], \hb and \ha measurements when the uncertainties on the peak velocities exceed $200$, $300$ and $200$\,\kms\, respectively. 
This excludes $4$, $6$ and $12$ per cent of the [\ion{O}{III}], \hb and \ha measurements respectively. 
We also exclude [\ion{O}{III}] measurements from 16 objects with very broad, blueshifted [\ion{O}{III}] emission that is strongly blended with the red wing of \hb (these objects are discussed in Section~\ref{sec:extreme_oiii}).
After these cuts, there are $182$, $85$ and $162$ objects being compared in samples (a), (b) and (c) respectively. 

The scatter between the different redshift estimates ($360$, $280$, and $230$\,\kms\, for (a), (b) and (c) respectively) is consistent with previous studies of redshift uncertainties from broad emission-lines \citep[e.g.][]{shen16b}. 
The systematic offset between the \ha and \hb estimates is effectively zero. 
However, the [\ion{O}{III}] redshifts appear to be systematically offset in comparison to both \ha and \hbns, in the sense that [\ion{O}{III}] is blueshifted in the rest-frame of the Balmer lines. 
This effect is strongest when [\ion{O}{III}] is compared to \hbns, in which case [\ion{O}{III}] is shifted by $\sim100$\,\kms\, to the blue.

\citet{hewett10} found that [\ion{O}{III}] was blueshifted by $\sim45$\,\kms\, relative to a rest-frame defined using photospheric \ion{Ca}{II}\ll$3935$,$3970$ absorption in the host galaxies of $z<0.4$ SDSS AGN and that [\ion{O}{III}] is increasingly blue-asymmetric at higher luminosities. 
Therefore, the $100$\,\kms offset we measure is consistent with \citet{hewett10} once the very different luminosities of the two samples are accounted for. 
\todo{Ask Paul if this is okay} 

\section{Results}

\subsection{Strength and kinematics of [\ion{O}{III}]}
\label{sec:ch4-basicresults}

\begin{figure}
    \captionsetup[subfigure]{labelformat=empty}
    \centering
    \subfloat[\label{fig:parameter_hists_a}]{}
    \subfloat[\label{fig:parameter_hists_b}]{}
    \subfloat[\label{fig:parameter_hists_c}]{}
    \subfloat[]{{\includegraphics[width=0.8\columnwidth]{figures/chapter04/parameter_hists.pdf} }}
    \caption[{Probability density distributions of the [\ion{O}{III}] parameters EQW (a), $w_{80}$ (b) and asymmetry $A$ (c).}]{Probability density distributions of the [\ion{O}{III}] parameters EQW (a), $w_{80}$ (b) and asymmetry $A$ (c), generated using Gaussian kernel density estimator. The $1200$\,\kms\, upper limit on the velocity-width of the Gaussian functions used to model [\ion{O}{III}] is responsible for the peak at $1200$\,\kms\, in (b).}     
    \label{fig:parameter_hists}
\end{figure}

In our sample of $354$ quasars we observe a significant diversity in [\ion{O}{III}] emission properties. 

The probability density distribution of the [\ion{O}{III}] EQW is shown in Figure~\ref{fig:parameter_hists_a}. 
The median of the distribution is $14$\,\AA\, and the $68$ percentile range is $3$ to $30$\,\AA.
The maximum EQW is $391$\,\AA.  
In $10$ per cent of the sample [\ion{O}{III}] is very weak, with ${\mathrm EQW} < 1$\,\AA.  

The median of the line-width (characterized by $w_{80}$ and shown in Figure~\ref{fig:parameter_hists_b}) is $1540$\,\kms\, and the $68$ percentile range is $950$ to $2100$\,\kms, with a minimum of $300$\,\kms\, and a maximum of $3200$\,\kms.
These line-widths are significantly broader than would be expected for static NLR gas, and suggest high-velocity outflowing gas is common in these objects. 

The [\ion{O}{III}] asymmetry is shown in Figure~\ref{fig:parameter_hists_c}. 
In $40$ per cent of the sample [\ion{O}{III}] is fit with a single Gaussian. 
The asymmetry is zero in this model and so these objects are excluded. 
For the [\ion{O}{III}] emission-lines modelled with two Gaussians, [\ion{O}{III}] is blue-asymmetric in $90$ per cent.
The median asymmetry is $-0.37$ and the $68$ percentile range is $-0.61$ to $-0.12$.
Again, the strong blue-asymmetries seen in these objects are highly suggestive that outflows are prevalent in these objects. 

We also find weak correlations between these three [\ion{O}{III}] parameters. 
The EQW is anti-correlated with both the line-width and asymmetry: as the [\ion{O}{III}] emission gets weaker it gets broader and more blue-asymmetric \citep[e.g.][]{shen14}.  

\subsection{Luminosity-dependence of [\ion{O}{III}] properties}

In this section, we compare our sample of luminous $2 \lesssim z \lesssim 4$ quasars to a sample of $z\lesssim1$ SDSS quasars in order to investigate the luminosity and redshift dependence of key [\ion{O}{III}] parameters. 
We use $20\,663$ quasars with [\ion{O}{III}] measurements from the \citet{shen11} catalogue. 
The median redshift of these objects is $0.55$ and the median bolometric luminosity is $10^{45.5}$\,\ergs.

\begin{figure}[t!]
\centering 
    \includegraphics[width=\columnwidth]{figures/chapter04/eqw_lum.pdf} 
    \caption[{The [\ion{O}{III}] EQW as a function of the quasar bolometric luminosity for the sample presented in this chapter (blue circles) and the low-$z$ SDSS sample (grey points and contours).}]{The [\ion{O}{III}] EQW as a function of the quasar bolometric luminosity for the sample of luminous quasars presented in this chapter and the $z\lesssim1$ SDSS sample. An upper limit at ${\mathrm EQW}=1$\,\AA\, indicates points with ${\mathrm EQW} < 1$\,\AA. The red line shows the median [\ion{O}{III}] EQW in luminosity as a function of luminosity. The average EQW decreases from $18$ to $12$\,\AA\, over the luminosity range considered. At the same time, the fraction of quasars with very weak [\ion{O}{III}] (${\mathrm EQW} < 1$\,\AA) is ten times higher in the luminous quasar sample.}     
    \label{fig:eqw_lum}
\end{figure}

In Figure~\ref{fig:eqw_lum} we show the [\ion{O}{III}] EQW as a function of the quasar bolometric luminosity. 
Bolometric luminosities are estimated from monochromatic continuum luminosities at $5100$\,\AA, using the correction factor given by \citet{richards06}. 
Considering only the objects for which [\ion{O}{III}] is detected with ${\mathrm EQW} > 1$\,\AA, we observe a modest decrease in the [\ion{O}{III}] EQW as the luminosity increases (from $18$\,\AA\, at $L_{\mathrm Bol}=10^{45.25}$\,\ergs to $12$\,\AA\, at $L_{\mathrm Bol}=10^{47.75}$\,\ergs. 
However, [\ion{O}{III}] ${\mathrm EQW} < 1$\,\AA\, in $10$ per cent of the luminous quasars, compared to just one per cent of the $z \lesssim 1$ SDSS sample.
This is explored further in Section~\ref{sec:ch4-civtrends}.  

Many authors have reported the [\ion{O}{III}] EQW to decrease with quasar luminosity \citep[e.g.][]{brotherton96,sulentic04,baskin05b,zhang11,stern12}.
The origin of this correlation - known as the [\ion{O}{III}] Baldwin effect \citep[e.g.][]{baldwin77} - has not been demonstrated conclusively. 
The size of the NLR is predicted to scale with the square root of the luminosity of the source of ionising photons \citep[e.g.][]{netzer90} and low-luminosity Seyfert galaxies appear to obey this relationship \citep[e.g.][]{bennert02}. 
Extrapolating this relationship to high luminosity quasars leads to the prediction of NLRs with galactic dimensions.
Under these conditions, the size of the NLR will be limited by the density and ionisation state in the NLR. 
In other words, the NLR can't continue to grow beyond the radius at which there is no longer gas available to be ionised and the luminosity of the NLR will saturate \citep[e.g.][]{hainline13,hainline14}. 

\begin{figure}[t!]
    \centering
    \includegraphics[width=\columnwidth]{figures/chapter04/lum_w80.pdf} 
    \caption[{}]{[\ion{O}{III}] velocity-width $w_{80}$ as a function of quasar bolometric luminosity. Objects with extreme [\ion{O}{III}] profiles (Section~\ref{sec:extreme_oiii}) are shown in red. The grey dots show $z\lesssim1$ SDSS quasars. The FWHM measurements given by \citet{shen11} have been converted into equivalent $w_{80}$ values by assuming $w_{80} \simeq {\mathrm FWHM} / 0.919$. The build-up of points at $w_{80}=1300$\,\kms\, is caused by the upper-limit $1200$\,\kms\, imposed by \citet{shen11} on the [\ion{O}{III}] FWHM. The average [\ion{O}{III}] velocity-width increases significantly with luminosity. \todoinline{Need to check 1+z in luminosity calculation.}} 
    \label{fig:lum_w80}
\end{figure}

In Figure~\ref{fig:lum_w80} we show that the [\ion{O}{III}] velocity-width is also strongly correlated with the quasar bolometric luminosity.
The typical [\ion{O}{III}] velocity-width increases from $440$\,\kms\, at $\log L_{\mathrm Bol}=45.5$\,\ergs\, to $1850$\,\kms\, at $\log L_{\mathrm Bol}=48$\,\ergs.  
This demonstrates that the highest velocity outflows are associated with the most luminous AGN which suggests that the outflows are driven by radiative forces. 

Considering only objects in a narrow luminosity range ($47 < \log L_{\mathrm Bol} < 47.5$\,\ergs) we observe no correlations between the redshift and either the [\ion{O}{III}] velocity-width or EQW.   
The lack of any evolution in typical [\ion{O}{III}] properties between $z=0$ and $z=1.5$ has previously been reported \citep[e.g.][]{harrison16}; our sample demonstrates that the [\ion{O}{III}] properties do not evolve from $z=1.5$ all the way to $z=4$. 

\subsection{EV$1$ trends in high-redshift quasars}

\begin{figure}[t!]
\centering 
    \includegraphics[width=\columnwidth]{figures/chapter04/ev1_lowz.pdf} 
    \caption[{EV$1$ parameter space.}]{The distribution of objects in the EV$1$ parameter space. The distribution of luminous quasars (shown using circles) is similar to the distribution of $z \lesssim 1$ SDSS quasars (shown using contours), with the displacement to higher \hb FWHM indicative of higher BH masses in the luminous sample.}      
    \label{fig:ev1_lowz}
\end{figure}

The FWHM of the broad \hb emission-line, the strength of [\ion{O}{III}] and the relative strengths of optical \ion{Fe}{II} and \hb have been identified as the features responsible for the largest variance in the spectra of AGN and form part of EV$1$ \citet{boroson92}.   
In Figure~\ref{fig:ev1_lowz} we show the [\ion{O}{III}] EQW as a function of the \hb FWHM and the optical \ion{Fe}{II} strength. 
The optical \ion{Fe}{II} strength is defined as the ratio of the \ion{Fe}{II} and \hb EQW, where the \ion{Fe}{II} EQW is measured between $4434$ and $4684$\,\AA.
There are $231$ objects in our sample with spectra that include \hbns, [\ion{O}{III}], and at least $>150$\,\AA\, of the $4434$-$4684$\,\AA\, \ion{Fe}{II} region.  
For comparison, $z\lesssim1$ SDSS quasars are also shown in Figure~\ref{fig:ev1_lowz}. 
 
In our sample, these parameters follow very similar correlations to what is observed at low-redshift.
In particular, we observe a strong anti-correlation between the [\ion{O}{III}] and \ion{Fe}{II} EQW.  
The \hb FWHM are displaced to higher values, which is consistent with the high-redshift, high-luminosity sample having larger BH masses. 
Thus, we confirm earlier results using much smaller samples that suggest that the same EV$1$ correlations exist in high-redshift quasars \citep[e.g.][]{netzer04,sulentic04,sulentic06,runnoe13,shen16a}.
This suggests that similar underlying physical processes govern the spectral properties of AGN and quasars over a wide range of redshifts and luminosities. 

\subsection{Connections with \ion{C}{IV} emission properties}
\label{sec:ch4-civtrends}

\begin{figure}[t!]
\centering 
    \includegraphics[width=0.9\textwidth]{figures/chapter04/ev1.pdf} 
    \caption[]{Parameter space of \ion{C}{IV} blueshift and EQW. Our sample is shown with points, and quasars from the full SDSS catalogue are shown with grey contours. The [\ion{O}{III}] EQW varies systematically with position in the \ion{C}{IV} blueshift-EQW parameter space.}      
    \label{fig:ev1}
\end{figure}

Like the EV$1$ parameter space, the \ion{C}{IV} blueshift and EQW are diagnostics that similarly span the diversity of broad emission-line properties in high redshift quasars \citep{sulentic07,richards11}. 
In Figure~\ref{fig:ev1} we show the [\ion{O}{III}] EQW as a function of the \ion{C}{IV} blueshift and EQW.

When [\ion{O}{III}] is strong, the \ion{C}{IV} blueshift is measured relative to the [\ion{O}{III}] peak. 
Otherwise, the \ion{C}{IV} blueshifted is measured relative to \hb or \hans.  
In Section~\ref{sec:ch4_redshifts} we found that redshifts measured from all three of these lines are consistent to within $\sim300$\,\kms, which is small in comparison to the dynamic range in \ion{C}{IV} blueshifts we see in Figure~\ref{fig:ev1}.
Also shown are the \ion{C}{IV} line parameters of $32\,157$ SDSS DR$7$ quasars at redshifts $1.6 < z < 3.0$. 
For this sample, systemic redshifts are taken from Allen \& Hewett (2017, in preparation). 

The [\ion{O}{III}] EQW decreases systematically from the small \ion{C}{IV} blueshift, large EQW region of the parameter space to the large \ion{C}{IV} blueshift, small EQW region.
In the top left of the distribution (\ion{C}{IV} blueshift $<1000$\,\kms, ${\mathrm EQW} > 60$\,\AA) the mean [\ion{O}{III}] EQW is $47$\,\AA; this drops dramatically to $6$\AA\, in the bottom right ((\ion{C}{IV} blueshift $>2000$\,\kms, ${\mathrm EQW} < 30$\,\AA). 

Qualitatively, the distribution of objects in the \hb FWHM - \ion{Fe}{II} strength EV$1$ parameter space (Figure~\ref{fig:ev1_lowz}) is very similar to the distribution of objects in the \ion{C}{IV} blueshift-EQW parameter space (Figure~\ref{fig:ev1}).
In Figure~\ref{fig:line_comparison_ha} we show that objects with large \ion{C}{IV} blueshifts also have narrow \ha emission-lines.
However, the converse is not true: many of the objects with narrow \ha emission-lines also have small \ion{C}{IV} blueshifts. 
In contrast, Figure~\ref{fig:ev1} demonstrates that the [\ion{O}{III}] EQW provides a less degenerate mapping between the EV$1$ and \ion{C}{IV} parameter spaces.

\begin{figure}
    \centering
    \includegraphics[width=\columnwidth]{figures/chapter04/civ_blueshift_oiii_eqw.pdf} 
    \caption[{[\ion{O}{III}] EQW as a function of the \ion{C}{IV} blueshift.}]{[\ion{O}{III}] EQW as a function of the \ion{C}{IV} blueshift. The [\ion{O}{III}] EQW is strongly anti-correlated with the \ion{C}{IV} blueshift. On the other hand, no strong luminosity-dependent trends (indicated by the colours of the points) are evident.}     
    \label{fig:civ_blueshift_oiii_eqw}
\end{figure}

The same data is shown in a two-dimensional projection of [\ion{O}{III}] EQW against \ion{C}{IV} blueshift in Figure~\ref{fig:civ_blueshift_oiii_eqw}. 
The luminosity of the quasars is indicated by the colour of the points. 
Both the [\ion{O}{III}] EQW and the \ion{C}{IV} blueshift are known to depend on the quasar luminosity. 
However, Figure~\ref{fig:civ_blueshift_oiii_eqw} demonstrates that the strong correlation between the \ion{C}{IV} blueshift and [\ion{O}{III}] EQW is clearly not being driven by the mutual dependence of these parameters on the luminosity. 

Blueshifted \ion{C}{IV} emission is thought to arise in a high-velocity accretion disc wind.
The strong anti-correlation between the \ion{C}{IV} blueshift and the [\ion{O}{III}] EQW suggests that these outflows are having a dramatic impact on gas extended over kilo-parsec scales in the NLR.
Dynamical time-scales for the impact of fast moving outflows even on large NLRs are very short: it would take $10^6$ years for an outflow travelling at $3000$\,\kms\, to reach $3$\,kilo-parsec. 
Therefore, if the BLR can break out in to the interstellar medium of the host-galaxy, it can sweep away the NLR gas on a relatively short time-scale.
Large \ion{C}{IV} blueshifts are rarer in lower-luminosity quasars. 
This could explain our observation that objects with very weak [\ion{O}{III}] ($EQW < 1$\,\AA) are ten times rarer in $z\lesssim$ SDSS quasars than in our sample of luminous quasars.   
\todo{Ask Paul: is this claim justified? Other explanations?}

\subsection{A link between BLR and NLR outflows}

\begin{figure}
\centering 
    \includegraphics[width=\columnwidth]{figures/chapter04/civ_blueshift_oiii_blueshift.pdf} 
    \caption[{The relation between the blueshifts of \ion{C}{IV} and [\ion{O}{III}].}]{Relationship between the \ion{C}{IV} ($v_{50}$(\ion{C}{IV}) - $v_{\mathrm peak}$([\ion{O}{III}])) and [\ion{O}{III}] ($v_{10}$([\ion{O}{III}]) - $v_{\mathrm peak}$([\ion{O}{III}])) blueshift.  The \ion{C}{IV} and [\ion{O}{III}] blueshifts are correlated with $\rho_{\mathrm S}=0.46$. This correlation is independent of the luminosity (indicated by the colour of the points). Objects with errors on the [\ion{O}{III}] and \ion{C}{IV} blueshifts exceeding $250$ or $125$\,\kms\, respectively are not shown in (c); these objects fall above the dashed line in (a) and (b). }     
    \label{fig:oiii_civ_blueshifts}
\end{figure}
 
In Figure~\ref{fig:oiii_civ_blueshifts} we show the [\ion{O}{III}] blueshift as a function of the \ion{C}{IV} blueshift.
[\ion{O}{III}] appears to be more blueshifted in quasars with large \ion{C}{IV} blueshifts.
Although the scatter is large, the correlation appears to be significant ($\rho_{\mathrm S}=0.46$, p-value $=6e\text{-}7$). 
This suggests a direct connection between the gas kinematics in the broad and narrow line regions. 

The [\ion{O}{III}] blueshift is defined as $v_{10}$([\ion{O}{III}]) - $v_{\mathrm peak}$([\ion{O}{III}]) and the \ion{C}{IV} blueshift is defined as $v_{50}$(\ion{C}{IV}) - $v_{\mathrm peak}$([\ion{O}{III}]).
In the previous section, we saw how [\ion{O}{III}] is very weak in quasars with \ion{C}{IV} blueshifts exceeding $\sim2000$\,\kms. 
The [\ion{O}{III}] blueshift cannot be reliably measured if the emission-line ${\mathrm EQW} \lesssim 8$ (Section~\ref{sec:ch4-loweqw}) and so this limits the dynamic range of \ion{C}{IV} blueshifts probed in Figure~\ref{fig:oiii_civ_blueshifts}. 

We do not show objects for which the errors on the [\ion{O}{III}] and \ion{C}{IV} blueshifts exceed $250$ or $125$\,\kms\, respectively. 
These objects, shown in the top two panels of Figure~\ref{fig:oiii_civ_blueshifts}, have a similar dynamic range to the main sample, meaning our results should not be biased by their exclusion.  
We also remove the objects with extreme [\ion{O}{III}] emission, because the systemic redshift determined from the peak of the [\ion{O}{III}] emission is strongly biased in these objects. 

We considered a number of alternative approaches to parametrising both the [\ion{O}{III}] line shape and the systemic redshift. 
Very similar trends are observed when the [\ion{O}{III}] line shape is parametrised using $v_{25} - v_{\mathrm peak}$, $v_{50} - v_{\mathrm peak}$, $w_{80} = v_{90} - v_{10}$, or the asymmetry $A$.
The same trend is also observed when the systemic redshift is defined using the peak of the \hb emission. 
We also demonstrate in Figure~\ref{fig:oiii_civ_blueshifts} that the correlation between the \ion{C}{IV} and [\ion{O}{III}] blueshifts is not driven by the luminosity. 

In Section~\ref{sec:ch4-basicresults} we reported an anti-correlation between the [\ion{O}{III}] EQW and asymmetry. 
[\ion{O}{III}] emission becomes broader and more blueshifted as the EQW decreases.
\citet{shen14} showed how the [\ion{O}{III}] EQW decreases as the optical \ion{Fe}{II} strength or luminosity increase. 
However, the amplitude of the core [\ion{O}{III}] emission decreases faster than the wing component.
A by-product of this effect is that the overall profile becomes broader and more blueshifted, as the broad wing component becomes more prominent. 
If the core [\ion{O}{III}] decreases faster than the wing component as the \ion{C}{IV} blueshift increases, this could also induce a correlation between the [\ion{O}{III}] and \ion{C}{IV} blueshifts.  
In this picture, the stable NLR is being removed by the outflowing material.
The blueshifted wing in the [\ion{O}{III}] emission has a different origin.  
We do not attempt to decompose the [\ion{O}{III}] emission in to broad and narrow components and so further work is required to determine whether this is the case for the \ion{C}{IV} blueshift. 

\subsection{Extreme [\ion{O}{III}] profiles}
\label{sec:extreme_oiii}

\begin{figure}
    \centering
    \includegraphics[width=\columnwidth]{figures/chapter04/example_spectrum_grid_extreme_oiii_1.pdf} 
    \caption[{Model fits to the continuum- and \ion{Fe}{II}-subtracted \hbns/[\ion{O}{III}] emission in $18$ quasars with extreme [\ion{O}{III}] emission profiles.}]{Model fits to the continuum- and \ion{Fe}{II}-subtracted \hbns/[\ion{O}{III}] emission in $18$ quasars with extreme [\ion{O}{III}] emission profiles. The data is shown in grey, the best-fitting model in black, and the individual model components in orange. The peak of the [\ion{O}{III}] emission is used to set the redshift, and $\Delta{v}$ is the velocity shift from the rest-frame transition wavelength of \hbns. Below each spectrum we plot the data- minus-model residuals, scaled by the errors on the fluxes.}     
    \label{fig:example_spectrum_grid_extreme_oiii}
\end{figure}

\begin{figure}
\ContinuedFloat
    \centering
    \includegraphics[width=\columnwidth]{figures/chapter04/example_spectrum_grid_extreme_oiii_2.pdf} 
    \caption[]{Continued.}     
\end{figure}

Figure~\ref{fig:example_spectrum_grid_extreme_oiii} shows the spectra of $18$ objects which we visually identified as having broad, blueshifted [\ion{O}{III}] emission which is heavily blended with the red wing of \hbns. 
Because the emission is so heavily-blended, it is difficult to determine unambiguously what combination of \hbns, [\ion{O}{III}] and \ion{Fe}{II} is responsible for the unusual plateau-like emission observed in these objects. 
Therefore, uncertainties on the [\ion{O}{III}] emission properties are generally high in these objects. 

In Figure~\ref{fig:lum_w80} we show that the optical luminosities of all of these objects are larger than the sample median.
The mean luminosity of the quasars with extreme [\ion{O}{III}] emission is $10^{47.9}$\,\ergs, compared to $10^{47.2}$\,\ergs for the rest of the sample. 
Figure~\ref{fig:lum_w80} demonstrates that these quasars occupy a unique region of the $w_{80}$-L$_{\mathrm Bol}$ parameter space: $w_{80}\gtrsim1500$\,\kms\, and L$_{\mathrm Bol}\gtrsim10^{47.5}$\,\ergs. 

These [\ion{O}{III}] emission-lines are similar to the lines observed in four extremely dust-reddened quasars at $z\sim2$ recently identified by \citet{zakamska16}. 
The extreme nature of the [\ion{O}{III}] emission in these objects led \citet{zakamska16} to propose that these objects are being observed transitioning from a dust-obscured, star-burst phase to a luminous, blue quasar \citep[e.g.][]{sanders88}.
Punctuated fuelling episodes, e.g. driven by galaxy mergers, satellite accretion and even secular processes,
will lead to AGN experiencing activity-, outflow- and obscuration-dominated cycles with some overlap between phases. 
This suggests that these extreme outflows signatures are only associated with rare/short-lived phases in the AGN cycle represented by the reddened quasars and are not ubiquitous in the luminous AGN population.
The early phase of feedback is likely to highly obscured by dusty inflowing material \citep[e.g.][]{haas03}.
Quasar-driven outflows will eventually blow away the obscuring dust, revealing a luminous, relatively un-obscured quasars.  

We note that six of these objects are found within the foot-print of the FIRST radio survey. 
Of these four are radio loud (three core-dominated and one lobe-dominated). 
Given the fraction of radio-loud objects in our sample is $11$ per cent, the binomial probability of finding four radio-loud objects in our sample of six is $0.1$ per cent. 
In general, the differences between the $30$ radio-loud and $197$ radio-quiet sources in our sample are not large. 
The radio-loud objects have slightly broader ($w_{80}\simeq1600/1480$\,\kms) and weaker (${\mathrm EQW}\simeq14.5/13.9$\,\AA) [\ion{O}{III}]. 
However, these modest differences could be explained by the higher luminosity of the radio-loud objects ($L_{\mathrm Bol}\simeq10^{47.5/47.1}$\,\ergs).
In light of this, the high fraction of radio-loud objects amongst the quasars with broad and peculiar [\ion{O}{III}] emission is interesting. 
The disturbed gas kinematics could be induced by radio emission. 
Relativistic jets are known to accelerate gas to velocities of up to a few thousand \kms\, \citep[e.g.][]{nesvadba06,nesvadba08}.

\begin{figure}
\centering 
    \includegraphics[width=0.8\textwidth]{figures/chapter04/fivemicron_w90.pdf} 
    \caption[{}]{}     
    \label{fig:fivemicron_w90}
\end{figure}

\section{Mass outflow rate and kinetic power}

READ ZAKAMSKA DISCUSSION

\section{Mean field independent component analysis}

Blind source separation (BSS) techniques can generate a set of component spectra that can then be combined with varying weights to reconstruct individual input spectra.
Principle component analysis (PCA) is one example of a BSS technique that has been applied extensively to analyse astronomical spectra \citep[e.g.][]{mittaz90,francis92,yip04}. 
However, while input spectra can be reconstructed from the PCA-derived components, in general it is not possible to physically interpret the individual components. 
Mean field independent component analysis (MFICA) is a different BSS technique. 
\citet{allen13} used this technique to analyse the SDSS spectra of emission-line galaxies and demonstrated its effectiveness in identifying distinct emission sources in the spectra. 

In this chapter, we use a set of ten MFICA-derived spectral components to reconstruct the optical spectra of luminous quasars.
The set of weights measured for each quasar provide a compact representation of its spectral properties.  
The distributions of these weights in the whole sample reveal many of the results already described in this chapter.
Furthermore, by equating components with physical properties of interest we can use the component weights in place of more commonly used emission-line parameters to understand the physical processes occurring in these quasars. 
At the same time, with the MFICA-derived components we are able to extend our analysis to a lower S/N regime than is possible with a more traditional approach (i.e. fitting multiple Gaussian components).

At the present time, there are two significant issues with the MFICA analysis. 
The first is the limited spectral diversity of the objects used to derive the components. 
The second is cross-talk between the spectral components.
However, both of these issues are straightforward to overcome and, as we will demonstrate below, the intial results are very promising.   

\subsection{Generating the spectral components}

We use a set of ten spectral components that have been generated using MFICA from a carefully selected sample of $z \lesssim 1$ SDSS quasars\footnote{This was done by Prof. Paul Hewett}.
The MFICA components were generated in the rest-frame interval $4000$-$5600$\,\AA.
This restricted the redshift range of the SDSS spectra that can be used to XX.  
A sample of $2,154$ SDSS quasars was selected.
Six positive independent components and four lower-amplitude `correction' components that could be negative were found to be sufficient to reconstruct the spectra. 
\todo{Ask Paul for details.} 

As we showed in Section~XX, [\ion{O}{III}] emission properties are strongly luminosity dependent.
Specifically, [\ion{O}{III}] is broader, weaker and more asymmetric in more luminous quasars.  
There are insufficient numbers of objects in the $z<1$ SDSS sample with these characteristics. 
As a result, the MFICA-derived spectral components are unable to accurately reconstruct the very broad [\ion{O}{III}] emission-lines seen in luminous quasars. 
We can solve this by finding some more representative broad [\ion{O}{III}] lines in SDSS from which to derive the components. 

\subsection{Reconstructing input spectra}

\begin{figure}[t!]
    \centering
    \includegraphics[width=0.8\textwidth]{figures/chapter04/mfica_components.pdf} 
    \caption{\hbns/[\ion{O}{III}] emission J$002952$+$020607$. The MFICA reconstruction is shown in black, and the spectrum in grey. The first three components, and the sum of components four, five and six are shown individually.}     
    \label{fig:mfica_components}
\end{figure}

Input spectra are reconstructed using a $\chi^2$ minimisation to determine the optimum set of component weights. 
Each of the individual MFICA components has been adjusted to give the same overall shape as a quasar template spectrum. 
We approximate the overall shape of this template by fitting a single power-law to emission-line free windows at $4200$-$4230$, $4435$-$4700$ and $5100$-$5535$\,\AA. 
We then flatten each of the MFICA components by dividing by this power-law. 
An identical process is performed on each spectrum we fit, so that both the components and the spectrum to be fitted have essentially zero large-scale slope. 
For each quasar in our sample we perform a variance-weighted least-squares minimisation to determine the optimum value of the components weights.
The first six component weights are constrained to be non-negative, and the fit is performed in logarithmic wavelength space, so that each pixel corresponds to a fixed velocity-width.   
The relative shift of the MFICA components is also allowed to vary in the optimisation procedure, to account for errors in the systemic redshifts used to transform the spectra into rest-frame wavelengths. 

In general, the components are able to accurately reproduce the spectra in our sample. 
We find the components are able to accurately reproduce the \ion{Fe}{II} emission in many of the objects for which the \citet{boroson92} template was a poor match. 
However, as discussed above, some of the broader [\ion{O}{III}] lines not well reproduced. 

\subsection{Interpretation of individual components}

An example of this reconstruction is shown in Figure~\ref{fig:mfica_components}. 
The individual spectral components relate to the underlying physical constituents of the quasar. 
This correspondence is summarised in Table~\ref{tab:icacomps}. 
The component $w_1$ seems to correspond to \ion{Fe}{II} emission, the components $w_2$ and $w_3$ to broad \hb emission, the components $w_4$ and $w_5$ to narrow [\ion{O}{III}] emission at the systemic redshift, and the component $w_6$ to broad, blueshifted [\ion{O}{III}] emission. 

\begin{table}[t!]
  \centering
  \footnotesize 
  \caption{Physical interpretation of the ICA components.}
  \label{tab:icacomps}
    \begin{tabular}{cc} 
    \hline
    Component & Origin \\
    \hline
    $w_1$& \ion{Fe}{II} \\
    $w_2$& \hbns \\
    $w_3$& \hbns \\
    $w_4$& [\ion{O}{III}] core \\
    $w_5$& [\ion{O}{III}] core \\
    $w_6$& [\ion{O}{III}] wing \\
    \hline
    \end{tabular}
\end{table} 

\subsection{Reconstructing the [\ion{O}{III}] emission}

If the [\ion{O}{III}] emission is confined to a relatively small number of components can then recover emission. 
Discuss issue with cross talk. 
In the previous approach we measured non-parametric line parameters from the Gaussian model. 
We would like to be able to take a similar approach here. 
This requires us to first reconstruct the [\ion{O}{III}] emission. 
The vast majority of the [\ion{O}{III}] emission is in just three of the ICA components; the remaining three contribute very little. 
Therefore, we can set the first three weights to zero to leave only the [\ion{O}{III}] emission. 
However, there is some cross-talk between components, particularly in the wings of the line. 
This is something we can address in the future (ask Paul how). 

\subsection{Some results}

\begin{figure}
\centering 
    \includegraphics[width=\textwidth]{figures/chapter04/mfica_component_weights.pdf} 
    \caption[{The relative weight in each of the six positive ICA components for the high-luminosity and low luminosity samples.}]{The relative weight in each of the six positive ICA components for the high-luminosity (blue) and low luminosity samples (grey). In the high-luminosity sample \ion{Fe}{II} emission is stronger (component $w_1$). The core [\ion{O}{III}] emission (components $w_4$, $w_5$) is weaker but the strength of the blueshifted wing ($w_6$) is the same and so the relative contribution from the blueshifted component to the total [\ion{O}{III}] emission is higher.}     
    \label{fig:mfica_component_weights}
\end{figure}


In Figure~\ref{fig:mfica_component_weights} we show the relative weights of each of the six positive ICA components. 
Also shown are the same measurements for a sample of low-redshift, low-luminosity AGN. 

\begin{figure}
    \centering
    \includegraphics[width=\textwidth]{figures/chapter04/civ_blueshift_oiii_strength.pdf} 
    \caption[{The ICA component weight $w_4$, which is a proxy for the strength of core [\ion{O}{III}], as a function of the \ion{C}{IV} blueshift.}]{The ICA component weight $w_4$, which is a proxy for the strength of core [\ion{O}{III}], as a function of the \ion{C}{IV} blueshift. The \ion{C}{IV} blueshift is measured relative to the near-infrared ICA redshift. \todoinline{Don't need to show luminosity}}     
    \label{fig:civ_blueshift_oiii_strength}
\end{figure}

\begin{figure}
    \centering
    \includegraphics[width=\columnwidth]{figures/chapter04/mfica_composites.pdf} 
    \caption[{Median ICA-reconstructed spectra as a function of the \ion{C}{IV} blueshift.}]{Median ICA-reconstructed spectra as a function of the \ion{C}{IV} blueshift.}     
    \label{fig:mfica_composites}
\end{figure}

There is a very well defined relation: when \ion{C}{IV} is strongly blueshifted [\ion{O}{III}] is very weak. 
This is very similar to what we found when we used Gaussian functions to model the emission. 
The correlation between \ion{C}{IV} blueshift and [\ion{O}{III}] EQW is shown in a different way in Figure~\ref{fig:mfica_composites}. 
Here we divide our sample into four bins according to the \ion{C}{IV} blueshift. 
From the quasars in each \ion{C}{IV} blueshift bin we then find then generate an ICA spectrum using the median weights from each quasar. 
The differences in the spectra as a function of the \ion{C}{IV} blueshift are dramatic. 
[\ion{O}{III}] becomes progressively weaker and more blueshifted.
The anti-correlation with \ion{Fe}{III} and the blue-ward \ion{Fe}{II} also clear, but there is no change in the redward \ion{Fe}{II}. 
ICA looks promising, and in the future, the weights could be added to EV$1$, or even replace the traditional components. 

We can take the median weights as a function of blueshift and use these to generation of high signal-to-noise ratio (S/N) examples of such spectra. 

\section{Summary}

\begin{itemize}

\item We have found that outflows in the NLR, as indicated by broad velocity-widths and asymmetries in [\ion{O}{III}], are ubiquitous in the luminous quasar population. 

\item [\ion{O}{III}] is broad, blueshifted. Show that [\ion{O}{III}] velocity-width is correlated with the quasar optical luminosity. We interpret the broad, blueshifted wing as being to outflowing ionised gas, suggesting that kilo-parsec-scale outflows in ionized gas are common in this sample of high-luminosity, high-redshift quasars.

\item [\ion{O}{III}] EQW is weakly anti-correlated with the luminosity, but strongly correlation with the \ion{C}{IV} blueshift. 

\item Weak correlation between \ion{C}{IV} blueshift and [\ion{O}{III}] blueshift, but this could arise because the [\ion{O}{III}] core is decreasing faster than the wing component \citep[e.g.][]{shen14}. 

\item Accurate systemic redshift estimates are essential in a number of applications, and researchers have devoted a large amount of telescope time to obtaining near-infrared spectra to access [\ion{O}{III}] for this purpose. While we find a reasonable agreement between redshift estimates based on [\ion{O}{III}] and the broad Balmer lines, we also note that in XX per cent of our sample [\ion{O}{III}] is very weak or strongly blueshifted, and can not be used to measure the systemic redshift. 

\item At fixed luminosity, I find that as the blueshift of the \ion{C}{IV} emission increases, the [\ion{O}{III}] emission becomes weaker and more blueshifted, and disappears entirely in quasars with the most extreme \ion{C}{IV} blueshifts.  

\end{itemize}





