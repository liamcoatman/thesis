% !TEX root = ../main.tex

%************************************************
\chapter{Summary and future prospects}
\label{ch:summary} 

%************************************************

Supermassive BHs and their host-galaxies are thought to evolve in tandem, with the energy output from the rapidly-accreting BH regulating star formation and the growth of the BH itself. 
The goal of better understanding this process has led to much work focussing on the properties of quasars and AGN at relatively high redshifts, $z\gtrsim 2$, when cosmic star formation and BH accretion both peaked. 
At these redshifts, however, ground-based statistical studies of the quasar population generally have no access to the rest-frame optical spectral region, which is needed to measure \hbns-based BH masses and NLR outflow properties. 
The cornerstone of this thesis has been a new near-infrared spectroscopic catalogue providing rest-frame optical data on $434$ luminous quasars at redshifts $1.5 \lesssim z \lesssim 4$.
This chapter provides a summary of the key results, together with a discussion of future directions for this research. 

\section{Correcting \ion{C}{IV}-Based Virial Black Hole Masses}

At high redshift, $z \gtrsim 2$, quasar BH masses are derived using the velocity-width of the \ion{C}{IV} broad emission-line, based on the assumption that the observed velocity-widths arise from virial-induced motions.  
However, \ion{C}{IV} exhibits significant asymmetric structure which suggests that the associated gas is not tracing virial motions. 
We find the \ion{C}{IV} emission-based BH masses to be systematically in error by a factor of more than five at $3000$\,\kms\, in \ion{C}{IV} emission blueshift and the overestimate reaches a factor of $10$ for quasars exhibiting the most extreme blueshifts, $\gtrsim5000$\,\kms. 
Using the monotonically increasing relationship between the \ion{C}{IV} blueshift and the mass ratio BH(\ion{C}{IV})/BH(\hans) we derive an empirical correction to all \ion{C}{IV} BH-masses.
The correction depends only on the \ion{C}{IV} line properties - i.e. the FWHM and blueshift - and allows single-epoch virial BH mass estimates to be made from optical spectra, such as those provided by the SDSS, out to redshifts exceeding $z\sim 5$. 

\section{Quasar-driven galaxy-wide outflows in ionised gas} 

Quasars driving powerful outflows over galactic scales is a central tenet of galaxy evolution models involving `quasar feedback' and significant resources have been devoted to searching for observational evidence of this phenomenon.  
We have used [\ion{O}{III}] emission to probe ionised gas extended over kilo-parsec scales in luminous $z\gtrsim2$ quasars.
Broad [\ion{O}{III}] velocity-widths and large blueshifts indicate that strong outflows are prevalent in this population.  
We estimate the kinetic power of the outflows to be up to a few percent of the quasar bolometric luminosity, which is similar to the efficiencies required in recent quasar-feedback models. 
[\ion{O}{III}] emission is very weak in quasars with large \ion{C}{IV} blueshifts, suggesting that quasar-driven winds are capable of sweeping away gas extended over kilo-parsec scales in the host galaxies. 

\section{Outflows and hot dust emission}

Using data from a number of recent wide-field photometric surveys, we have built a parametric SED model that is able to reproduce the median optical to infrared colours of tens of thousands of SDSS AGN at redshifts $1 < z < 3$. 
In individual objects, we find significant variation in the near-infrared SED dominated by emission from hot dust. 
We find that the hot dust abundance is strongly correlated with the strength of outflows in the quasar BLR (indicated by the \ion{C}{IV} blueshift), suggesting that the hot dust may be in a wind emerging from the outer edges of the accretion disc. 

\section{Future prospects}

\subsection{Further improvements to BH mass estimates}

With it's twelfth data release in $2016$, the number of AGN and quasars in the SDSS spectroscopic catalogue alone reached almost $400\,000$. 
SDSS-IV eBOSS will take this number to $800\,000$ \citep{myers15} and it will grow to several million in the near future with planned-surveys including $4$MOST and DESI \citep{dejong12,levi13}.
These enormous surveys will open up new possibilities in large-scale studies of AGN and quasar demographics using \ion{C}{IV}-based single-epoch virial BH mass estimates.

Allen \& Hewett (2017, in preparation) will soon publish improved redshifts for all quasars in the SDSS DR$7$ and DR$12$ catalogues. 
At the same time, we will publish catalogues of unbiased BH masses for both SDSS DR$7$ and DR$12$ based on the Allen \& Hewett redshifts. 
The components from the MFICA used in the Allen \& Hewett redshift algorithm will also be published.
With these components, if a rest-frame ultraviolet spectrum is available, it will be straightforward to determine the systemic redshift, via a simple optimisation procedure, and hence calculate the \ion{C}{IV} blueshift and BH mass. 

While the large \ion{C}{IV} blueshift-dependent systematic error in the \ion{C}{IV}-based BH masses is removed using the method developed in Chapter~\ref{ch:bhmass}, the remaining scatter between the \ion{C}{IV}- and Balmer-based BH masses is significant, particularly for objects with small \ion{C}{IV} blueshifts.
It may be possible to reduce this scatter further by adding more features from the rest-frame ultra-violet spectrum to the model. 

An efficient way to do this would be to adopt a data-driven approach. 
The MFICA components used in the Allen \& Hewett redshift algorithm provide a compact representation of the rest-frame ultra-violet spectral properties of SDSS quasars.  
Taking the set of $230$ objects with near-infrared spectra (and hence reliable Balmer-based BH masses), a model could be built that learns how these MFICA component weights depend on the BH mass. 
After the training step, the model could be used to predict a BH mass based only on the MFICA component weights for the ultra-violet spectrum.
There are numerous algorithms available to tackle this class of supervised learning problem (e.g. random forests). 

Taking this one step further, a data-driven model could be built to predict the unseen rest-frame optical spectrum from the ultra-violet region probed by SDSS spectra at redshifts $z\gtrsim2$.
This approach is inspired by the data-driven model for deriving stellar labels from spectroscopic data developed by \citet{ness15}.
The MFICA component weights can be thought of as a set of `labels' describing the optical spectrum. 
A flexible generative model could be built that predicts the flux at each wavelength in the SDSS rest-frame ultra-violet spectrum as a function of these weights.
The coefficients in this model, which could be linear or a low-order polynomial, could be trained on the sub-sample of quasars with spectra covering the full optical to ultra-violet region. 
Once the coefficients in the model have been determined, the model can statistically infer the optical MFICA component weights based only on the ultra-violet spectrum. 
The optical spectrum can then straightforwardly be reconstructed from the MFICA components and parameters of interest, including the BH mass, can be calculated, without the need for follow-up near-infrared spectroscopy. 

It is important to recall that the sample of AGN with reverberation mapping measurements is strongly biased to low luminosity Seyfert $1$ galaxies, and the maximum redshift is just $z\sim0.3$.
The reliability of single-epoch virial BH mass estimates for quasars at high-redshifts, even using low-ionisation lines like \hbns, is dependent on the (as of yet untested) extrapolation of reverberation-mapping results to quasars at much higher luminosities. 
In the future, this situation will be improved with the results of large on-going statistical reverberation mapping projects for luminous quasars at high-redshifts \citep[e.g.][]{shen15,kingoz15}. 

\subsection{Studying feedback with IFU spectroscopy}

In Chapter~\ref{ch:nlr}, we made some order of magnitude calculations of the energetics of the outflows traced by the [\ion{O}{III}] emission based on a number of assumptions about the properties of this gas. 
However, to study the morphology and energetics of the outflows in detail, we must turn to spatially-resolved integral field unit (IFU) spectroscopy.

Extended outflows have been detected at high redshifts ($z\gtrsim2$) in a handful of objects using SINFONI on the VLT \citep[e.g.][]{carniani15}. 
With the analysis presented in Chapter~\ref{ch:nlr} we now have a much improved understanding of the distribution of [\ion{O}{III}] emission properties in luminous quasars at high redshifts. 
Quasars with a range of [\ion{O}{III}] emission properties could be selected from this sample as targets for IFU spectroscopy. 
The MFICA component weights, which are effective even at low S/N, could be used to isolate the systemic and outflowing components in the [\ion{O}{III}] emission in the spatially resolved spectra.
Measuring the extent of the outflowing gas traced by [\ion{O}{III}] could reveal the nature of the correlation between the [\ion{O}{III}] and \ion{C}{IV} blueshifts.

Large optical spectroscopic surveys (e.g. SDSS) have provided constraints on the prevalence and drivers of ionised outflows in tens of thousands of quasars.
A new generation of surveys using multi-object IFU spectrographs \citep[including KMOS, MaNGA and SAMI;][]{croom12,sharples13,bundy15} are now providing spatially-resolved information on the physical properties of galaxies (such as gas kinematics and star formation histories) for statistically powerful samples at a range of redshifts. 
Ionised gas accounts for only a small fraction of the total gas in galaxies.
Sub-millimetre interferometers (including ALMA) provide complementary information by probing the cold, molecular gas from which stars are formed. 
Massive molecular outflows in quasar host galaxies are now being detected with new facilities up to very high redshifts \citep[e.g.][]{maiolino12} and will lead to great advances in our understanding of the AGN/host galaxy connection. 


 

 







