% !TEX root = ../main.tex

%************************************************
\chapter{Conclusions / Future Work}
\label{ch:conclusions} 

%************************************************

We have explored relationships between BLR outflows, NLR outflows and hot dust emission. 

Put some stuff from research proposals here 

\section{Future: Red quasars}

Punctuated fuelling episodes, e.g. driven by galaxy mergers, satellite accretion and even secular processes,
almost certainly lead to AGN experiencing activity-, outflow- and obscuration-dominated cycles with some overlap between phases. 
However, quantitatively, it remains unclear how these phases relate to the fundamental properties of the accreting black-hole (e.g.  mass (M$_{\mathrm{BH}}$), bolometric luminosity (L$_{\mathrm{bol}}$) and Eddington ratio (L/L$_{\mathrm{Edd}}$) and the elements of the non-spherical geometry).


% \section{Future work}
% \todo{To do}
% \begin{itemize}
% \item Publish definitive masses using Allen \& Hewett redshifts 
% \item See Park criticism (low blueshift end)
% \item Data-driven mapping - see research proposal and Joe's email.
% \end{itemize}

% Allen \& Hewett will publish improved redshifts for all quasars in the SDSS DR$7$ and DR$12$ catalogues. 
% At the same time we will publish catalogues of unbiased BH masses for both SDSS DR$7$ and DR$12$ based on the Allen \& Hewett redshifts. 
% The components from the mean-field independent component analysis \citep[see][for an application to astronomical spectra]{allen13} used in the Allen \& Hewett redshift algorithm will also be published.
% With these components, if a rest-frame ultraviolet spectrum is available, it will be straightforward to determine the systemic redshift, via a simple optimisation procedure, and hence calculate the \ion{C}{IV} blueshift. 

% Large scatter remains, particularly at low \ion{C}{IV} blueshifts. 
% Some of this is correlated with the \ha FWHM, which is in turn correlated with EV1. 
% However, no good - need something in the UV spectrum. 
% Remarkable we can do so well with a single parameter, but try Blueshift+EQW (e.g. Runnoe)
% But better to use data-driven approach (see research proposal stuff). 

% Mention John's clustering work. 