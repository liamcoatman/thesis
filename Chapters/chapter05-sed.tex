% !TEX root = ../main.tex

%************************************************
\chapter{SED Properties}
\label{ch:sed} 

% Refer to: 
% HotDustPaper/
 % summary-150309.ipynb
 % notes.md
 % correlations_summary_141204.ipynb
 % correlations_summary.md
 % various *.md 

%************************************************

\section{Introduction}

\begin{figure}
  \centering
  \includegraphics[width=\textwidth]{figures/chapter05/shangsed.pdf}
  \caption{Median radio-loud SED from \citet{shang11}.}
  \label{fig:seyfert_sed}
\end{figure}

AGN emit strongly over many decades in frequency (Figure~\ref{fig:seyfert_sed}). 
At different frequencies, the emission originates from processes occurring in different regions of the \ac{AGN}. 
Hard X-ray emission is dominated by Compton up-scattering of accretion disk photons by electrons in a hot corona \citep[e.g.][]{sunyaev80}, \ac{UV}/optical by thermal accretion disc emission, \ac{IR} by dust at a wide range of temperatures, and radio by synchrotron emission in relativistic jets.   

Significant diversity is observed in the \ac{SED}s of individual objects. 
However, the systematic study of the dependence of the \ac{SED} shape on physical parameters has, until very recently, been limited by the difficulty in obtaining a large sample of quasars with good multi-wavelength coverage and large dynamic range in luminosity and redshift. 
However, we are able to take advantage of a number of recent, sensitive, wide-field photometric surveys, including SDSS (in the UV/optical), UKIDSS (in the near-infrared) and WISE (in the mid-infrared).
We will combine this information with the \ac{BH} mass and mass-normalised accretion rate estimates and outflow diagnostics which we developed in Chapters~\ref{ch:bhmass} and \ref{ch:nlr}. 
We will determine whether there are \ac{SED}-related systematics as a function of outflow signatures and \ac{BH} mass or Eddington ratio. 

Since the physical processes that power \ac{AGN} are generally understood only qualitatively, almost all \ac{AGN} \ac{SED} templates are empirical. 
The empirical template of \citet{elvis94} is still the most commonly cited, despite many additions and updates \citep[e.g.][]{polletta00, kuraszkiewicz03, risaliti04, richards06,  polletta07, lusso10, shang11, marchese12, trichas12}. 
However, these composite spectra are often constructed from quasars with a huge range in luminosity as a function of wavelength. 
In addition, the presence of significant host galaxy at optical wavelengths in low-redshift objects is an additional complication which has not always been taken care of adequately. 
There is therefore a strong rationale for taking a parametric approach to modelling quasar \ac{SED}s. 
This is the approach we take in this chapter. 
We then investigate whether the systematic dependence of the model parameters on quasar properties including the BH mass, luminosity, accretion rate and outflow diagnostics. 

\section{Data}

\begin{table}
  \small
  \centering
  \begin{tabular}{l c c}
    \hline 
    Survey & Band & $\lambda_{\rm eff}$ [$\mu{\rm m}$] \\
    \hline 
    SDSS & $u$ & 0.3543 \\
         & $g$ & 0.4770 \\
         & $r$ & 0.6231 \\
         & $i$ & 0.7625 \\
         & $z$ & 0.9134 \\
    UKIDSS & $Y$ & 1.0305 \\
           & $J$ & 1.2483 \\
           & $H$ & 1.6313 \\
           & $K$ & 2.2010 \\
    WISE & $W1$ & 3.4 \\
         & $W2$ & 4.6 \\
         & $W3$ & 12.0 \\
         & $W4$ & 22.0 \\           
    \hline
  \end{tabular}
  \caption{Available photometry}
  \label{tab:photometry}
\end{table}

\subsection{SDSS DR7}

We use the Seventh Data Release (DR7) of the \ac{SDSS} spectroscopic quasar catalogue \citep{schneider10}, which includes 105,783 objects across 9380 deg$^2$. 
The \ac{SDSS} obtained images in five broad optical pass-bands: $u$, $g$, $r$, $i$ and $z$ (Table~\ref{tab:photometry}).  
We use BEST point-spread function (PSF) magnitudes, correcting for Galactic extinction using the maps of \citet{schlegel98}, assuming a Milky Way (MW) extinction curve \citep{pei92} and an extinction to reddening ratio ${\rm A}(V) / {\rm E}(B-V) = 3.1$. 
Although the SDSS asinh magnitude system is intended to be on the AB system \citep{oke83}, the photometric zero-points are known to be slightly off the AB standard. 
To account for this we add 0.03 mag to the $u$, $g$, $r$ and $i$ magnitudes, and 0.05 mag to the $z$ magnitude.  
\todo{Where did these numbers come from?}

\subsection{UKIDSS Large Area Survey}

We use the ninth data release (DR9) of the UKIRT Infrared Deep Sky Survey \citep[UKIDSS;][]{lawrence07} Large Area Survey (ULAS) which has observed $\sim 3,200$ deg$^2$ in four near-IR pass-bands: $Y$, $J$, $H$ and $K$. 
The ULAS magnitudes are aperture corrected magnitudes in a 2$''$ diameter aperture and are also corrected for Galactic extinction using the \citet{schlegel98} map. 

\subsection{WISE All-WISE Survey}

The Wide-field Infrared Explorer \citep[WISE;][]{wright10} mapped almost the entire sky in four mid-IR band-passes: $W1$, $W2$, $W3$ and $W4$. 
The WISE AllWISE Data Release (`AllWISE') combines data from the nine month cryogenic phase of the mission that led to the `AllSky' data release with data from the NEOWISE program \citep{mainzer11}. 
WISE magnitudes are given in the Vega system, and Vega to AB conversion factors are given in the WISE Explanatory Supplement \citep{cutri13}. 

\subsection{Quasar sample}

We include only the 63,855 quasars with $i$ band magnitudes brighter than 19.1, i.e. the quasars selected by the main SDSS quasar selection algorithm for quasars with colours consistent with being at redshifts $z < 3$ \citep{richards02}. 
Cross-matching (with a 2$''$ radius and picking only the nearest neighbour) the SDSS DR7Q catalogue with the ULAS catalogue, which covers only $\sim 38$\% of the SDSS foot-print, resulted in 20,954 matches. 
% Manda did cross-match 
Cross-matching to WISE resulted in 20,665 objects detected in one or more of the WISE band-passes. 
% Manda did cross-match 
We exclude quasars flagged as broad absorption line (BAL) quasars by \citet{allen11} from the sample (leaving 19,853 quasars).
The redshift and luminosity distribution of this sample is shown in Figure~\ref{fig:lum_z}. 

For a given $i$ magnitude, a quasar with a blue spectrum is more likely to be undetected at longer wavelengths than a quasar with a red spectrum. 
Therefore, as we allow fainter quasars in to our sample we will be biased towards objects with redder spectra.
We verified that above the $i=19.1$ limit the sample is 95\% complete in all band-passes with S/N $>$ 5 (excluding WISE $W3$ and $W4$) and that this fraction is not changing rapidly with the brightness of the sample. 

\begin{figure}
  \centering
  \includegraphics[width=\textwidth]{figures/chapter05/lum_z.pdf}
  \caption{Distribution of our sample in the redshift-luminosity plane.}
  \label{fig:lum_z}
\end{figure}

% dr7dat.py 
% readdat.py
% DefingSample.ipynb 

\subsection{Generating the quasar catalogue}

We match the Seventh Data Release (DR7) of the \ac{SDSS} spectroscopic quasar catalogue \citep{schneider10} to the tenth data release of the UKIRT Infrared Deep Sky Survey \citep[UKIDSS;][]{lawrence07} Large Area Survey (matching radius 2$''$) and to the Wide-field Infrared Explorer \citep[WISE;][]{wright10} `AllWISE' data release (matching radius 3$''$). 
The cross-matched catalogue contains 36,607 objects. 

UKIDSS photometry corresponds to the `apermag3' values while WISE photometry is from profile fitting (w1mpro, w2mpro etc.) - see AllWISE Explanatory supplement. 
SDSS is the BEST point-spread function (PSF) magnitudes.

We need to correct for galactic extinction (using quasar template) and apply AB-to-Vega zero points. 

Vega (V=+0.026) to AB offsets compared to Manda's [= Hewett et al. 2006]

u  0.913  0.927
g -0.081 -0.103
r  0.169  0.146
i  0.383  0.366
z  0.542  0.533
Y  0.641  0.634
J  0.941  0.938
H  1.378  1.379
K  1.897  1.900
W1 2.691  2.699
W2 3.331  3.339 





\section{Quasar SED}

We have 19,853 quasars with photometric data from SDSS, UKIDSS and WISE. 
Our quasars cover the redshift range $0.2 < z < 4$, and so this data covers the rest-frame wavelength range from 800\AA\, to 3.8$\mu$m. 
In this region the \ac{SED} is dominated by the accretion disc, emission lines and thermal emission from the hottest ($T\sim1200$K) dust. 
Host galaxy emission is also significant for quasars at redshifts $z\lesssim1$, and the effect of dust extinction at the \ac{AGN} redshift is another factor which must be considered.   
In this section, we describe how we have modelled emission from these different physical processes. 
The model spectrum is shown in Figure \ref{fig:modelsed}, with each of the main components indicated. 

\section{SED Model}

\begin{figure}
  \centering
  \includegraphics[width=\textwidth]{figures/chapter05/sed_model.pdf}
  \caption{Model spectrum at $z=1$, showing the contributions to the total flux from the blue power-law slope, red power-law slope, Balmer continuum, blackbody, emission line spectrum and host galaxy}
  \label{fig:modelsed}
\end{figure}


\subsection{Accretion Disc}

Thermal accretion disc emission in the 0.1 - 1 $\mu$m region is characterised by a broken power-law with three free parameters: a break-wavelength $\lambda_{\rm break}$, a blue power-law index $\alpha_{\rm blue}$ for wavelengths shorter than the break wavelength, and a red power-law index $\alpha_{\rm red}$ for wavelengths longer than the break wavelength.

\subsection{Balmer Continuum}

High order Balmer lines, optically thin Balmer continuum emission, two-photon emission and \ion{Fe}{II} emission blend together to form a distinct feature in quasar spectra at $\sim3000$\AA. 
We simulate the Balmer continuum we use the empirical model given by \citet{grandi82}: 

\begin{equation}
  F(\lambda) = C_{\rm BC} \times B_\lambda(T_e)(1-e^{-\tau_\lambda}); \quad \lambda \leq \lambda_{\rm BE}
\end{equation}

where $C_{\rm BC}$ is a normalisation factor, $B_\lambda(T_e)$ is the Planck function, $T_e=13150$K is the effective temperature, $\lambda_{\rm BE}=3460$\AA\, is the wavelength at the Balmer edge, and $\tau_\lambda = \tau_{BE}\left( \nicefrac{\lambda_{BE}} {\lambda} \right)^{-3}$ is the optical depth with $\tau_{\rm BE}=45$ the optical depth at $\lambda_{\rm BE}$. 
This function is convolved with a Gaussian with $\sigma=5000$\kms to simulate the effect of bulk velocity shifts comparable to those present in broad \ac{AGN} emission lines. 

\subsection{Hot Dust}

Thermal emission from hot dust, which dominates the \ac{SED} at wavelengths longer than $1\mu$m, is modeled using a simple blackbody

\begin{eqnarray}  
  F_\lambda = C_{\rm BB} \times \frac{2 hc^2}{\lambda^5}\frac{1}{ e^{\frac{hc}{\lambda k_\mathrm{B}T_{\rm BB}}} - 1}, 
\end{eqnarray}

with two free parameters: the temperature $T_{\rm BB}$ and normalisation. 

\subsection{Emission Lines}

We use an emission line template taken from \citet{francis91}, which has been extended by \citet{maddox06} to include the \hans and Pa$\alpha$ emission lines. 
All emission lines, with the exception of \hans, are scaled using a single free parameter $C_{\rm EL}$, which preserves relative \ac{EQW}s:

\begin{eqnarray}
  F_{\lambda} =  C_{\rm EL} \times \frac{F_{\lambda, \rm el}}{F_{\lambda, \rm cont}} \times F_{\lambda} ; \quad \lambda < 4700{\rm \AA} \;{\rm and}\; \lambda > 7000{\rm \AA} 
\end{eqnarray} 

where $F_{\lambda, \rm el}$ is the emission line template, $F_{\lambda,\rm cont}$ is the continuum flux in the template, and $F_{\lambda}$ is the continuum flux in the \ac{SED} model.  
\hans, one of the strongest broad emission lines, is scaled separately: 

\begin{equation}
  F_{\lambda} =  C_{\rm EL} \times C_{{\rm H} \alpha} \times \left( \frac{L(z)} {L(z_{\rm nrm})} \right)^{-\beta} \times \frac{F_{\lambda, \rm el}}{F_{\lambda, \rm cont}} \times F_{\lambda}; \quad 4700{\rm \AA} < \lambda < 7000{\rm \AA} 
\end{equation}

The luminosity dependence of the \ha EQW (i.e. the Baldwin effect) is parametrised with a simple power-law with slope $\beta=0.04$.
The redshift dependence of the mean \ac{AGN} luminosity $L(z)$ for the SDSS quasar catalogue has been determined empirically.

\subsection{Host Galaxy}

Emission from the host galaxy is important for \ac{AGN} at redshifts $z\lesssim1$, particularly in the region around the $1\mu$m inflection point in the quasar \ac{SED}. 
We use a $z=0$ Sb template from \citet{mannucci01}, which does not evolve with redshift.
The template is scaled by a multiplicative factor $C_{\rm Gal}$ and added to the \ac{AGN} \ac{SED}. 
We define a new parameter, $\eta$, the fractional contribution from the host galaxy to the total flux in the interval 4000 and 5000\AA:

\begin{eqnarray}
  \eta \equiv \frac{C_{\rm Gal}F_{\rm Gal}}{F_{\rm AGN} + C_{\rm Gal}F_{\rm Gal}},
\end{eqnarray}

where $F_{\rm Gal}$ and $F_{\rm AGN}$ are the flux of the galaxy and \ac{AGN} respectively. 
Rearranging for the scaling factor $C_{\rm Gal}$ gives:

\begin{eqnarray}
  C_{\rm Gal} = \frac{\eta}{1 - \eta} \frac{F_{\rm AGN}}{F_{\rm Gal}}.
\end{eqnarray}

The fractional contribution to the total emission from the host galaxy changes as a function of the \ac{AGN} luminosity and, in a flux-limited sample, the mean \ac{AGN} luminosity increases as the redshift increases. 
We parametrize the \ac{AGN} luminosity dependence of the host galaxy luminosity as a power-law:

\begin{eqnarray}
  \label{eq:lgal}
  \frac{L_{\rm Gal}}{L_{\rm AGN}} &=& L_{\rm AGN}^{\beta - 1} 
\end{eqnarray}

with slope $\beta=0.42$ \citep{maddox06}. 
The galaxy scaling factor $C_{\rm Gal}$ becomes 

\begin{eqnarray}
  C_{\rm Gal} &=& \frac{\eta}{1 - \eta} \frac{F_{\rm AGN}}{F_{\rm Gal}} \left[ \frac{ L_{\rm Gal}(z)} {L_{\rm AGN}(z)} \right] \left[ \frac{ L_{\rm Gal}(z_{\rm nrm})} {L_{\rm AGN}(z_{\rm nrm})} \right]^{-1} \\
  &=& \frac{\eta}{1 - \eta} \frac{F_{\rm AGN}}{F_{\rm Gal}} \left[ \frac{L_{\rm AGN(z)}} {L_{\rm AGN(z_{\rm nrm}})} \right]^{\beta -1}, 
\end{eqnarray}

where $z_{\rm nrm}$ is an arbitrary redshift at which the fractional contribution from the host galaxy is by definition $\eta$. 

\subsection{Dust Extinction}
\label{sec:sed-extinction} 

The selection criteria of the SDSS DR7Q catalogue are sensitive to quasars with moderate amounts of dust reddening \citep[possibly as high as $E(B-V)$ $\sim$ 0.5;][]{richards03} at the redshift of the quasar, and so we included the effect of dust extinction in our model. 
We use an extinction curve appropriate for the quasar population which has been derived by Paul Hewett. 
To derive the quasar extinction curve, UKIDSS photometry was used to provide an $E(B-V)$\footnote{$E(B-V)=A(B)-A(V)$} estimate, via the magnitude displacement of each quasar from the locus of un-reddened objects. 
At redshifts $2 < z < 3$ the reddening measure is made at rest-frame wavelengths 3500-7000\AA, where Galaxy, LMC and SMC extinction curves are very similar. 
The SDSS spectra of the same objects are then employed to generate an empirical extinction curve in the ultraviolet, down to 1200\AA. 
The resulting curve has no 2200\AA~ feature and rises rapidly with decreasing wavelength but is not as steep as the SMC curve. 
The extinctions curves give the colour excess $E(B-\lambda) = A$ relative to the colour excess $E(B-V)$ as a function of wavelength $\lambda$. 
The ratio of total to selective extinction, $R$, is defined as: 

colour excess $E(B-V)$ is related to the extinction in the $V$ band, $A(V)$, via the ratio  $R$, 

\begin{eqnarray}
  R_V = \frac{A(V)}{E(B-V)}
\end{eqnarray}

where we assume $R_V = 3$. 
Hence the extinction at a wavelength lambda $A(\lambda)$ is 

\begin{eqnarray}
  A(\lambda) = E(B-V) \times \left[ \frac{E(\lambda-V)}{E(B-V)} + R \right] 
\end{eqnarray}

where the colour excess $E(B-V)$ is a free parameter in our model. 
The attenuation of the flux at a given wavelength is then:

\begin{eqnarray}
  F_\lambda = F_\lambda10^{-A(\lambda)/2.5}
\end{eqnarray}

in the rest frame of the quasar. 

\subsection{Empirical Correction}

\todoinline{Describe Paul's empirical correction.}

\section{The `Standard' SED Model} 

\todoinline{
\begin{itemize}
    \item Given the same parameters, my model and Paul's look identical 
    \item I'm generating model colours using my model and Paul's best-fit parameters, and Paul's correction
    \item Do my model colours look the same as Paul's? (i.e. is there a bug in my code?)
    \item Can I generate the same median colours as Paul? (i.e. what sample is being used? what magnitudes?)
    \item Can I do my own fit to the data?  
\end{itemize}  
}

\begin{figure}
  \centering
  \includegraphics[width=\textwidth]{figures/chapter05/throughput.pdf}
  \caption{Model spectrum at three different redshifts (each arbitrarily scaled), and throughput functions for SDSS, UKIDSS and WISE band-passes.}
  \label{fig:filters}
\end{figure}

We will begin by deriving a `standard' \ac{SED} model by constraining a single set of parameters with our sample of 19,853 quasars, encompassing a range of redshifts, luminosities, accretion rates etc. 
The free parameters in our model are the blue power-law slope, the red power-law slope, the power-law break wavelength, the blackbody temperature, the blackbody normalisation, the emission line \ac{EQW} scaling, the \ha scaling and the fractional contribution from the host galaxy to the total flux. 
The reddening $E(B-V)$ is fixed to zero, since a large fraction of \ac{SDSS} quasars have very small amounts of dust reddening \citep{richards03}. 
We generate a set of model observed spectra at redshifts from $z=0.25$ to $z=3.75$ in intervals of $\Delta z = 0.1$. 
The \ac{SED} model is shown at three different redshifts in Figure~\ref{fig:filters}. 
The predicted broadband magnitude of the model is given by integrating the spectrum over the throughput for each of the bands.  

\begin{eqnarray}
  m_\lambda(P) & = & -2.5{\rm log}(f_\lambda(P)) - m_0(P), 
\end{eqnarray}

where $m_0(P)$ is the zero-point magnitude of band $P$ and the mean flux density $f_{\lambda}(P)$ is given by 

\begin{eqnarray}
  \label{eq:flux}
  f_{\lambda}(P) & = & \frac{\int P(\lambda) f_\lambda(\lambda) \lambda d\lambda }{\int P(\lambda) \lambda d\lambda}
\end{eqnarray}

where $P(\lambda)$ is the dimensionless throughput function of the band-pass. 
Magnitudes are calculated in the AB system (Oke \& Gunn 1983), in which case the zero-point flux per unit wavelength is 

\begin{eqnarray}
  \frac{f_\lambda(\lambda)}{{\rm erg}~{\rm cm}^{-2}~{\rm s}^{-1} {\rm\AA}^{-1}} = 0.1087 \left(\frac{\lambda}{\rm \AA}\right)^{-2}.
\end{eqnarray}


We divide our quasar sample in to the same redshift bins.
In each bin we normalise the quasar \ac{SED}s in the SDSS $i$ band, and then calculate the median \ac{SED}. 
The model \ac{SED} in each redshift bin is similarly normalised. 
The chi-squared statistic is then minimised using the `nelder-mead' algorithm. 

Our \ac{SED} model is valid only up to $\lambda \sim 3\mu$m in the quasar rest frame (the approximate wavelength of the peak in hot dust emission); beyond this additional contributions to the total flux from cooler dust will become significant. 
This prevents us from using the two highest wavelength WISE bands in the fit. 
We also exclude the SDSS $u$ and $g$ band-passes from the fit at $z > 2.7$ and $z > 3.7$ respectively, where these bands start to be affected by Ly$\alpha$ forest absorption.

\section{Results}

The best-fitting parameters from the fit are shown in Table \ref{tab:params}. 
\todo{Re-do fit}
The colours ($u - g$, $g - r$, etc.) of the median SED, the individual quasars, and the best-fitting model are plotted as a function of redshift in Figs.~\ref{fig:color_1} and \ref{fig:color_2}.
\todo{Need to show individual quasars.}  
Most of the large variations that can be seen in the median colours of the quasars as a function of redshift are due to strong emission lines being redshifted in to and out of the band-passes.
Take away message is that a single, fairly simple parametric is able to reproduce the median colours of tens of thousands of \ac{AGN} with a large dynamic range in redshift and luminosity. 
However, there is a significant scatter about the median model, which we will investigate in the next section?
Does any part of this scatter have a systematic dependence on properties of the BH (mass, accretion rate) or outflow diagnostics.  

\begin{table}
  \centering
  \begin{tabular}{c c c}
    \hline 
    Parameter & Symbol & Value \\
    \hline 
    Blue power-law index & $\alpha_{\rm blue}$ & 0.58 \\
    Red power-law index & $\alpha_{\rm red}$ & -0.04 \\
    Power-law break & $\lambda_{\rm break}$ & 2945 \\
    Blackbody temperature & $T_{\rm BB}$ & 1216 K \\
    Blackbody normalisation & $C_{\rm BB}$ & 0.22 \\
    Emission line scaling & $C_{\rm EL}$  & 0.63 \\
    \ha emission line scaling & $C_{{\rm H}\alpha}$  & 0.63 \\
    Galaxy fraction & $\eta$ & 0.29 \\
    \hline
    E(B-V) & E(B-V) & 0.00 \\
    \hline
  \end{tabular}
  \caption{Model parameters.}
  \label{tab:params}
\end{table}

\begin{figure}
\includegraphics[width=\textwidth]{figures/chapter05/sed_color_plot_1.pdf}
\caption{Colours of median SED and best-fitting model, with and without correction. \todoinline{Corrected is in orange, uncorrected in green. Check with Paul. Correction often makes colours a lot worse. Once got to the bottom of this just show with correction.}}
  \label{fig:color_1}
\end{figure} 


\begin{figure}
\includegraphics[width=\textwidth]{figures/chapter05/sed_color_plot_2.pdf}
\caption{Colours of median SED ({\it black circles}), individual objects ({\it grey points}), best-fitting  model ({\it black line}) as a function of redshift.}
  \label{fig:color_2}
\end{figure} 

\section{Discussion of Fit}

\begin{figure}
  \centering
  \includegraphics[width=\textwidth]{figures/chapter05/model_residuals.pdf}
  \caption{Residuals from fit as a function of rest-frame wavelength.}
  \label{fig:residuals}
\end{figure}

In Figure \ref{fig:residuals} we show the difference between the magnitudes from the best-fitting model and the median magnitudes from the sample. 
We have transformed the effective wavelengths of the band-passes to the rest frame of the quasars in each redshift bin, to give to the residuals as a function of rest-frame wavelength. 
We represent the residuals measured in each band-pass using a different coloured line. 
Differences between residuals from different band-passes at the same rest-frame wavelength could indicate redshift evolution of the typical quasar SED. 

The residuals indicate that over a large redshift range the model does a fairly good at reproducing the median observed colours of the sample. 
Most discrepancies are at the $<0.1$ mag level. 
A single model is effective at reproducing the median colours, suggesting that the properties of a typical quasar do not change significantly over a wide range of redshifts and luminosities. 
Many authors have found no significant dependence of the mean \ac{SED} on properties such as redshift, bolometric luminosity, \ac{BH} mass, or accretion rate \citep[e.g.][]{elvis12,hao13}. 
On the other hand, for the individual objects there is a significant scatter about the mean.
\todo{Need individual points on plot to show this.} 

\section{Hot Dust}

\begin{figure}
\centering
\includegraphics[width=\columnwidth]{figures/chapter05/w1w2_versus_redshift_ratio.pdf}
\caption{$W1 - W2$ colours of sample as a function of redshift. Above a certain density threshold points are represented by a density plot. On top we plot the colours of our standard SED model, with a fixed temperature and a varying NIR (1 - 3 $\mu$m) to UV ratio.}
  \label{fig:w1w2colorsratio}
\end{figure}

The spread in the KW1W2 colours (Figure~\ref{fig:w1w2colorsratio}), probing the rest-frame $\sim$1-2 micron region, is significant and strongly suggests the presence of real variation in the hot dust temperature and luminosity among the quasars. 

\subsection{Parametrising the hot dust emission}

We characterise the hot dust properties of our sample in terms of the temperature and luminosity of a blackbody.  
We choose to parametrise the luminosity in terms of the NIR to UV luminosity ratio (which is proportional to the covering factor of hot dust ($L_{NIR}/L_{Bol}$) used in other studies \citep{roseboom13}. 
The UV and NIR luminosity are calculated between 2000 and 9000\AA\, and 1 and 3 $\mu$m respectively.

Some previous studies \citep[e.g.][]{wang13,zhang14} have instead parametrised the near-IR emission using a power-law ($\propto \lambda^{\beta_{\rm NIR}}$), with $\beta \simeq 0.5$. 
We tested this parametrisation, and evaluated it's effectiveness relative to using a blackbody. 
We normalise the power-law at 9000\AA, where its flux is set equal to the flux of the UV/optical model. 
The NIR power-law slope is fit between $\sim$1 and 2.4$\mu$m (with the exact wavelength region being fit depending on the redshift of the quasar). 
We found large residuals in the best-fitting model which varied systematically as a function of $\lambda_{eff}/(1+z)$.  
This suggests that the power-law model is a poor fit to the shape of the near-IR emission. 
One needs to take care in looking at trends with luminosity given the observed-frame passband information on the rest-frame SED can produce some strong systematics with redshift, particularly if the SED-model is not a good fit to the actual SED. 
A similar conclusion was reached by \citet{gallagher07}.

\subsection{Sample}

Our goal is to determine the temperature and abundance of the hot dust component in individual quasars.  
These properties will be measured by fitting a model to the SDSS-UKIDSS-WISE photometry. 
Constraining a T$\sim$1200K blackbody component in the SED model requires photometric data covering $\sim$1-3$\mu$m in the rest-frame of the quasar. 

The observed-frame wavelength coverage of the available pass-bands limits the redshift range of the quasars which can be used. 
We consider only quasars at redshifts $z>1$ where the relative host galaxy contribution to the SED is negligible. 
At redshifts $1 \lesssim z \lesssim 1.5$ the available ugrizYJHKW1W2 photometry provides good coverage of the rest-frame SED up to $\sim$2$\mu$m.
At $z\sim1.5$ the W2 passband is shifted to $\sim$1.8$\mu$m; at higher redshifts W2 is probing much shorter wavelengths than the peak of a T$\sim$1200K blackbody. 
Because the shape of the blackbody is not well constrained by the available photometry, the uncertainty on the blackbody temperature measurement increases sharply for quasars at redshifts $z\gtrsim1.5$ 

For the quasars at $z \sim 1$, the WISE W3 band is probing rest-frame wavelengths of $\sim5-6\mu$m. 
This region of the \ac{SED} is dominated by emission from cooler, more distant dust, which is not accounted for in our model.
However, at redshifts $z \gtrsim 2$ the WISE W3 passband probes sufficiently short wavelengths to be useful in constraining the shape of the hot blackbody component. 
Therefore for quasars at redshifts $z > 2$ we again have sufficient constraints from the ugrizYJHKW1W2W3 photometry to determine the temperature and normalisation of the blackbody component. 
There are few objects in our sample with redshifts $z > 2.7$, and so we set this as an upper limit on the redshift of our sample. 
Because of these constraints, our sample is divided in to two parts: one at low redshifts ($1 < z < 1.5$) and the other at higher redshifts ($2 < z < 2.7$). 

We impose a lower-limit signal-to-noise ratio (S/N) $>$ 5 magnitudes in the $K$, $W1$ and $W2$ band-passes for the low-$z$ sample and S/N > 5 in the $W1$, $W2$, and $W3$ band-passes for the high-$z$ sample to ensure reliable photometry.
This gives us 5,910 quasars in our low-$z$ sample and 1,989 quasars in our high-$z$ sample. 

We will hold most model parameters fixed, and vary only the blackbody parameters which parametrise the NIR emission. 
Therefore we need to define a sub-sample of objects which we know are well fit by our standard SED model in the UV/optical region. 
This means excluding objects with extreme emission line \ac{EQW}s and/or significant dust extinction.
We use the $i-K$ colours of the quasars as a measure of the overall colour of the quasars as it provides the longest baseline in wavelength without being affected by absorption in the Ly$\alpha$ forest at high redshifts. 
A significant amount of the scatter in $i-K$ can be attributed to intrinsic variations in the UV power-law slopes of the individual quasars, which is why we allow a negative reddening. 
However, there is a clear `red tail' to the colour distribution which can be explained by dust reddening at the redshift of the quasar.
We discarded from our sample quasars with $i - K$ colours redder than our standard model with dust reddening E(B-V) = 0.075 and bluer than E(B-V) = -0.075 (Figure~\ref{fig:ikzplot}). 
Following this cut we are left with 4,615 quasars in our low-$z$ sample and 1,692 quasars in our high-$z$ sample. 

\begin{figure}
  \centering
  \includegraphics[width=\columnwidth]{figures/chapter05/ik_versus_z_low_ext.pdf}
  \caption{$i-K$ colours of non-BALQSO DR7Q quasars with $i>19.1$ as a function of redshift. The lines show the colours of our model with varying amounts of dust extinction. Quasars with extinction $|E(B-V)|>0.075$ are excluded.}
  \label{fig:ikzplot}
\end{figure}

\subsection{Diversity in hot dust properties}

In Figure~\ref{fig:w1w2colorsratio} we plot the $W1 - W2$ colours of the sample as a function of redshift at $z<3$. 
In this redshift range the $W1$ and $W2$ band-passes are probing the 1.2 - 2.8$\mu$m and 1.6 - 3.8 $\mu$m region of the rest frame SED respectively. 
For reference, the peak wavelength is at 2.4$\mu$m for a blackbody radiating at 1200K. 
At any given redshift we see a $\sim 0.5$ mag dispersion in the $W1-W2$ colours. 

On the same axes in Figure~\ref{fig:w1w2colorsratio} we have plotted the $W1 - W2$ colours derived from our SED model with a fixed blackbody temperature (1216K) and a ratio of NIR to UV luminosity ranging from 0.0 to 1.0, with the other model parameters held constant. 
We conclude that even with the sample restricted to be fairly uniform in its UV/optical properties, we still get an interesting spread in W1-W2 colours, which we can use to learn about the diversity of NIR properties in our sample. 
In the rest of this chapter we will characterise the hot dust properties of our sample, and test its relation to quasar properties such as luminosity, black-hole mass and normalised accretion rate, and outflow-properties. 

\section{Fitting procedure}

We will fit a model to the individual quasar SEDs, allowing the temperature and normalisation of the black body component to vary. 
The model spectrum is redshifted to the redshift of the quasar being fit and is then multiplied by the $ugrizYJHMW1W2W3$ throughput functions and normalised appropriately to give AB magnitudes. 
We minimise the chi-squared statistic using the minimisation is done using the 'nelder-mead' algorithm.
To avoid significant absorption in the Ly$\alpha$ forest at high-$z$, we restrict our fitting to wavelengths greater than 2000A; when the effective wavelength of a band-pass falls below this limit the band-pass is excluded from the fit. 
\todo{2000A is quite large given the Ly-alpha forest impacts from 1216A.}
$W3$ is only used for the quasars at redshifts $2 < z < 2.7$. 

\section{Results}

\todo{Show some example fits? Show overlayed data/model with alpha=0.1?}

In Figure~\ref{fig:ratio_tbb_density} we see that the two parameters are clearly correlated. 
For a lower temperature blackbody the NIR to UV luminosity ratio is larger. 
Such a correlation is to be expected: as the blackbody temperature is lowered, the peak shifts to longer-wavelengths (following Wien's displacement law). 
Because of this degeneracy we need to be very careful to separate out real trends of $R_{NIR/UV}$ with other quasar properties from indirect trends resulting from a mutual dependence on $T_{BB}$.  

\begin{figure}
  \centering
  \includegraphics[width=\textwidth]{figures/chapter05/ratio_tbb_density.pdf}
  \caption{Ratio of NIR to UV luminosity ($R_{NIR/UV}$) against temperature ($T_{BB}$) for low-$z$ sample. The density of points is shown in more dense regions of the space, and individual objects in less dense regions. }
  \label{fig:ratio_tbb_density}
\end{figure}

In Figure~\ref{fig:ratio_tbb_density} we show that there is quite a range of temperature and normalisation present in our sample. 
However, we need to check how much of this is due simply to uncertainties in the fits stemming from uncertainties in the photometry. 
In order to achieve this we took our standard SED model with a single temperature and normalisation blackbody component, and generated 200 mock SEDs with a brightness distribution similar to that of our real sample. 
We estimated the mean uncertainty of the magnitudes in the K, W1, and W2 band-passes as a function of apparent brightness. 
We then sampled the K, W1, and W2 magnitudes from Gaussian distributions, with a mean equal to the magnitude of the model SED, and the width equal to the mean uncertainty at the appropriate brightness. 
Finally, we fit these mock SEDs using our standard fitting procedure. 
The results are shown in the Figure below, on top of the results from our real sample (shown as grey contours). 
We can see that uncertainty in the photometry introduces a significant scatter to the temperature, but that this scatter is less than the intrinsic scatter in the data. 
This demonstrates that there is a real distribution of hot dust temperatures and luminosities in our sample. 

\begin{figure}
  \centering
  \includegraphics[width=\textwidth]{figures/chapter05/ratio_tbb_contours.pdf}
  \caption{Ratio of NIR to UV luminosity ($R_{NIR/UV}$) against temperature ($T_{BB}$). The grey contours show equally-spaced lines of constant probability density generated using a Gaussian kernel-density estimator on our data sample. The black points are for our mock data.}
  \label{fig:ratio_tbb_contours}
\end{figure}

\subsection{Correlations with quasar properties}

\begin{figure}
  \centering
  \includegraphics[width=\textwidth]{figures/chapter05/correlations_contour.pdf}
  \caption{Best-fit blackbody temperature against UV luminosity (left), black-hole mass (centre) and Eddington ratio (right) for $1 < z < 1.5$ sample (black) and $2 < z < 2.7$ sample (black). In region of high-density we represent the density with contours generated using a Gaussian kernel density estimation. \todoinline{Needs re-making with new BH masses.} \todoinline{Maybe just show as one sample?}}
  \label{fig:correlations_contour}
\end{figure}

We now look for correlations between the properties of the blackbodies we have fitted to the hot dust emission and other properties of the quasar such as redshift, \ac{BH} mass, normalised accretion rate (Eddington ratio), and outflow diagnostics.  
\todo{Calculate new BH masses and redo this section.}

% There is a clear anti-correlation between the UV luminosity and the best-fit blackbody temperature. 
% We calculated a Spearman rank-order correlation coefficient -0.33. 
% \todo{Do we believe this trend is real?}
% \todo{This is just for low-z sample.}
% The black-hole masses are virial estimates calculated by Shen et al. 2011 using the MgII emission line in the SDSS spectra. 
% The Eddington ratios (bolometric luminosity normalised by Eddington luminosity) are also calculated by Shen et al. 2011 using bolometric corrections in Richards et al. (2006a) using 3000\AA monochromatic luminosities. 
% There are no significant correlations between $T_BB$ and $R_{NIR/UV}$ and the UV luminosity, black-hole mass or Eddington ratio. 

% The dynamic range in luminosity is very limited. 
% I will combine the low and high $z$ samples. 
% As first step see if there is a difference in the median $R_{NIR/UV}$ for low/high luminosity samples. 

% At low-$z$  we get a much larger range in blackbody temperatures from our fits. 
% We discussed how the W3 S/N > 5 cut might be be biasing the high-z sample if the subset being removed had properties distinct from the remainder of the sample. 
% The W3 S/N > 5 cut removes about 25\% of the sample. 

% We observe a postive correlation between the black-hole mass and the NIR to UV luminosity ratio which is quite different from what we observed in our low-$z$ sample. 
% We believe that this is just a manifestation of the fact that at high redshift the black-hole masses are derived from CIV. 
% We will show below how the FWHM of CIV has a positive correlation with the hot dust abundance, and large CIV FWHM leads to larger black hole mass estimates. 
% This explains the apparent correlation between the IR/UV ratio and the black hole mass. 
% Eddington ratio measures the luminosity relative to the Eddington luminosity. 
% Higher blackhole mass estimates will lead to lower Eddington ratios, which is why the Eddington ratio appears to decrease with increasing IR/UV ratio. 
% For the sources in our low-$z$ sample the black-hole mass is measured using the broad MgII emission line. 
% As we will show below, the properties of the MgII emission line have no dependence on the hot dust properties. 


\subsubsection{Composite spectra}

\begin{figure}
  \centering
  \includegraphics[width=\textwidth]{figures/chapter05/z07_pls_comps.jpg}
  \caption{Composite SDSS spectra for objects at $z\sim0.7$. We have divided sample into objects with objects best-fit by small (red line) and large (red line) values of $\beta$. \todoinline{Remake if possible}.}
  \label{fig:pls_comp}
\end{figure}

Is there a connection between the hot dust properties and EV1? 
To test this we can divide the quasar sample by hot dust properties, and then generate composite spectra. 
The \ac{EV1} original EV1 correlates - \ion{Fe}{II}, \hb, [\ion{O}{III}] - are at around 4000-6000\AA. 
The \ac{SDSS} spectra are probing shorter wavelengths at redshifts $z\gtrsim1$.
Recall that our sample does not include any quasars at redshifts $z<1$, where the host galaxy emission starts to become significant. 
\todo{Need to decide what to do here. If I say host galaxy is significant using the power-law slope won't help. Use blackbody fits instead?}

The $z < 0.8$ SDSS spectrum composite comparison for the small and large $\beta_{NIR}$ sub-samples (Figure~\ref{fig:pls_somp}) is a very direct illustration of EV1. 
Hot dust emission increases with \ion{Fe}{II} EW. 
We also note that the amount of hot dust correlates with the \ion{Si}{III}/\ion{C}{III}] emission ratios. 
The \ion{Si}{III}/\ion{C}{III}] ratio is generally considered to be a good indicator of density and is one of the primary EV1 correlates. 
The relative flux ratio of \ion{Si}{III} to \ion{C}{III}] increases when \ion{C}{IV} is more blue-shifted \citep{richards11}. 
The \ion{Mg}{II} emission line has exactly the same profile/shape for the two samples (apparent changes in \ion{Mg}{II} seen in Fig.~\ref{fig:pls_comp} are the result of changes in \ion{Fe}{II} at wavelengths just short-ward of the line). 
Finally, we note that objects with more hot dust are slightly redder.

\citet{shen14} also find that torus emission is enhanced in quasars with larger $R_{FeII}$.
They suggests that this may be caused by more efficient disc winds that facilitate the formation of a dusty torus. 

\subsubsection{High-$z$}

In Fig.~\ref{fig:civ_hot_dust} we show how the ratio of NIR to UV luminosity depends on the blueshift and rest-frame \ac{EQW} of the \ion{C}{IV} line.
\ion{C}{IV} blueshifts are calculated as in Section XX. 
We see that the NIR to UV luminosity ratio is strongly correlated with the blue-shift of the \ion{C}{IV} emission line. 
A similar trend was noted by \citet{wang13}. 
Interestingly, we note strong similarities to the object subsets selected according to their \ion{C}{IV}-emission properties in \citet{richards11} (see Figures 11 \& 12).  
We note that the correlation between the hot dust and the \ion{C}{IV} emission properties will lead to apparent correlations between the host dust and the BH mass. 
\todo{Need to re-do this and understand why beta-related trend is apparently stronger than with the blackbody parameters.}
 
\begin{figure}
\centering
  \includegraphics[width=\columnwidth]{figures/chapter05/hot_dust_ratio.pdf}
  \caption{Rest-frame \ac{EQW} and blueshift of the \ion{C}{IV} line for 7,115 SDSS DR7 quasars. The colours of the hexagons denote the median hot dust (T$\simeq$1200\,K) abundance for all quasars at a given \ac{EQW} and blueshift. Quasars with the most extreme outflow signatures are predominantly hot-dust rich. Only bins containing a minimum of two objects are plotted.}
  \label{fig:civ_hot_dust}
\end{figure}

\begin{figure}
\centering
  \includegraphics[width=\columnwidth]{figures/chapter05/hot_dust_beta.pdf}
\caption{Rest-frame \ac{EQW}and blueshift of the \ion{C}{IV} line for 7,115 SDSS DR7 quasars. The colours of the hexagons denote the median hot dust (T$\simeq$1200\,K) abundance for all quasars at a given \ac{EQW} and blueshift. Quasars with the most extreme outflow signatures are predominantly hot-dust rich. Only bins containing a minimum of two objects are plotted. \todoinline{Change hot dust abundance.}}
  \label{fig:hot_dust_beta}
\end{figure}

\begin{figure}
\centering
  \includegraphics[width=\columnwidth]{figures/chapter05/blueshift_composite.pdf}
\caption{}
  \label{fig:blueshift_composite}
\end{figure}

\section{Discussion}

\citet{roseboom13} studied a similar sample of luminous type 1 quasars. 
They, like us, modelled the NIR emission using a blackbody and modelled the emission at longer wavelengths using a clumpy torus model. 
They find that while $L_{1-5\mu m}$/$L_{IR}$ appears relatively insensitive to $L_{bol}$ and $L_{IR}$, a strong correlation appears between $L_{1-5\mu m}$/$L_{IR}$ and $L_{IR}/L_{bol}$ (i.e. the dust covering factor). 
They explain this correlation by postulating that as the covering factor of the torus decreases, the maximum inclination at which a type 1 quasar would be seen increases. 
An increase in the inclination will mean direct sight lines to more of the inner wall of obscuring material closest to the accretion disc.

\citet{mor11} also looked at the hot dust properties of a sample of $0.75 < z < 2$ quasars, with photometry from SDSS and WISE. 
They modelled the NIR emission with hot clouds of pure graphite dust. 
They reported an anti-correlation between the covering factor of hot dust clouds and the quasar bolometric luminosity. 
Like us, they neglect cooler dust components which will dominate the SED at longer wavelengths. 
As we have discovered (see Figure residual plot), the missing flux decreases with redshift because we observe shorter rest-frame wavelengths when the observed spectrum is redshifted to a greater degree. 
This will induce an anti-correlation between the luminosity of the hot dust component and the luminosity of the quasar (which is correlated with redshift). 
At z=0.75, the W3 band-pass (the longest in their fits) is sensitive to flux from 6.9$\mu$m; at this wavelength we expect the contribution from cooler dust to dominate over the hot dust. 
It is possible that this effect could explain the tension with our own result that $R_{NIR/UV}$ does not depend on the quasar luminosity in our low-$z$ sample. 

\citet{shen14} quantify the relative torus emission using the $r-W1$ colour for a sample of $0.4 < z < 0.8$ SDSS quasars. 
At these redshifts W1 is observing between 1.9 and 2.4 microns in the rest-frame of the quasar, which suggests that they are sensitive to the same component of hot dust which we are investigating. 
They observe a mild trend of decreasing relative torus emission as the quasar luminosity increases. 
We note that their use of the r-W1 at much higher redshifts may be problematic, as the W1 flux will be increasingly dominated by direct emission from the accretion disc. 

\citet{gallagher07} undertook a similar investigation for a much smaller sample of 234 radio-quiet quasars.

Reverberation measurements of nearby AGNs suggest the near-infrared emission is dominated by hot dust very close to the central source \citep[few tens of light days; e.g.][]{minezaki04,suganuma06}. 
The hot dust signature could contain information about inner face of an obscuring torus structure and/or constrain the dust content of an accretion disc wind. 
Several studies have shown that the luminosity of the NIR excess emission correlates with that of the central engine with a slope close to unity \cite[e.g.][]{gallagher07}, suggesting that the dust is reprocessing radiation from the accretion disc. 

Outflows may emerge from the outer region of the accretion disc or even the innermost region of the torus, in which the gas clouds are dusty and relatively cold.  
Indeed, there is observational evidence for dusty outflows close to the central engine \citep[e.g.][]{bowler14}.
The dust is heated by the central engine, and radiates in the near-infrared band. 
\citet{wang13}, fitting the NIR emission with a single power-law, found that objects with strong outflow signatures (blue-shifted \ion{C}{IV}) have more hot dust emission relative to the accretion disc emission in a large sample of $z\sim2$ non-BAL quasars. 
It could be that this correlation is induced by a third factor that simultaneously affects outflows and dust emission, for instance the inclination angle or metallicity. 
Alternatively the dust could be intrinsic to outflows and may have a non-trivial contribution to the outflow acceleration.
Also found by \citet{shen14}. 

Several other investigations have drawn attention to the rest-frame near-infrared SEDs, with populations of `dust free' objects postulated \citep{hao10,hao11,jiang10,mor11} 


\subsection{Eddington ratio}

Wang et al., Zhang et al., and Mor \& Trakhtenbrot find no significant dependence of the amount of hot dust on the Eddington ratio. 
Is this because the Eddington ratio is wrong or because it's more complicated? (can high accretion objects with no evidence for strong outflows.)

\subsection{Spectral properties}

In the dusty wind model - first proposed by \citet{konigl94} and later developed by, amongst others, \citet{everett05}, \citet{elitzur06}, \citet{keating12} - the `torus' is the dusty part of a magneto-hydrodynamic wind beyond the dust sublimation radius. 
The MHD wind is roughly polar, and so the hot dust forms a vertical `wall' around the accretion disc.  
UV photons from the accretion disc accelerate the wind via radiation line driving. 
That flattens the geometry of the wind and exposes more surface area that is viewable on a relatively face-on line of sight.  
The radiation pressure is increased at higher luminosities and/or accretion rates.
This can flatten the geometry of the wind, thereby increasing the range of angles for which the inner edge of the dusty wind - where dust is at it's sublimation temperature - can be observed. 
A direct prediction is therefore that the in a quasars with high accretion rates and strong outflows, the emission from hot dust should be enhanced. 


\section{Further work}

What more is needed to test model(s)?  