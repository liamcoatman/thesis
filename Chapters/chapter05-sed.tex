% !TEX root = ../main.tex

%************************************************
\chapter{Outflows and hot dust emission}
\label{ch:sed} 

% Refer to: 
% HotDustPaper/
 % summary-150309.ipynb
 % notes.md
 % correlations_summary_141204.ipynb
 % correlations_summary.md
 % various *.md 

%************************************************

\section{Introduction}

Many quasars and AGNs show an excess in their near-infrared continuum emission. 
This feature is generally attributed to thermal emission from dust heated by optical/ultra-violet radiation from the accretion disc. 
The wavelength of the feature ($\sim2$\,$\mu$m) corresponds to the spectral peak for graphite dust at its sublimation temperature \citep[${\mathrm T}\sim1500$\,K;][]{barvainis87}. 
Reverberation measurements of nearby AGNs suggest that the hot dust is very close to the central source \citep[few tens of light days; e.g.][]{minezaki04,suganuma06}.
This places the dust at the innermost edge of the putative torus-like structure, at a radius set by the sublimation temperature of the dust grains.  

Studies have fitted the near-infrared SEDs of AGNs using a blackbody spectrum to represent emission from hot dust \citep[e.g.][]{edelson86,barvainis87,kishimoto07,mor09,riffel09,deo11,landt11,mor11,roseboom13}. 
A hot dust component is present in the vast majority of AGNs, although populations of `dust-free' objects have also been discovered \citep{hao10,hao11,jiang10,mor11}. 
It is not yet clear how the hot dust properties relate to other AGN properties such as BH mass, luminosity and accretion rate. 

In recent years the picture of the torus has evolved away from a static `doughnut' towards a more general circum-nuclear, geometrically and optically thick dust distribution. 
As we have previously discussed (e.g. Section~\ref{sec:ch1-winds}), winds and outflows launched from the accretion disc are very common in AGN. 
In the dusty wind model - first proposed by \citet{konigl94} and later developed by, amongst others, \citet{everett05}, \citet{elitzur06}, \citet{keating12} - the torus is the dusty part of an accretion disc wind beyond the dust sublimation radius.  
The dusty clouds are uplifted above the disc where they are directly exposed to the central engine. 
The dust is heated, and radiates in the near-infrared band.
At the same time, radiation pressure due to dust can efficiently accelerate the wind. 
The wind is roughly polar, and so naturally provides circum-nuclear obscuration around the accretion disc and dust-free BLR.   
This model is supported by recent interferometric observations of nearby Seyfert galaxies which find that the mid-infrared emission is dominated by dust in the polar regions \citep[e.g.][]{raban09,honig12,honig13,tristram14,lopez-gonzaga16}.

Studying the relationship between emission from hot dust and outflow diagnostics in the BLR can help place constraints on this dusty wind model \citep[e.g.][]{wang13}. 
This is now possible by combing data from the SDSS spectroscopic and photometric surveys with photometric surveys such as UKIDSS and WISE. 
At redshifts $2\lesssim z \lesssim3$, SDSS spectra reveal BH masses, accretion rates and diagnostics of the BLR dynamics. 
At the same time, the available photometric data provides full UV to infrared rest-frame coverage of the SED.
In particular, the WISE photometry is sensitive to the 3\,$\mu$m region of the SED which is dominated by hot dust. 

In this Chapter we build a simple parametric SED model that is able to reproduce the median optical-infrared colours of tens of thousands of SDSS AGN at redshifts $1 \lesssim z \lesssim 3$ (Section~\label{sec:ch5-standardmodel}).
We use this model to measure the hot dust properties of a large sample of $2 < z < 2.7$ quasars for which we have already measured \ion{C}{IV} line properties, BH masses and Eddington ratios (Section~\ref{sec:ch5-hotdust}).

\section{Data}

\begin{table}
  \footnotesize
  \centering
  \begin{tabular}{lcccc}
    \hline 
    Survey & Passband & $\lambda_{\mathrm eff}$ [$\mu$m] & AB offset & $A_{\mathrm filter}/E(B-V)$ \\
    \hline 
    SDSS & $u$ & $0.3543$ & $ 0.913$ & $4.875$ \\
         & $g$ & $0.4770$ & $-0.081$ & $3.793$ \\
         & $r$ & $0.6231$ & $ 0.169$ & $2.721$ \\
         & $i$ & $0.7625$ & $ 0.383$ & $2.099$ \\
         & $z$ & $0.9134$ & $ 0.542$ & $1.537$ \\
    UKIDSS & $Y$ & $1.0305$ &  $0.641$ & $1.194$ \\
           & $J$ & $1.2483$ &  $0.941$ & $0.880$ \\
           & $H$ & $1.6313$ &  $1.378$ & $0.569$ \\
           & $K$ & $2.2010$ &  $1.897$ & $0.352$ \\
    WISE & $W1$ & $3.4$ & $2.691$ & $0.182$\\
         & $W2$ & $4.6$ & $3.331$ & $0.130$\\
         & $W3$ & $12.0$ & & \\           
    \hline
  \end{tabular}
  \caption[{Available photometry, effective wavelength, Vega to AB magnitude offsets, conversion from \ebv to extinction.}]{Available photometry, effective wavelength, Vega to AB magnitude offsets, conversion from \ebv to extinction. \todoinline{Need $W3$ offset/ extinction. Ask Paul or just use WISE values/zero.}}
  \label{tab:photometry}
\end{table}

\subsection{SDSS}

We use spectroscopic data from the Seventh Data Release (DR$7$) of the SDSS spectroscopic quasar catalogue \citep{schneider10}.
The SDSS photometric survey obtained images in five broad optical passbands: $u$, $g$, $r$, $i$ and $z$ (Table~\ref{tab:photometry}).  
We use BEST point-spread function magnitudes from the SDSS DR$7$ quasar catalogue.

\subsubsection{UKIDSS Large Area Survey}

We use the tenth data release (DR$10$) of the UKIRT Infrared Deep Sky Survey \citep[UKIDSS;][]{lawrence07} Large Area Survey (ULAS) which has observed $\sim 3,200$ deg$^2$ in four near-infrared passbands: $Y$, $J$, $H$ and $K$. 
We use `apermag3' magnitudes, which are aperture corrected magnitudes in a $2''$ diameter aperture.

\subsubsection{WISE All-WISE Survey}

The Wide-field Infrared Explorer \citep[WISE;][]{wright10} mapped the entire sky in four mid-IR passbands: $W1$, $W2$, $W3$ and $W4$. 
The WISE AllWISE Data Release (`AllWISE') combines data from the nine-month cryogenic phase of the mission that led to the `AllSky' data release with data from the NEOWISE program \citep{mainzer11}. 
We use the profile-fitting `mpro' magnitudes.   
Only information from the first three WISE passbands are used in this work.

\subsection{Computing Vega-AB magnitude offsets}

Vega magnitudes are used throughout this Chapter. 
This is the native magnitude system for UKIDSS and WISE.
We add $0.08$\,mag to the UKIDSS $Y$ passband magnitudes to bring the photometry into better agreement with the SDSS $z$ and UKIDSS $J$ photometry. 
SDSS uses an `asinh' magnitude system \citep{lupton99} which is intended to be on the AB system \citep{oke83}.
However, the photometric zero-points are known to be slightly off the AB standard. 
The $z$ passband is in error by $0.02$ ($z_{\mathrm AB} = z_{\mathrm SDSS} + 0.02$)\footnote{http://classic.sdss.org/dr7/algorithms/fluxcal.html.}.
The $u$ passband was in error by $0.04$\,dex at the time of DR$7$. 
However, using an updated $u$ throughput function, we find that the $u$ zero-point is now consistent with the AB system. 

Using a reference template, we calculate the AB magnitude of Vega in each passband (Table~\ref{tab:photometry}). 
The mean flux density $f_\lambda(P)$ in a passband defined by a throughput function $P(\lambda)$ is given by: 

\begin{eqnarray}
\label{eq:flux}
  f_\lambda(P) = \frac {\int P(\lambda)f_\lambda(\lambda)\lambda d\lambda} {\int P(\lambda)\lambda d\lambda} 
\end{eqnarray}

where $f_\lambda(\lambda)$ is the flux density of the object. 
The predicted passband is then given by:   

\begin{eqnarray}
\label{eq:mag}
  m_\lambda(P) & = & -2.5{\mathrm log}(f_\lambda(P)) - m_0(P), 
\end{eqnarray}

where $m_0(P)$ is the zero-point magnitude of passband $P$, given by evaluating Equation~\ref{eq:flux} for a reference object. 
In the AB system this is a constant spectral flux density of $3631$\,Jy. 
In flux per unit wavelength this is:  

\begin{eqnarray}
  \frac{f_\lambda(\lambda)}{{\mathrm erg}~{\mathrm cm}^{-2}~{\mathrm s}^{-1} {\mathrm\AA}^{-1}} = 0.1087 \left(\frac{\lambda}{\mathrm \AA}\right)^{-2}.
\end{eqnarray}

In the Vega system, a spectrum of the A$0$V star Vega is used. 
Although the magnitude of Vega is by design zero in every passband, more recent measurements reveal a small magnitude offset.
The Vega to AB magnitudes given in Table~\ref{tab:photometry} assume Vega to have a magnitude $0.026$. 

\subsection{Galactic extinction correction}
\todo{Ask Paul how this is done, how much of a difference it makes}

$A(u)$, the Galactic extinction in the $u$ passband at the position of the object, is given in the SDSS catalogue. 
It is computed using the maps of \citet{schlegel98} and assuming a $z=0$ elliptical template. 
Quasar and galaxy SEDs have very different shapes in the optical/near-infrared region, and therefore we re-derive the extinction corrections using a $z=1.5$ quasar SED template. 

$A(u)$ is divided by $5.155$ to give \ebv, the relative extinction between the $B$ and $V$ passbands (see table 6 in \citealt{schlegel98}). 
Conversions from the selective extinction \ebv\, to the total extinction $A(\lambda)$ in each passband were calculated using a quasar SED template. 
These are given in the Table~\ref{tab:photometry}. 

\subsection{Cross-matching SDSS to UKIDSS and WISE}

There are $105\,783$ objects in the SDSS DR$7$ quasar catalogue. 
While WISE mapped virtually the entire sky, the UKIDSS footprint covers approximately one third of the SDSS footprint. 
$36\,607$ objects are cross-matched to the UKIDSS (with a $2''$ matching radius) and WISE (with a $3$$''$ matching radius) catalogues.

\subsection{Sample definition}

We include only the $20\,637$ quasars with $i$ passband magnitudes brighter than $19.1$, i.e. the quasars selected by the main $z<3$ SDSS quasar selection algorithm $z < 3$ \citep{richards02}. 
We verified that above the $i=19.1$ limit the sample is $95$ per cent complete in all passbands.
This suggests that our sample will not be biased towards quasars with redder spectra\footnote{For a given $i$ magnitude, a redder spectrum is more likely to be detected at longer wavelengths than a bluer spectrum.}. 
BAL quasars are excluded using the \citet{allen11} catalogue because the \ion{C}{IV} line parameters of these quasars can not be reliably measured.
This leaves $19\,837$  objects. 

We further limit our sample to the redshift range $1 < z < 3$. 
Galaxy SEDs peak at $\sim1$\,$\mu$m, and fall away towards shorter wavelengths. 
On the other hand, AGN SEDs continues to increase short-ward of $1$\,$\mu$m. 
As a result, the contrast between the AGN and galaxy luminosity increases as the redshift increases.
\todo{I don't fully understand this point}
Imposing a $z=1$ lower redshift limit on the redshift of our sample ensures that contributions to the SED from quasar host galaxies are negligible.
The completeness of the SDSS DR$7$ quasar selection algorithm decreases steeply beyond $z\sim3$, and this sets the upper redshift limit. 
The redshift and luminosity distribution of the final sample, containing $12\,934$ quasars, is shown in Figure~\ref{fig:lum_z}. 

\begin{figure}
  \centering
  \includegraphics[width=\textwidth]{figures/chapter05/lum_z.pdf}
  \caption[{Distribution of our sample in the redshift-luminosity plane.}]{Distribution of our sample in the redshift-luminosity plane.}
  \label{fig:lum_z}
\end{figure}

\section{Constructing an AGN SED model}

Since the physical processes that power AGN are generally understood only qualitatively, almost all AGN SED templates are empirical. 
The empirical template of \citet{elvis94} is still the most commonly cited, despite many additions and updates \citep[e.g.][]{polletta00, kuraszkiewicz03, risaliti04, richards06,  polletta07, lusso10, shang11, marchese12, trichas12}. 
However, these composite spectra are often constructed from quasars with a huge range in luminosity as a function of wavelength. 
In addition, the presence of significant host galaxy at optical wavelengths in low-redshift objects is an additional complication which has not always been taken care of adequately.
\todo{Ask Paul for details on this.} 
There is therefore a strong rationale for taking a parametric approach to modelling quasar SEDs. 
This the approach adopted in this work. 

We construct an SED model that is valid between $1216$\,\AA\, and $3$\,$\mu$m.
In this region the SED is dominated by the accretion disc, broad UV/optical emission-lines and thermal emission from the hottest (${\mathrm T}\sim1200$K) dust. 
In this Section, we describe how each of these components are represented in our parametric SED model.  
The effect of dust extinction at the AGN redshift is also incorporated into the model. 
At high redshifts, Ly$\alpha$ forest absorption becomes significant. 
Because we do not attempt to model this effect, our model is valid only at wavelengths long-ward of $1216$\,\AA. 
We model dust emission using a single temperature (${\mathrm T}\sim1200$\,K) blackbody, which peaks at $\sim2$\,$\mu$m. 
At longer wavelengths, emission from cooler dust farther from the central engine becomes increasingly important. 
We do not include this emission in our model, which restricts its validity to $\lesssim3$\,$\mu$m.
The model spectrum is shown in Figure \ref{fig:modelsed}, with each of the main components indicated. 

\begin{figure}[h!]
  \centering
  \includegraphics[width=\textwidth]{figures/chapter05/sed_model.pdf}
  \caption[{Model quasar spectrum at $z=1$, showing the contributions to the total flux from the accretion disc, Balmer continuum, hot dust and emission lines.}]{Model quasar spectrum at $z=1$, showing the contributions to the total flux from the accretion disc, Balmer continuum, hot dust and emission lines. }
  \label{fig:modelsed}
\end{figure}

\subsection{Accretion disc}

Thermal accretion disc emission in the $0.1 - 1$\,$\mu$m region is characterised by a broken power-law with three free parameters: a break-wavelength, $\lambda_{\mathrm break}$, a blue power-law index, $\alpha_{\mathrm blue}$, for wavelengths shorter than the break wavelength, and a red power-law index, $\alpha_{\mathrm red}$, for wavelengths longer than the break wavelength.

\subsection{Balmer continuum}

High order Balmer lines, optically thin Balmer continuum emission, two-photon emission and \ion{Fe}{II} emission blend together to form the `Balmer' continuum at $\sim3000$\AA.
We simulate the Balmer continuum using the empirical model given by \citet{grandi82}: 

\begin{eqnarray}
  F(\lambda) = C_{\mathrm BC} \times B_\lambda(T_e)(1-e^{-\tau_\lambda}); \quad \lambda \leq \lambda_{\mathrm BE}
\end{eqnarray}

where $C_{\mathrm BC}$ is a normalisation factor, $B_\lambda(T_e)$ is the Planck function, $T_e=13150$K is the effective temperature, $\lambda_{\mathrm BE}=3460$\AA\, is the wavelength at the Balmer edge, and $\tau_\lambda = \tau_{BE}\left( \nicefrac{\lambda_{BE}} {\lambda} \right)^{-3}$ is the optical depth with $\tau_{\mathrm BE}=45$ the optical depth at $\lambda_{\mathrm BE}$. 
This function is convolved with a Gaussian with width $\sigma=5000$\kms\, to simulate the effect of bulk velocity shifts comparable to those present in broad AGN emission-lines. 

\subsection{Hot dust}

Thermal emission from hot dust, which dominates the SED at wavelengths longer than $1$\,$\mu$m, is modeled using a blackbody

\begin{eqnarray}  
  F_\lambda = C_{\mathrm BB} \times \frac{2 hc^2}{\lambda^5}\frac{1}{ e^{\frac{hc}{\lambda k_\mathrm{B}T_{\mathrm BB}}} - 1}, 
\end{eqnarray}

with two free parameters: the temperature $T_{\mathrm BB}$ and normalisation $C_{\mathrm BB}$ relative to the power-law continuum. 

\subsection{Emission-lines}

We use an emission-line template taken from \citet{francis91}, which has been extended by \citet{maddox06} to include the \ha and Pa$\alpha$ emission-lines\footnote{The spectrum is not significantly different from the \citet{vandenberk01} SDSS composite.}. 
All emission-lines, with the exception of \ha and \hbns, are scaled using a single free parameter $C_{\mathrm EL}$, which preserves relative EQWs:

\begin{eqnarray}
  F_{\lambda} =  C_{\mathrm EL} \times \frac{F_{\lambda, \mathrm el}}{F_{\lambda, \mathrm cont}} \times F_{\lambda} 
\end{eqnarray} 

where $F_{\lambda, \mathrm el}$ is the emission-line template, $F_{\lambda,\mathrm cont}$ is the continuum flux in the template, and $F_{\lambda}$ is the continuum flux in the SED model.
The redshifts and luminosities of the quasars contributing to the emission-line template change as a function of wavelength. 
To account for possible variations in the strengths of the different lines, \ha and \hb are scaled separately: 

\begin{eqnarray}
  F_{\lambda} =  C_{\mathrm EL} \times C_{{\mathrm H} \alpha} \times \left( \frac{L(z)} {L(z_{\mathrm nrm})} \right)^{\beta} \times \frac{F_{\lambda, \mathrm el}}{F_{\lambda, \mathrm cont}} \times F_{\lambda}.
\end{eqnarray}

The luminosity dependence of the \ha and \hb EQW \citep[i.e. the Baldwin effect;][]{baldwin77} is parametrised with a power-law with slope $\beta=-0.04$.
The dependence of the mean AGN luminosity on redshift, $L(z)$, is determined empirically for the SDSS quasar catalogue. 

\subsection{Dust extinction}
\label{sec:sed-extinction} 

We simulate the effect of dust extinction at the quasar redshift using a custom extinction curve that is appropriate for the quasar population. 
To derive the quasar extinction curve, UKIDSS photometry was used to provide an \ebv\, estimate, via the magnitude displacement of each quasar from the locus of un-reddened objects. 
At redshifts $2 < z < 3$ the reddening measure is made at rest-frame wavelengths $3500-7000$\,\AA, where Galaxy, LMC and SMC\footnote{LMC and SMC: Large and Small Magellanic Clouds.} extinction curves are very similar. 
The SDSS spectra of the same objects are then employed to generate an empirical extinction curve in the ultra-violet, down to $1200$\,\AA. 
The resulting curve has no $2200$\,\AA\, feature and rises rapidly with decreasing wavelength but is not as steep as the SMC curve. 
The extinction curve gives the colour excess $E(B-\lambda)$ relative to the colour excess \ebv\, as a function of wavelength $\lambda$. 
The colour excess \ebv\, is related to the extinction in the $V$ passband, $A(V)$, via the ratio $R$: 

\begin{eqnarray}
  R_V = \frac{A(V)}{E(B-V)}
\end{eqnarray}

where we assume $R_V = 3$. 
Hence the extinction at a wavelength lambda $A(\lambda)$ is 

\begin{eqnarray}
  A(\lambda) = E(B-V) \times \left[ \frac{E(\lambda-V)}{E(B-V)} + R \right] 
\end{eqnarray}

where the colour excess \ebv\, is a free parameter in our model. 
The attenuation of the flux at a given wavelength is then:

\begin{eqnarray}
  F_\lambda = F_\lambda10^{-A(\lambda)/2.5}
\end{eqnarray}

in the rest frame of the quasar. 

\section{Deriving a SED template for the quasar population} 
\label{sec:ch5-standardmodel}

In this Section we determine a single set of SED model parameters for all $19\,853$ quasars, encompassing a range of redshifts, luminosities, accretion rates etc. 

\subsection{Fitting procedure}

\begin{figure}
  \centering
  \includegraphics[width=\textwidth]{figures/chapter05/throughput.pdf}
  \caption[{Model quasar spectrum at three different redshifts, and throughput functions for SDSS, UKIDSS and WISE passbands.}]{Model quasar spectrum at three different redshifts (each arbitrarily scaled), and throughput functions for SDSS, UKIDSS and WISE passbands.}
  \label{fig:filters}
\end{figure}

The free parameters in our SED model are summarised in Table~\ref{tab:params}. 
The reddening \ebv\, is fixed to zero, since a large fraction of SDSS quasars have very small amounts of dust reddening \citep{richards03}. 

We divide the quasar sample into redshift bins from $z=1$ to $z=3$ in intervals of $\Delta z = 0.1$.
In each redshift bin median passband magnitudes are calculated, and normalised such that $i=18$.
As described above, the SED model is valid between $\sim1200$ and $30000$\,\AA.  
We use the $rizYJHKW1W2$ passbands to constrain the model, which covers $1550-23000$\,\AA\, in the rest-frame. 
Model SEDs are generated at redshifts corresponding to the centres of the redshift bins.
The SED model is shown at three different redshifts in Figure~\ref{fig:filters}.
Model magnitudes are calculated using Equations~\ref{eq:flux} and \ref{eq:mag} and are normalised such that $i=18$.
We find the best-fitting model parameters by minimising the $\chi^2$ statistic for the $9 \times 21 = 181$ model and data magnitudes using the Nelder-Mead algorithm. 

\subsection{Results from fit}

\begin{table}
  \footnotesize
  \centering
  \begin{tabular}{c c c}
    \hline 
    Parameter & Symbol & Value \\
    \hline 
    Blue power-law index & $\alpha_{\mathrm blue}$ & $-0.478$ \\
    Red power-law index & $\alpha_{\mathrm red}$ & $-0.199$ \\
    Power-law break & $\lambda_{\mathrm break}$ & $2402$ \\
    Blackbody temperature & $T_{\mathrm BB}$ & $1306$\,K \\
    Blackbody normalisation & $C_{\mathrm BB}$ & $2.673$ \\
    Emission-line scaling & $C_{\mathrm EL}$  & $1.240$ \\
    \ha emission-line scaling & $C_{{\mathrm H}\alpha}$  & $0.713$ \\
    Balmer continuum scaling & & $0.135$ \\
    \hline
  \end{tabular}
  \caption{Best-fitting SED model parameters from fit to the median colours of quasars at redshifts $1 < z < 3$.}
  \label{tab:params}
\end{table}

The best-fitting parameters from the fit are given in Table~\ref{tab:params}. 
The colours ($r-i$, $i-z$, etc.) of the median SED and the best-fitting model are plotted as a function of redshift in Figure~\ref{fig:color}.
Most of the large variations that can be seen in the median colours of the quasars as a function of redshift are due to strong emission-lines being redshifted into and out of the passbands.

\begin{figure}[t!]
\includegraphics[width=\textwidth]{figures/chapter05/sed_color_plot_1.pdf}
\caption[{Median colours of quasars as a function of redshift and best-fitting SED model.}]{Median colours of quasars as a function of redshift and best-fitting SED model.}
  \label{fig:color}
\end{figure} 

\begin{figure}[t!]
\ContinuedFloat
    \centering
    \includegraphics[width=\columnwidth]{figures/chapter05/sed_color_plot_2.pdf} 
    \caption{Continued.}     
\end{figure}

In Figure~\ref{fig:residuals} we show the data minus model residuals as a function of the rest-frame wavelength. 
The residuals indicate that over a large redshift range the model is very effective at reproducing the median observed colours of the sample. 
Discrepancies are at the $<0.1$\,mag level. 
At a given rest-frame wavelength, there are no significant discrepancies between residuals in different passbands. 
This indicates that there is no significant evolution in the median SED as a function of redshift. 
We conclude that a single, fairly simple parametric is effective at reproducing the median colours of tens of thousands of AGN with a large dynamic range in redshift and luminosity.

\begin{figure}[t!]
  \centering
  \includegraphics[width=\textwidth]{figures/chapter05/model_residuals.pdf}
  \caption[{Residuals from fit as a function of rest-frame wavelength.}]{Residuals from fitting a single SED model to the colours of $1 < z < 3$ quasars as a function of rest-frame wavelength.}
  \label{fig:residuals}
\end{figure}

\section{Diversity of hot dust properties}
\label{sec:ch5-hotdust}

In Figure~\ref{fig:w1w2colorsratio} we plot the $W1 - W2$ colours of the sample as a function of redshift. 
At any given redshift we see a $\sim 0.5$\,mag dispersion in the $W1-W2$ colours. 
In this redshift range the $W1$ and $W2$ passbands are probing the $1.2 - 2.8$\,$\mu$m and $1.6 - 3.8$\,$\mu$m regions of the rest frame SED respectively. 
The peak wavelength is at $2.4$\,$\mu$m for a blackbody radiating at $1200$\,K.
Therefore, the large spread in $W1-W2$ colours is highly suggestive of a range of hot dust properties in this sample. 

We characterise the hot dust properties of our sample in terms of the temperature of the blackbody component and the near-infrared to ultra-violet luminosity ratio, $R_{\mathrm NIR/UV}$. 
The ultra-violet and near-infrared luminosities are calculated between $2000$ and $9000$\,\AA\, and $1$ and $3$\,$\mu$m respectively in the SED.

The temperature is likely related to the distance of the dust from the central engine.
Dust that is closer in will be hotter, with the sublimation temperature of the dust grains setting the minimum radius.  
The value of $R_{\mathrm NIR/UV}$ is related to the covering factor of the hot dust. 
However, this simple interpretation is somewhat complicated by the fact that at large inclinations sight-lines to the hot dust may be obscured by cooler dust in the putative torus. 

In Figure~\ref{fig:w1w2colorsratio} we have plotted the $W1-W2$ colours derived from our SED model with a fixed blackbody temperature ($1306$\,K) and varying $R_{\mathrm NIR/UV}$.
The $W1-W2$ colours indicate that the hot dust luminosity in this sample varies by a factor of $\sim5$. 

In the rest of this Chapter, we will measure hot dust parameters for individual quasars via SED fitting. 
We will characterise the range of hot dust properties present in the sample, and test its relation to quasar properties such as luminosity, black-hole mass and normalised accretion rate, and outflow-properties. 

\begin{figure}[t!]
\centering
\includegraphics[width=\columnwidth]{figures/chapter05/w1w2_versus_redshift_ratio.pdf}
\caption[{$W1 - W2$ colours of sample as a function of redshift.}]{$W1 - W2$ colours of sample as a function of redshift. Above a density threshold of four points per pixel points are represented by a two-dimensional histogram. On top we plot the colours of our standard SED model, with a fixed temperature and a varying near-infrared ($1 - 3$\,$\mu$m) to ultra-violet ratio.}
  \label{fig:w1w2colorsratio}
\end{figure}

\subsection{Defining a sample with uniform UV/optical properties}

We limit our sample to $2329$ quasars with redshifts $2 < z < 2.7$. 
At these redshifts, \ion{C}{IV} emission is located within the wavelength coverage of the SDSS spectra.
This will allow us to test the relationship between the hot dust and BLR outflow properties. 

The shifting of passbands due to redshift limits the redshift range of the quasars for which hot dust properties can be reliably constrained.
Constraining a ${\mathrm T}\sim1200$\,K blackbody component in the SED model requires photometric data covering $\sim1-3$\,$\mu$m in the rest-frame of the quasar. 
At redshifts $2 < z < 2.7$ $W1W2W3$ is probing the $0.9-4$\,$\mu$m region of the SED. 
Therefore the available data is sensitive to the hot dust component across the entirety of the redshift interval. 

In general, care must be taken looking for trends with luminosity (and related properties including the BH mass and Eddington ratio) given the observed-frame passband information on the rest-frame SED can produce some strong systematics with redshift.
However, the redshift interval is narrow enough to prevent this from being a significant problem. 

Holding the rest of the model parameters fixed, we will vary only the parameters of the blackbody. 
This requires the SED model to be a reasonable fit to the quasar SEDs in the ultra-violet/optical region. 
In practice, this means excluding objects with extreme emission-line EQWs and/or significant dust extinction.
We use $i-K$ as a measure of the overall colour of the quasars as it provides the longest baseline in wavelength without being affected by absorption in the Ly$\alpha$ forest at high redshifts. 
$i-K$ colours are shown as a function of redshift in Figure~\ref{fig:ikzplot}.
In the same plot we show the quasar SED model with $E(B-V)=-0.075,0,0.075$. 
A significant amount of the scatter in $i-K$ can be attributed to intrinsic variations in the ultra-violet power-law slopes of the individual quasars, which is why we allow a negative `reddening'. 

The SDSS and UKIDSS photometry are separated by $3-4$ years in the source rest-frame. 
Therefore, some of the $i-K$ scatter could be due to temporal variations in the brightness of the targets. 
However, the red-asymmetry of the $i-K$ colours about the un-reddened SED model suggests that this effect is sub-dominant to intrinsic colour differences. 
We discarded from our sample quasars with $i - K$ colours redder than our standard model with dust reddening \ebv $= 0.075$ and bluer than \ebv $=-0.075$ (Figure~\ref{fig:ikzplot}). 
Following this cut we are left with $2030$ quasars in our high-$z$ sample. 

\begin{figure}[t!]
  \centering
  \includegraphics[width=\columnwidth]{figures/chapter05/ik_versus_z_low_ext.pdf}
  \caption[{$i-K$ colours of non-BAL DR$7$ quasars with $i>19.1$ as a function of redshift.}]{$i-K$ colours of non-BAL DR$7$Q quasars with $i>19.1$ as a function of redshift. The lines show the colours of our model with varying amounts of dust extinction. Quasars with extinction $|E(B-V)|>0.075$ are excluded.}
  \label{fig:ikzplot}
\end{figure}


\subsection{Fitting procedure}

We will fit a model to the individual quasar SEDs, allowing the temperature and normalisation of the blackbody component to vary. 
The model spectrum is redshifted to the redshift of the quasar being fit and passband magnitudes are calculated using Equations~\ref{eq:flux} and \ref{eq:mag}.   
We minimise the inverse variance weighted chi-squared statistic using the Levenberg-Marquardt algorithm. 
We impose a minimum error of 0.1\,mag, corresponding to the model error for the medians colours (Figure~\ref{fig:residuals}). 
Data from $ugrizYJHKW1W2W3$ is used in the model. 
However, to avoid Ly$\alpha$ forest absorption, passbands are excluded if $\lambda_{\mathrm eff} < 1400$\AA.   

\subsection{Distribution of hot dust parameters}

The best-fitting hot dust temperature ($T_{\mathrm BB}$) and abundance ($R_{\mathrm NIR/UV}$) for the individual quasars are shown in Figure~\ref{fig:ratio_tbb_density}.
Even after restricting the sample to have a relative narrow range of ultra-violet/optical SED shapes, we see significant diversity in the hot dust abundance, with the near-infrared ultra-violet luminosity ratio having a broad range from $0.1$ to $0.6$.
The temperature takes on a relatively narrow range of values: $1177\pm136$\,K. 
This is consistent with the dust radius being set by the sublimation temperature of the dust grains. 

We note a strong correlation between the temperature and $R_{\mathrm NIR/UV}$. 
This is a result of the dependence of the blackbody peak on temperature and the fixed wavelength interval used to calculate the blackbody luminosity. 

\begin{figure}[t!]
  \centering
  \includegraphics[width=0.9\textwidth]{figures/chapter05/ratio_tbb_density.pdf}
  \caption[{Ratio of near-infrared to ultra-violet luminosity ($R_{NIR/UV}$) against temperature ($T_{BB}$) for low-$z$ sample.}]{Histograms of the ratio of near-infrared to ultra-violet luminosity ($R_{NIR/UV}$) and blackbody ($T_{BB}$) and the correlation between these two parameters. }
  \label{fig:ratio_tbb_density}
\end{figure}

\subsection{Relationship between hot dust and BLR outflows}

\begin{figure}[t!]
\centering
  \includegraphics[width=0.9\textwidth]{figures/chapter05/hot_dust_ratio.pdf}
\caption[{Hot dust abundance as a function of rest-frame EQW and blueshift of the \ion{C}{IV} line.}]{Rest-frame EQW and blueshift of the \ion{C}{IV} line. The colours of the hexagons denote the median hot dust (T$\simeq$1200\,K) abundance for all quasars at a given EQW and blueshift. Quasars with the most extreme outflow signatures are predominantly hot-dust rich.}
  \label{fig:civ_hot_dust}
\end{figure}

In this Section, we compare the hot dust parameters to the blueshift and EQW of the \ion{C}{IV} emission. 
\ion{C}{IV} blueshift measurements are described in Section~\ref{sec:ch3-application}. 
The \ion{C}{IV} blueshift is defined with respect to a systemic redshift measured by Allen \& Hewett (2017, in preparation). 
This information is available for $98$ per cent of the objects in our sample. 

In Figure~\ref{fig:civ_hot_dust} we show the that near-infrared to ultra-violet luminosity ratio is correlated with the \ion{C}{IV} blueshift. 
A similar result was recently reported by \citet{wang13}. 
On the other hand, we find no correlations with the hot dust temperature. 

The profiles of the emission-lines with large \ion{C}{IV} blueshifts suggest that the BLR dynamics in these objects are dominated by high-velocity outflows. 
As we discussed at the beginning of this Chapter, outflows from further out in the accretion disc could contain significant amounts of dust. 
As the dusty wind is lifted above the accretion disc, it would be directly exposed to ultra-violet radiation from the inner accretion disc. 
Radiation pressure could efficiently accelerate the wind owing to the high cross-section of the dust grains \citep[e.g.][]{fabian12}.  

Radiation pressure could flatten the geometry of the wind. 
The greater the radiation pressure, the flatter the geometry of the wind. 
This leads to a wider opening angle, which exposes more surface area that is viewable on a relatively face-on line of sight. 
This leads to the enhanced hot dust emission we observe in the quasars with high accretion rates and strong outflows. 

A prediction of this model is an anti-correlation between the torus covering factor and the hot dust abundance. 
The torus covering factor would be reduced by the accretion disc wind, and this would increase the maximum inclination at which a type I quasar could be seen. 
This would mean direct sight lines to more of the dust closest to the accretion disc. 
Such a correlation has been identified by \citet{roseboom13}. 

At lower redshifts, \citet{shen14} have found that the hot dust properties are correlated with EV$1$. 
\citet{shen14} quantify the relative torus emission using the $r-W1$ colour for a sample of $0.4 < z < 0.8$ SDSS quasars. 
At these redshifts $W1$ is observing between $1.9$ and $2.4$\,$\mu$m in the rest-frame of the quasar, which suggests that they are sensitive to the same component of hot dust which we are investigating. 
\citet{shen14} also find that torus emission is enhanced in quasars with larger $R_{FeII}$.
Our work connecting the EV$1$ correlations observed in low-redshift AGN with the diversity of \ion{C}{IV} emission properties in the high-redshift quasar population (Chapter~\ref{ch:bhmass}) reveals a correlation between the \ion{C}{IV} blueshift and the \ion{Fe}{II} EQW.
Therefore, our findings are consistent with \citet{shen14}, in a different luminosity/redshift regime. 


\subsection{Correlations with quasar properties}

\begin{figure}
\captionsetup[subfigure]{labelformat=empty}  
  \centering
  \subfloat[\label{fig:correlations_contour_a}]{}
  \subfloat[\label{fig:correlations_contour_b}]{}
  \subfloat[\label{fig:correlations_contour_c}]{}
  \subfloat[\label{fig:correlations_contour_d}]{}
  \subfloat[\label{fig:correlations_contour_e}]{}
  \subfloat[]{{\includegraphics[width=\textwidth]{figures/chapter05/correlations_contour.pdf} }}
  \caption[{Best-fit near-infrared to ultra-violet luminosity ($R_{NIR/UV}$) as a function of ultra-violet luminosity, BH mass and Eddington ratio.}]{Best-fit near-infrared to ultra-violet luminosity ($R_{NIR/UV}$) as a function of ultra-violet luminosity, BH mass and Eddington ratio. In (b) and (c) BH mass estimates, which are based on the \ion{C}{IV} FWHM, are taken from \citet{shen11}. In (d) and (e) BH mass estimates have been corrected using the procedure described in Chapter~\ref{ch:bhmass}. Using the corrected masses the correlations between $R_{NIR/UV}$ and the BH mass and Eddington ratio are significantly reduced.}
  \label{fig:correlations_contour}
\end{figure}

The distribution of blackbody temperatures is relatively narrow $\sim200$\,K and so, as expected, we do not observe any correlations between the temperature and other quasar properties. 
Correlations between the hot dust abundance $R_{\mathrm NIR/UV}$ and the ultra-violet luminosity, BH mass and Eddington ratio are shown in Figure~\ref{fig:correlations_contour}. 

The ultra-violet luminosity is measured at $1350$\,\AA\, in the spectral modelling done by \citet{shen11}. 
We do not observe any correlation between $R_{\mathrm NIR/UV}$ and the ultra-violet luminosity (Figure~\ref{fig:correlations_contour_a}. 
However, the dynamic range in luminosity is small in this sample ($\sim1$\,dex) because of the restricted redshift range. 

In Figure~\ref{fig:correlations_contour_b} we show $R_{\mathrm NIR/UV}$ as a function of the BH mass. 
Masses are \ion{C}{IV} FWHM-based single-epoch virial estimates, as computed by \citet{shen11}. 
$R_{\mathrm NIR/UV}$ is positively correlated with the BH mass; the Spearman correlation coefficient, $\rho_{\mathrm S}$, is $0.26$. 
However, in the previous Section, we found $R_{\mathrm NIR/UV}$ to be correlated with the \ion{C}{IV} blueshift and, in Chapter~\ref{ch:bhmass}, we demonstrated that BH masses are severely overestimated in quasars with large \ion{C}{IV} blueshifts.
We therefore predict that the apparent correlation between $R_{\mathrm NIR/UV}$ and the BH mass is due to systematic biases in the \ion{C}{IV}-based masses. 
We can test this by comparing $R_{\mathrm NIR/UV}$ with BH mass estimates which have been corrected using the prescription described in Chapter~\ref{ch:bhmass} (Figure~\ref{fig:correlations_contour_d}). 
As expected, the correlation vanishes ($\rho_{\mathrm S}=0.07$)
A similar result is found when $R_{\mathrm NIR/UV}$ is compared to the Eddington ratio, which is inversely proportional to the BH mass.  
Using masses from \citet{shen11} an anti-correlation is observed between $R_{\mathrm NIR/UV}$ and the Eddington ratio ($\rho_{\mathrm S}=-0.36$; Figure~\ref{fig:correlations_contour_c}) but this disappears when corrected masses are employed ($\rho_{\mathrm S}=-0.12$; Figure~\ref{fig:correlations_contour_e}). 
This demonstrates how using conventional BH mass estimates based on \ion{C}{IV} can lead to spurious correlations with other quasar properties, but that this can be avoided using the improved mass estimates presented in Chapter~\ref{ch:bhmass}. 

