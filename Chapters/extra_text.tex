Put somewhere: 


Punctuated fuelling episodes, e.g. driven by galaxy mergers, satellite accretion and even secular processes,
almost certainly lead to AGN experiencing activity-, outflow- and obscuration-dominated cycles with some overlap between phases. 
However, quantitatively, it remains unclear how these phases relate to the fundamental properties of the accreting black-hole (e.g.  mass (M$_{\rm{BH}}$), bolometric luminosity (L$_{\rm{bol}}$) and Eddington ratio (L/L$_{\rm{Edd}}$) and the elements of the non-spherical geometry).



Near-infrared spectra of highly reddened quasars (Banerji et al. 2012, 2013, 2014) are already available, with many of the H$\alpha$ line profiles showing strong asymmetries, indicative of outflows. 
Similar line profiles are also seen for a population of optically luminous, submillimetre-bright quasars (Orellana et al. 2011). 
H$\alpha$ line profiles for our optically-selected sample, will show if the H$\alpha$ outflow signatures are ubiquitous in the luminous AGN population or if they are only associated with rare/short-lived phases in the AGN cycle represented by the reddened/submillimetre-bright quasars.
If the highly dust-reddened quasars are being observed in a blowout/feedback phase in galaxy formation the NLR may have been swept away, depending on the duration of the fast outflow phase.

