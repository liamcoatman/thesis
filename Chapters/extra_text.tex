Put somewhere: 


Punctuated fuelling episodes, e.g. driven by galaxy mergers, satellite accretion and even secular processes,
almost certainly lead to AGN experiencing activity-, outflow- and obscuration-dominated cycles with some overlap between phases. 
However, quantitatively, it remains unclear how these phases relate to the fundamental properties of the accreting black-hole (e.g.  mass (M$_{\rm{BH}}$), bolometric luminosity (L$_{\rm{bol}}$) and Eddington ratio (L/L$_{\rm{Edd}}$) and the elements of the non-spherical geometry).



Near-infrared spectra of highly reddened quasars (Banerji et al. 2012, 2013, 2014) are already available, with many of the H$\alpha$ line profiles showing strong asymmetries, indicative of outflows. 
Similar line profiles are also seen for a population of optically luminous, submillimetre-bright quasars (Orellana et al. 2011). 
H$\alpha$ line profiles for our optically-selected sample, will show if the H$\alpha$ outflow signatures are ubiquitous in the luminous AGN population or if they are only associated with rare/short-lived phases in the AGN cycle represented by the reddened/submillimetre-bright quasars.
If the highly dust-reddened quasars are being observed in a blowout/feedback phase in galaxy formation the NLR may have been swept away, depending on the duration of the fast outflow phase.



Significant diversity in quasar SEDs is observed, the most obvious difference being the Type I/II dichotomy, which is explained by orientation-based unification schemes (Antonucci 1993). 
Such schemes require an anisotropic parsec scale obscuring structure that surrounds the central accreting black hole (e.g., Krolik \& Begelman 1988; Antonucci 1993). 
In this picture, the bulk of the radiation from the central engine is absorbed by the obscuring structure (commonly referred to as the torus) and re-emitted mainly in mid-infrared (MIR) wavelengths.  

At the same time, AGN-driven outflows are present in a large fraction of the luminous quasar population. 
The mass and energy associated with these outflows is believed to be significant in the context of feedback and its effect on the host galaxy. 

While orientation-based models, including a parsec-scale obscuring structure, have been successful in explaining some of the diverse properties of quasars (most notably the difference between type 1 and type 2 quasars), quasars are also likely to evolve through fuelling phases. 
Numerical simulations indicate that feedback is triggered in a fueling phase during a gas-rich galaxy merger (Hopkins et al. 2008; Narayanan et al. 2010), satellite accretion or secular processes (e.g. Fanidakis et al. 2012). 
Objects in the early feedback phase are likely to be both highly luminous (accreting close to the Eddington limit) and highly obscured by dusty inflowing material (Haas et al. 2003). 
In time, the energy output of the  central engine becomes sufficiently powerful to drive outflows, which can blow away the obscuring clouds of dust and gas. 
The quasar emission is then relatively unobscured until it declines as a result of the depletion of the available gas reservoirs. 
As the quasar transitions through these stages in its lifetime, key observational properties, such as the luminosity and the SED, are likely to change. 
Quantitatively, however, it remains unclear how these phases relate to the fundamental properties of the accreting black-hole (e.g.  mass (M$_{\rm{BH}}$), bolometric luminosity (L$_{\rm{bol}}$) and Eddington ratio (L/L$_{\rm{Edd}}$) and the elements of the non-spherical geometry). 
However, multiple authors have found no significant dependence of the spectral energy distribution (SED) on properties such as redshift, bolometric luminosity, SMBH mass, or accretion rate (e.g. Elvis et al. 2012, Hao et al. 2013) and quasars up to redshift 7 have been shown to have similar UV spectra to low redshift quasars (e.g. Mortlock et al. 2011). 

In one model, quasars are surrounded by an inner `wall' of gas and dust, in a cylinder-like geometry. 
As a radiatively-driven outflow develops, material at the extremes of the cylinder is driven-off, exposing more of the inner edge of the obscuring material at the equator (a `torus'), which contains the hot dust. 
Quasars are also observed with a wide distribution of reddening from dust on galactic scales. 
It is believed that luminous highly dust-reddened quasars may be in the process of expelling their dust and transitioning to ultra-violet bright objects which make up to the majority of the SDSS sample. 
Outflow signatures are very common in the rest-frame optical \ha line profiles of the heavily reddened quasars (Banerji et al. 2012, 2013), but it isn't clear how significant these outflows are on larger, kiloparsec scales.


Is there any evidence for the inner edge of the torus being further from the nucleus in more luminous quasars (i.e. a decrease in near-IR to optical/UV luminosity ratio with increasing optical/UV luminosity)? Are there quasars in our sample which are deficient in hot dust and, if so, are these objects being observed before a dusty torus has formed or are the torus and accretion disk misaligned? 

In one model, quasars are surrounded by an inner `wall' of gas and dust, in a cylinder-like geometry. 
As a radiatively-driven outflow develops, material at the extremes of the cylinder is driven-off, exposing more of the inner edge of the obscuring material at the equator (a `torus'), which contains the hot dust. 


\begin{figure}
\centering
  \includegraphics[width=\columnwidth]{figures/chapter06/ntt_proposal_figure2.pdf}
\caption{Representative rest-frame model spectra for the most hot-dust rich and hot-dust poor quasars in our SDSS sample, with error bars indicating the range of {\it WISE} W1, W2, and W3 magnitudes for the objects in these subsamples, transformed to the rest frame of the quasar. Grey lines show the W1, W2, and W3 fluxes for our sample of highly reddened quasars.}
  \label{fig:}
\end{figure}


Makoto Kishimoto: Slide 12 picture 

\subsection{Schemes that depend on viewing angle}

Disc emission is theoretically predicted to scale with the cosine of the disc inclination angle. 
If hot dust emission is (approximately) isotropic, the amount of hot dust emission relative to the accretion disc emission is expected to increase with increasing IA. 
However, the dust torus may block the hot dust emission at large IA \citep{roseboom13}. 
It complicates the situation and makes the dependence of $R_{NIR/UV}$ on IA uncertain. 
Runnoe et al. 2013 investiaged the inclination dependence of quasar SEDs. 
Their Figure 6 llustates that an edge on quasar spectrum tends to show a slightly enhanced NIR bump.

\citet{shen14} observed that, at fixed $R_{FeII}$, torus emission is enhanced when $FWHM_{H\beta}$ increases. 
\hbns’s width relates to the orbital velocity of gas along our line of sight. 
They conclude that \hbns's width reveals the orientation of the quasar's disc to our line of sight, with a wider line corresponding to a more edge-on disk. 
The width of other broad emission lines (e.g. \ion{Mg}{II}) should show the same dependance.   

\subsection{Dust-Free Outflow Scenario}

Correlations shown in Wang et al. and Zhang et al. could be induced by a third factor that simultaneously governs or relates to outflows and dust emission. 
Assume that hot dust emission is not directly related to outflows and is predominantly emitted by the innermost part of a hydrostatic and optically thick torus, as suggested by lots of previous studies.

1. Outflow strength is significantly dependent on the Eddington ratio, broadband SED (e.g. ionisation SED, UV continuum slope, $\beta_{UV}$) and luminosity. 
If these factors also have an impact on the hot dust emission, one might find a piece of evidence to support the dust-free outflow scenario. 
As shown in Zhang et al., Wang et al., Mor \& Trakhtenbrot, $\beta$ is almost independent of the Eddington ratio. And no correlation with luminosity. 

2. Metallicity. Quasars harboring strong outflows tend to have high gas metallicity. 
Wang et al. (2012) found that CIV blueshift increases with gas metallicity. 
Since dust forms more easily in higher metallicity gas, one may expect that metallicity is an appropriate factor which simultaneously influences outflow and hot dust emission. 
However, for a typical torus, the relative amount of dust emission is determined by the dust covering factor but not by the dust amount. 

\subsection{Dusty outflows}

BALs are on average redder than non-BAL quasars and quasars with large CIV blueshift are on average bluer than those with small blueshift. 
Since BAL and BEL outflows very likely represent the same physical component, this seeming contradiction can be reconciled by a special geometrical configuration in which dust is associated with outflows. 
In this case, dust reddening is preferably observed in the outflow direction (Wang et al.) 

Either outflows originally contain dust or dust is manufactured in outflows (Elvis et al. 2002). 

1. Dust is intrinsic to outflows. 
The idea is supported by the similar locations of BAL outflows and hot dust. 
Reverberation mapping results suggested that hot dust is located at the outer boundary of BEL regions (Suganuma et al. 2006). 
The outflow launching region is also suggested to be co-spatial with or outside of the BEL regions. 
Outflows may emerge from the outer region of the accretion disc or even the innermost region of the torus, in which the gas clouds are dusty and relatively cold. 
The dusty clouds are uplifted above the disk and are exposed to the central engine. 
The low density part is highly ionized and responsible for the blueshifted absorption and emission lines. 
Dust survives in the dense region and radiates in the NIR band. 
The dust carried by outflows is heated by the central engine to emit at the sublimation temperature, meanwhile dust absorption contributes to the outflow acceleration. 

2. Outflows interact with torus clouds. 
The process makes more dust in the clouds exposed to the central source. 
As a consequence, the IR emission increases, and the outflows become dusty. 
Since a stronger outflow can ablate dense clouds more effectively, the dependence of NIR emission on outflow strength is yielded. 
In this scenario, outflows confine the geometry and subtending angle of the dusty torus. 
One problem is that the interaction timescale is much shorter than the quasar lifetime. 
The correlation might disappear after outflows blow away all of the clouds in the outflow direction. 