Put somewhere: 


Punctuated fuelling episodes, e.g. driven by galaxy mergers, satellite accretion and even secular processes,
almost certainly lead to AGN experiencing activity-, outflow- and obscuration-dominated cycles with some overlap between phases. 
However, quantitatively, it remains unclear how these phases relate to the fundamental properties of the accreting black-hole (e.g.  mass (M$_{\rm{BH}}$), bolometric luminosity (L$_{\rm{bol}}$) and Eddington ratio (L/L$_{\rm{Edd}}$) and the elements of the non-spherical geometry).



Near-infrared spectra of highly reddened quasars (Banerji et al. 2012, 2013, 2014) are already available, with many of the H$\alpha$ line profiles showing strong asymmetries, indicative of outflows. 
Similar line profiles are also seen for a population of optically luminous, submillimetre-bright quasars (Orellana et al. 2011). 
H$\alpha$ line profiles for our optically-selected sample, will show if the H$\alpha$ outflow signatures are ubiquitous in the luminous AGN population or if they are only associated with rare/short-lived phases in the AGN cycle represented by the reddened/submillimetre-bright quasars.
If the highly dust-reddened quasars are being observed in a blowout/feedback phase in galaxy formation the NLR may have been swept away, depending on the duration of the fast outflow phase.



Significant diversity in quasar SEDs is observed, the most obvious difference being the Type I/II dichotomy, which is explained by orientation-based unification schemes (Antonucci 1993). 
Such schemes require an anisotropic parsec scale obscuring structure that surrounds the central accreting black hole (e.g., Krolik \& Begelman 1988; Antonucci 1993). 
In this picture, the bulk of the radiation from the central engine is absorbed by the obscuring structure (commonly referred to as the torus) and re-emitted mainly in mid-infrared (MIR) wavelengths.  

At the same time, AGN-driven outflows are present in a large fraction of the luminous quasar population. 
The mass and energy associated with these outflows is believed to be significant in the context of feedback and its effect on the host galaxy. 

While orientation-based models, including a parsec-scale obscuring structure, have been successful in explaining some of the diverse properties of quasars (most notably the difference between type 1 and type 2 quasars), quasars are also likely to evolve through fuelling phases. 
Numerical simulations indicate that feedback is triggered in a fueling phase during a gas-rich galaxy merger (Hopkins et al. 2008; Narayanan et al. 2010), satellite accretion or secular processes (e.g. Fanidakis et al. 2012). 
Objects in the early feedback phase are likely to be both highly luminous (accreting close to the Eddington limit) and highly obscured by dusty inflowing material (Haas et al. 2003). 
In time, the energy output of the  central engine becomes sufficiently powerful to drive outflows, which can blow away the obscuring clouds of dust and gas. 
The quasar emission is then relatively unobscured until it declines as a result of the depletion of the available gas reservoirs. 
As the quasar transitions through these stages in its lifetime, key observational properties, such as the luminosity and the SED, are likely to change. 
Quantitatively, however, it remains unclear how these phases relate to the fundamental properties of the accreting black-hole (e.g.  mass (M$_{\rm{BH}}$), bolometric luminosity (L$_{\rm{bol}}$) and Eddington ratio (L/L$_{\rm{Edd}}$) and the elements of the non-spherical geometry). 
However, multiple authors have found no significant dependence of the spectral energy distribution (SED) on properties such as redshift, bolometric luminosity, SMBH mass, or accretion rate (e.g. Elvis et al. 2012, Hao et al. 2013) and quasars up to redshift 7 have been shown to have similar UV spectra to low redshift quasars (e.g. Mortlock et al. 2011). 

In one model, quasars are surrounded by an inner `wall' of gas and dust, in a cylinder-like geometry. 
As a radiatively-driven outflow develops, material at the extremes of the cylinder is driven-off, exposing more of the inner edge of the obscuring material at the equator (a `torus'), which contains the hot dust. 
Quasars are also observed with a wide distribution of reddening from dust on galactic scales. 
It is believed that luminous highly dust-reddened quasars may be in the process of expelling their dust and transitioning to ultra-violet bright objects which make up to the majority of the SDSS sample. 
Outflow signatures are very common in the rest-frame optical \ha line profiles of the heavily reddened quasars (Banerji et al. 2012, 2013), but it isn't clear how significant these outflows are on larger, kiloparsec scales.
