%************************************************
\chapter{Introduction}\label{ch:introduction}
%************************************************

\section{The AGN-Host Galaxy Connection}

Super-massive black holes (BHs) are found at the centres of most nearby massive galaxies and the BH mass and mass of the host galaxy spheroid are strongly correlated \citep{ferrarese00,gebhardt00,kormendy13}. 
Although any underlying causal mechanism(s) responsible for the correlation is yet to be conclusively identified, there is considerable observational and theoretical support for models that involve BH-fuelling, outflows and a `feedback' relationship \citep[e.g.][]{king15}.  
The number density of quasars, which evolves strongly with redshift, peaks at redshifts $2 \lesssim z \lesssim 3$ \citep[e.g.][]{brandt05,richards06b} and the most massive (M$_{\rm BH} \gtrsim 10^9\msun$) present-day BHs experienced much of their growth during this epoch.  
The star formation rate, which closely follows the cosmological evolution of the quasar luminosity function, also peaks during this epoch \citep[e.g.][]{boyle98}. 
Quantifying the growth-rate of massive BHs at $2 \lesssim z \lesssim 3$ would therefore help significantly in understanding the role quasars play in galaxy evolution.

There is now considerable observational and theoretical support for models of galaxy formation that involve black hole-fuelling, outflows and a ‘feedback’ relationship between active black holes and star formation in the host galaxy. 
Super-massive black holes accreted most of their mass and galaxies formed most of their stars at redshifts $z\gtrsim2$ (e.g. Madau \& Dickinson 2014 for star formation; find quasar reference.)
During this key cosmological epoch star formation is believed to be suppressed by the energy output from the quasar, establishing the tight relationship between BH mass and host galaxy spheroid mass observed in the local Universe (e.g. Kormendy \& Ho 2013). 

\section{Measuring Black Hole Masses}

The goal of better understanding the relationship between super-massive BH accretion and star formation has led to much work focussing on the properties of quasars
and active galactic nuclei at these redshifts.
Accurate BH mass estimates for quasars are essential in these studies. 
Furthermore, as one of just two fundamental quantities describing a black hole on astrophysical
scales, the mass is of crucial importance to virtually all areas of quasar science, including the evolution and phenomenology of quasars, and accretion physics.

\subsection{Reverberation Mapping}

Reliable estimates of BH masses are a prerequisite for investigating the relationship between BHs and their host galaxies.  
If the line-emitting clouds in the broad line region (BLR) are assumed to be virialized and moving in a potential dominated by the central BH, then the BH mass is simply a product of the BLR size and the square of the virial velocity.
The reverberation-mapping technique uses the time lag between variations in the continuum emission and correlated variations in the broad line emission to measure the typical size of the BLR \citep{peterson93,peterson14}. 
The full width at half maximum (FWHM) or dispersion ($\sigma$; derived from the second moment) velocity of the prominent broad emission line of \hb (4862.7\AA)\footnote{Vacuum wavelengths are employed throughout the thesis.} is used as an indicator of the virial velocity, with extensions to other low-ionization emission lines such as \ha (6564.6\AA) and \ion{Mg}{II}\ll2796.4,2803.5 \citep[e.g.][]{vestergaard02,mclure02,wu04,kollmeier06,onken08,wang09,rafiee11}.
Extensive reverberation mapping campaigns have provided accurate BH masses for $\sim$50 active galactic nuclei (AGN) at relatively low redshifts and of modest luminosity \citep[e.g.][]{kaspi00,kaspi07,peterson04,bentz09,denney10}. 
[See galaxies talk for a few more details]

\subsection{Single-Epoch Virial Estimates}

Reverberation mapping campaigns have also revealed a tight relationship between the radius of the BLR and the quasar optical (or ultraviolet) luminosity \citep[the $R-L$ relation; e.g.][]{kaspi00,kaspi07}.
This relation provides a much less expensive method of measuring the BLR radius, and large-scale studies of AGN and quasar demographics have thus become possible through the calibration of single-epoch virial-mass estimators using the reverberation-derived BH masses \citep[e.g.][]{greene05,vestergaard06,vestergaard09,shen11,shen12,trakhtenbrot12}.
The uncertainties in reverberation mapped BH masses are estimated to be $\sim 0.4$ dex \citep[e.g.][]{peterson10}, and the uncertainties in virial masses are similar \citep[e.g.][]{vestergaard06}.
Since the structure and geometry of the BLR is unknown, a virial coefficient $f$ is introduced to transform the observed line-of-sight velocity inferred from the line width in to a virial velocity.
This simplification accounts for a significant part of the uncertainty in virial BH masses \citep[in addition to, for example, describing the BLR with a single radius $R$ and scatter in the $R-L$ relation;][]{shen13}. 
Furthermore, if the BLR is anisotropic \citep[for example, in a flattened disk; e.g.][]{jarvis06} then the line width will be orientation-dependent \citep[e.g.][]{runnoe13b,shen14,brotherton15}. 

For example, single epoch estimates have been used to calculate black hole masses in the highest redshift quasars to study the growth of SMBHs. 
This figure shows a compilation of SE mass estimates for quasars over a wide redshift range from different studies. These studies show that massive, $10^9$ BHs are probably already in place by $z\sim7$, when the age of the Universe is less than 1 Gyr.
This places strong constraints on BH growth models. 
Single epoch masses have also been used to study the distribution of quasars in the BH mass-luminosity plane, which conveys important information about the accretion process of these active black holes (e.g. Kollmeier et al. 2016). 
Reshift evolution of BH-bulge scaling relations (e.g. Bennert et al. 2011). 
Clustering (Shen \& Ho 2014; Timins et al.?). 