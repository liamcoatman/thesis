% !TEX root = ../main.tex

%************************************************
\chapter{Introduction}
\label{ch:intro}
%************************************************

\section{Discovery}

The first quasar was discovered when it was found that the star-like, thirteenth magnitude objected associated with the radio source 3C 273 was at a cosmological distance \citep[$z=0.158$;][]{schmidt63}. 
This implied an enormous luminosity ($4\times10^{12}$L$_\odot$) for such a compact object and it was quickly realised that energy source was the release of gravitational potential energy as mass is accreted onto a super-massive \marginpar{Super-massive: $10^{6 - 9}$ M$_\odot$} black hole (BH) at the centre of a galaxy \citep[e.g.][]{hoyle63,salpeter64,lynden-bell69,lynden-bell71}. 

\begin{figure}
  \centering
  \includegraphics[width=0.5\textwidth]{figures/chapter05/urry_model}
  \caption[{Illustration of the physical structure of an AGN in a simple orientation-based unification model.}]{Illustration of the physical structure of an AGN in a simple orientation-based unification model. Figure taken from \citet{urry95}.}
  \label{fig:agnmodel}
\end{figure}

\section{Basic structure}

An Active Galactic Nucleus\footnote{Throughout this thesis we use the terms `quasar' and `Active Galactic Nucleus (AGN)' interchangeably to describe active supermassive black holes, although the term quasar is generally reserved for the luminous (L$_{\rm Bol} > 10^{12}{\rm L}_{\odot}$) subset of AGNs.} (AGN) is significantly more compact than a cubic parsec, and yet can outshine the starlight from an entire galaxy. 
The basic features of the current paradigm explaining this phenomenon are essentially unchanged from \citet{salpeter64}, although many of the details remain unclear.
Hmmm... dont say this. they didnt know about accretion disks etc. 
Material is pulled towards a super-massive BH and sheds angular momentum through viscous and turbulent processes in a hot accretion disc \citep[e.g.][]{begelman85}. 
The accretion disc reaches temperatures of $\sim$$10^6$K, and radiates primarily at ultraviolet (UV) to soft-X-ray wavelengths. 
Hard X-ray emission originates in a hot corona near the BH, emission lines are produced in rapidly moving clouds of ionised gas and infrared emission is dominated by thermal emission from a dusty, parsec-scale structure. 
Collimated jets of relativistic plasma and/or associated lobes are also seen in the 10 per cent of quasars that are radio-loud \citep[e.g.][]{peterson97}. 

\subsection{The broad line region}

One of the pre-eminent features of many AGN spectra are broad optical and UV emission lines produced in the {\em broad line region (BLR)}. 
The BLR consists of gas clouds at distances from several light-days to several light-months that are photo-ionised by the ultraviolet continuum emission emanating from the accretion disc.  
Because of the close proximity to the central super-massive BH, bulk motions are dominated by gravity and radiation pressure.
The very broad emission line widths are assumed to Doppler-broadened, and imply line-of-sight velocities of many thousands of \kms. 

\subsection{The dusty torus}

Further out are dusty, molecular clouds which are co-planar with the accretion disc. 
These dusty clouds are generally referred to as the `torus'. 
In a Type II AGN, the accretion disc is observed in an edge-on configuration and, as a result, emission from the accretion and BLR is obscured by the dusty torus \citep[e.g.][]{antonucci93}.
Although this simple picture (shown in Figure~\ref{fig:agnmodel} as well as in countless other publications) is a useful staring point, the idea of a torus as a static, doughnut-like structure is almost certainly incorrect. 
For example, the problem of maintaining the large scale height required by unification schemes has long been recognized. 
In one alternative scenario, the torus is the dusty part of an accretion disc wind that extends beyond the dust-sublimation radius \citep[e.g.][]{konigl94,everett09,gallagher12,everett05,keating12,elitzur06}. 

\subsection{The narrow line region}

Further away from the central BH and beyond the dusty torus is the narrow emission line region (NLR). 
Like the BLR, the NLR is ionised by radiation from the central source. 
Unlike the BLR, densities in the NLR are low enough that forbidden transitions are not collisionally suppressed. 
Emission line widths are typically hundreds of \kms in the NLR. 
The NLR is sufficiently extended to be spatially resolved. 
\todo{Extent: ask Paul/Manda}

\section{Winds and outflows in AGN}

Quasars are very powerful sources of radiation, and are embedded in matter-rich environments at the centres of galaxies.
Strong winds, driven by some combination of gas pressure, radiation pressure due to dust or lines, and magnetic forces, are to be expected under these conditions \citep[e.g.][]{blandford82b,proga00,everett05}. 
In line with these expectations, evidence for outflows is common in the spectra of quasars. 

Perhaps the most dramatic evidence of outflows is seen in broad absorption line quasars \citep[BALQSOs;][]{weymann91}.
BALQSOs are characterised by broad absorption features in the ultra-violet resonance lines of highly ionised \ion{N}{V}, \ion{C}{IV} and \ion{Si}{IV}. 
The absorption is always blueshifted, and is evidence for fast outflows with velocities as large as 60\,000 \kms \citep[e.g.][]{turnshek88}. 
The observed \ion{C}{IV} BALQSO fraction in radio-quiet quasars is $\sim15$ per cent \citep[e.g.][]{hewett03,reichard03} and the intrinsic fraction has been estimated at $40$ per cent \citep{allen11}.
The blueshifting of high-ionisation lines in the BLR (including \ion{C}{IV}) also appears to be nearly ubiquitous in the quasar population \citep[e.g.][]{richards02,richards11}, suggesting winds are even more common.
Outflows are also used to explain narrow UV and X-ray absorption lines (NALs) which are seen in $\sim60$ per cent of Seyfert 1 galaxies \citep{crenshaw99} and some quasars \citep[e.g.][]{hamann97}. 
The wide range of emission and absorption line phenomena can be explained in disc wind models \citep[e.g.][]{murray95,elvis00,proga00,everett05}

\section{SDSS and the era of survey astronomy}

\todoinline{Emphasise the range of data now available and this allows us to do}

\section{The AGN-host galaxy connection}

The space density of quasars was much greater at $z\gtrsim2$, and declines steeply to $z=0$. 
The existence of inactive BHs at the centres of massive galaxies is therefore a fundamental test of the quasar paradigm. 
Significant resources have been devoted to searching for these BHs, which are now known to exist in the centres of many nearby massive galaxies \citep[e.g.][]{kormendy95,ferrarese05,kormendy13}.
Remarkably, given the sphere-of-influence of the BH is many orders of magnitude smaller than the size of the galaxy, the BH mass and mass of the host galaxy spheroid are strongly correlated \citep{ferrarese00,gebhardt00,graham01,tremaine02,marconi03,aller07,gultekin09}.  
Although any underlying causal mechanism(s) responsible for the correlation is yet to be conclusively identified, there is considerable observational and theoretical support for a `feedback' relationship in which the energy output from rapidly accreting BHs (in a quasar phase) couples with the gas in the host galaxy and quenches star formation \citep[e.g.][]{silk98,king03,dimatteo05,king15}. 

Models of galaxy evolution that invoke AGN feedback require these outflows to reach galactic scales and quench star formation in the AGN host galaxies. 
In recent years, a huge amount of resources have been devoted to searching for observational evidence of these galaxy-wide, AGN-driven outflows. 
This has resulted in recent detections of outflows in AGN-host galaxies using tracers of atomic, molecular, and ionised gas with enough power to sweep their host galaxies clear of gas \citep[e.g.][]{nesvadba06,arav08,nesvadba08,moe09,dunn10,alexander10,harrison12,harrison14,nesvadba10,rupke13,veilleux13,nardini15,feruglio10,alatalo11,cimatti13,cicone14}.  

Quasar feedback has also been invoked to explain the similarity of the cosmic BH accretion and star formation histories.
The number density of quasars, which evolves strongly with redshift, peaks at redshifts $2 \lesssim z \lesssim 3$ \citep[e.g.][]{brandt05,richards06b} and the most massive (M$_\bullet \gtrsim 10^9\msun$) present-day BHs experienced much of their growth during this epoch.  
The star formation rate, which closely follows the cosmological evolution of the quasar luminosity function, also peaks during this epoch \citep[e.g.][]{boyle98}. 
Quantifying the growth-rate of massive BHs at $2 \lesssim z \lesssim 3$ would therefore help significantly in understanding the role quasars play in galaxy evolution.

\section{Measuring black hole masses}

As one of just two fundamental quantities describing a BH on astrophysical scales, the mass is of crucial importance to virtually all areas of quasar science, including the evolution and phenomenology of quasars, and accretion physics.
The power output of quasars is directly proportional to the BH mass. 
There is much debate regarding what effect the energy output by quasars has on the evolution and structure of the host galaxy. 

The masses of BHs in many local, inactive galaxies have been measured by dynamical modelling spatially resolved kinematics. 
However, this requires the sphere-of-influence of the BH, $R_{\rm BH}$ \marginpar{$R_{\rm BH} = \frac{2GM_{\rm BH}}{\sigma\ast}$}, to be resolved. 
With BH masses only $\sim$0.1 per cent of the stellar mass of the host galaxies, $R_{\rm BH}\sim1-100$ pc.
With current instrumentation, resolving this region is only possible in very close by galaxies. 

The reverberation mapping method, first proposed by \citet{blandford82a}, uses the time delay between continuum variations and emission-line variations to estimate the size of the BLR, and hence the BH mass. 
Because it depends on temporal resolution rather than spatial resolution, this technique can be applied out to much greater distances. 

\subsection{Reverberation mapping}

Continuum variability is a common characteristic of quasars, owing to the stochastic nature of the accretion process.  
Because the BLR is photo-ionized by the continuum, the broad emission lines also vary with some characteristic lag, which is related to the light travel time across the BLR. 
The reverberation mapping technique uses the time lag between variations in the continuum emission and correlated variations in the broad line emission to measure the typical size of the BLR \citep[e.g.][]{peterson93,netzer97,peterson14}. 

Under the assumptions that the BLR dynamics are virialised \marginpar{The virial theorem states...} and the gravitational potential is dominated by the BH, the BH mass is given by:

\begin{equation}
M_{\rm BH} = f\left( \frac{\Delta V^2R}{G} \right)
\end{equation}

where $\Delta V$ is the line-width and $R$ is the reverberation BLR radius.  
In practice, reverberation mapping relies on dense spectrophotometric monitoring campaigns which span many years. 
The typical velocity in the BLR is measured from the width of the broad \hb line.
Since the structure and geometry of the BLR is unknown, a virial coefficient $f$ is introduced to transform the observed line-of-sight velocity inferred from the line width in to a virial velocity. 
In practice, the value of $f$ is empirically determined by requiring that the derived masses are consistent with those predicted from the M-$\sigma$ \marginpar{$\sigma$: velocity dispersion of galaxy} relation for local inactive galaxies. 
Although the reverberation mapping technique has proved to be effective, because it relies on resource-intensive spectro-photometric monitoring campaigns, lags have been measured for only $\sim50$ AGN \citep[e.g.][]{kaspi00,peterson04,kaspi07,bentz09,denney10,barth11,grier12}. 
This sample is strongly biased to low luminosity Seyfert 1 galaxies \marginpar{Seyfert 1:}, and the maximum redshift is just $z\sim0.3$. 

The full width at half maximum (FWHM) or dispersion ($\sigma$; derived from the second moment) velocity of the prominent broad emission line of \hb (4862.7\AA)\footnote{Vacuum wavelengths are employed throughout the thesis.} 
is used as an indicator of the virial velocity, with extensions to other low-ionization emission lines such as \ha (6564.6\AA) and \ion{Mg}{II}\ll2796.4,2803.5 \citep[e.g.][]{vestergaard02,mclure02,wu04,kollmeier06,onken08,wang09,rafiee11}.
\marginpar{FWHM: Full width of the line profile at half of maximum intensity}
 

\subsection{Single-epoch virial estimates}

Reverberation mapping campaigns have also revealed a tight relationship between the radius of the BLR and the quasar optical (or ultraviolet) luminosity \citep[the $R-L$ relation; e.g.][]{kaspi00,kaspi07}.
A slope of $\simeq0.5$ is found, which consistent with the naive prediction \citep[e.g.][]{peterson97}. 
An advantage of the technique is that it is inexpensive in telescope time. 
A single spectrum yields a mass measurement. 
This relation provides a much less expensive method of measuring the BLR radius, and large-scale studies of AGN and quasar demographics have thus become possible through the calibration of single-epoch virial-mass estimators using the reverberation-derived BH masses \citep[e.g.][]{greene05b,vestergaard06,vestergaard09,shen11,shen12,trakhtenbrot12}.
Single-epoch virial BH mass estimates normally take the form

\begin{equation}
  \label{eq:virialmass}
  \mathrm{M_{BH}} = 10^{a} \left( \frac{\Delta V}{1000~\mathrm{km~s^{-1}}} \right)^b \left[ \frac{L_{\lambda}}{10^{44}~\mathrm{erg~s^{-1}}} \right]^c
\end{equation}

\noindent where $\Delta V$ is a measure of the line width (from either the FWHM or dispersion), $L_\lambda$ is the monochromatic continuum luminosity at wavelength $\lambda$, and $a$, $b$, and $c$ are coefficients, determined via calibration against a sample of AGN with reverberation-mapping BH mass estimates. Several calibrations have been derived using different lines (e.g. \hbns, \ion{Mg}{II}, \ion{C}{IV}) and different measures of the line width (FWHM or dispersion) \citep[e.g.][]{vestergaard02,mclure02,vestergaard06,mcgill08,wang09,rafiee11,park13}.

The uncertainties in reverberation mapped BH masses are estimated to be $\sim 0.4$ dex \citep[e.g.][]{peterson10}, and the uncertainties in virial masses are similar \citep[e.g.][]{vestergaard06}.
Since the structure and geometry of the BLR is unknown, a virial coefficient $f$ is introduced to transform the observed line-of-sight velocity inferred from the line width in to a virial velocity.
This simplification accounts for a significant part of the uncertainty in virial BH masses \citep[in addition to, for example, describing the BLR with a single radius $R$ and scatter in the $R-L$ relation;][]{shen13}. 
By far the biggest uncertainty is the virial coefficient f . 
It is unknown, and it probably varies from source to source.
A spherical distribution of clouds on random, isotropic orbits has f = 3/4 for $\Delta$V = FWHM and
f = 3 for $\Delta$V = $\sigma$ (Netzer 1990).
Furthermore, if the BLR is anisotropic \citep[for example, in a flattened disk; e.g.][]{jarvis06} then the line width will be orientation-dependent \citep[e.g.][]{runnoe13b,shen14,brotherton15}. 

The main progress in this area in recent years, that enables comprehensive statistical studies of active black holes (BHs), is the success of the large reverberation mapping project. 
This allows reliable estimates of broad line region (BLR) sizes and BH masses. 
The main concern and the biggest unknown is the extension of the method to high redshifts where \hb measurements are no longer available. 
Something we will explore in Chapter~\ref{ch:bhmass}. 

For example, single epoch estimates have been used to calculate black hole masses in the highest redshift quasars to study the growth of SMBHs. 
This figure shows a compilation of SE mass estimates for quasars over a wide redshift range from different studies. 
These studies show that massive, $10^9$ BHs are probably already in place by $z\sim7$, when the age of the Universe is less than 1 Gyr.
The fact that a SMBH exists in a quasar at such high redshift is of great importance in physics.
The high redshift means that it was already there when our universe was very young, only about
800 million years old. And the fact that a SMBH was able to grow up in such a short time put
some very tight constraints upon both the cosmological parameters and the accretion history of the
SMBH itself (Willott et al. 2003).

Single epoch masses have also been used to study the distribution of quasars in the BH mass-luminosity plane, which conveys important information about the accretion process of these active black holes (e.g. Kollmeier et al. 2016). 
Redshift evolution of BH-bulge scaling relations (e.g. Bennert et al. 2011). 
Clustering (Shen \& Ho 2014; Timins et al.?). 
With the R–L relationship, we are able to explore the black hole mass function, not only locally but at high redshift, enabling us to trace the history of black hole growth.
Some exploratory work has been done on this and in fact there are claims that the M-sigma relation evolves over time. 
Estimates of such masses are important with respect to the relation between the MBH in the center of a stellar spheroid and the velocity dispersion. 

We emphasize that application of single-epoch spectroscopy to quasars rests on the untested assumption that machinery which is calibrated for sub-Eddington BHs with M$\sim10^7$ still works for BHs with masses up to $10^{10}$ that radiate near the Eddington limit. 
Refer forward to problems with \ion{C}{IV} (Chapter 3)

\section{SEDs}

AGN emit strongly over many decades in frequency. 
To first order SEDs are remarkably similar over many decades in luminosity and redshift. 
Significant diversity is observed in the SEDs of individual objects. 
However, the systematic study of the dependence of the SED shape on physical parameters has, until very recently, been limited by the difficulty in obtaining a large sample of quasars with good multi-wavelength coverage and large dynamic range in luminosity and redshift. 
However, we are able to take advantage of a number of recent, sensitive, wide-field photometric surveys, including SDSS (in the UV/optical), UKIDSS (in the NIR) and WISE (in the mid-infrared).

\section{Summary / what I need to get across}

It's a data rich time.
SDSS has been revolutionary - shown the power of large surveys. 
We have wide-field photometry in a number of bands - important because AGN emit strongly over many decades in frequency. 
With spectra from SDSS we can derive BH masses and outflow properties from optical lines. 
But these are shifted to infrared wavelengths at redshifts > 1, when things get interesting. 
Increasing availability of infrared-spectra. 
Looking to the future, huge spectroscopic surveys - WEAVE, 4MOST. 

Quasar black hole masses: \citet{shen13}, \citet{peterson10}, \citet{peterson11}, \citet{vestergaard11}, \citet{marziani12}. 
This has motivated a considerable amount of observational work searching for feedback signatures \citep[for recent reviews, see][]{alexander12,fabian12,heckman14}. 

Throughout this thesis we adopt a $\Lambda$CDM cosmology with $h_0=0.71$, $\Omega_M=0.27$, and $\Omega_\Lambda=0.73$. 
All wavelengths and equivalent width measurements are given in the quasar rest-frame, and all emission line wavelengths are given as measured in vacuum.

Get across: 

Quasars are not all the same!
There is a large range of continuum, emission, and absorption properties among quasars, which demands that quasars cannot be fully described by a single, static picture. 







\section{Overview of thesis}

Describe thesis in one chapter. 

\subsection{A near-infrared spectroscopic database of high-redshift quasars}

\subsection{Black Hole Masses}

Black-hole masses are crucial to understanding the physics of the connection between quasars and their host galaxies and measuring cosmic black hole-growth. 
At high redshift, $z \gtrsim 2.1$, black hole masses are normally derived using the velocity-width of the \ion{C}{IV}\ll1548,1550 broad emission line, based on the assumption that the observed velocity-widths arise from virial-induced motions.  
In many quasars, the \ion{C}{IV}-emission line exhibits significant blue asymmetries (`blueshifts') with the line centroid displaced by up to thousands of \kms\, to the blue. 
These blueshifts almost certainly signal the presence of strong outflows, most likely originating in a disc wind.
We have obtained near-infrared spectra, including the \ha\l6565 emission line, for 19 luminous ($L_{\rm Bol} = 46.5-47.5$ erg~s$^{-1}$) Sloan Digital Sky Survey quasars, at redshifts $2 < z < 2.7$, with \ion{C}{IV} emission lines spanning the full-range of blueshifts present in the population.  
A strong correlation between \ion{C}{IV}-velocity width and blueshift is found and, at large blueshifts, $>$2000\,\kms, the velocity-widths appear to be dominated by non-virial motions. 
Black-hole masses, based on the full width at half maximum of the \ion{C}{IV}-emission line, can be overestimated by a factor of five at large blueshifts. 
A larger sample of quasar spectra with both \ion{C}{IV} and \hbns, or \hans, emission lines will allow quantitative corrections to \ion{C}{IV}-based black-hole masses as a function of blueshift to be derived. 
We find that quasars with large \ion{C}{IV} blueshifts possess high Eddington luminosity ratios and that the fraction of high-blueshift quasars in a flux-limited sample is enhanced by a factor of approximately four relative to a sample limited by black hole mass.    

The \ion{C}{IV}$\lambda\lambda$1498,1501 broad emission line is visible in optical spectra to redshifts exceeding $z\sim5$. 
\ion{C}{IV} has long been known to exhibit significant displacements to the blue and these `blueshifts' almost certainly signal the presence of strong outflows.
As a consequence, single-epoch virial black hole (BH) mass estimates derived from \ion{C}{IV} velocity-widths are known to be systematically biased compared to masses from the hydrogen Balmer lines. 
Using a large sample of 230 high-luminosity ($L_{\rm Bol} = 10^{45.5}-10^{48}$ erg s$^{-1}$), redshift $1.5 < z < 4.0$ quasars with both \ion{C}{IV} and Balmer line spectra, we have quantified the bias in \ion{C}{IV} BH masses as a function of the \ion{C}{IV} blueshift. 
\ion{C}{IV} BH masses are shown to be a factor of five larger than the corresponding Balmer-line masses at \ion{C}{IV} blueshifts of 3000\kms and are over-estimated by almost an order of magnitude at the most extreme blueshifts, $\gtrsim 5000$\kms.
Using the monotonically increasing relationship between the \ion{C}{IV} blueshift and the mass ratio BH(\ion{C}{IV})/BH(\hans) we derive an empirical correction to all \ion{C}{IV} BH-masses.
The scatter between the corrected \ion{C}{IV} masses and the Balmer masses is 0.24 dex at low \ion{C}{IV} blueshifts ($\sim$0\kms) and just 0.10 dex at high blueshifts ($\sim$3000\kms), compared to 0.40 dex before the correction. 
The correction depends only on the \ion{C}{IV} line properties - i.e. full-width at half maximum and blueshift - and can therefore be applied to all quasars where \ion{C}{IV} emission line properties have been measured, enabling the derivation of un-biased virial BH mass estimates for the majority of high-luminosity, high-redshift, spectroscopically confirmed quasars in the literature.

\subsection{Narrow line region properties}

\subsection{SED Properties}








New constituent to the universe
2nd highest redshift
redshifts placed them at the very ede of the universe 
discovered almost 60 years ago
in recent years discvovered that quasars play a crucial role in the evolution of galaxies. can think of quasars as new insight into the evolution of galaxies and interaction with environment. 
should mentino sefert 1s - discovered when and found in local universe. enormous gap in luminosity so it took some to for link with quasars to be accepted. 
discovered that 3c273 variable on timescale of weeks 
proved to be agn only when host galaxies were unambiguously detected in 1980s. 
breakthrough came at the end of the 70s, when Rees proposed an astrophysical context for the MBH model with his flow chart (Rees 1978)
should mention boroson and green - not all quasars same and this is trying to understand why 
disk wind model - high ionisation lines come from outflowing clouds, but the accretion disk is blocking our view of the clouds receding on the far side
Low ionisatin lines arise in a different region, perhaps further out, a. 
In 70s it was thought that quasars were weird galaxies that for some resason had black holes at heir ecentres. 
It's now known that every galaxy has gone through an active phase and so should have a black hole. 
Kormendy in 1993 discovered that black hole masses and bulge masses were correlated. 
need to be clear its the short time scale variations that imply the small size
Seyfert galaxies - 1943 Cark Seyfert
seyfert < 10^45. quasras 1046-48

initially quasars were identified as radio sources
after it was realised that most quasars were radio quiet , led to UV excess surveys such as Bright Quasar Survey also known as Palomar-Green survey
Advent of larger CCDs led to development of tecniques sucas as that used by SDSS, in which quasar caniates lie significantl off stellar locus in multicolor space. 
Finally relasied that there are a sbstantial fraction that donot show optical porperties that distinchish them from inactive nucleibecause of obscuration. 
These are found through their mir-IR emission or X-ray. 
SDSS was imaging survey in five bands of 8400 square degrees, together with spectra of a million galaxies, 12000 quasarss and 22500 stars, 

Boroson and green among first to analyse quasar spectroscopic properties in a systematic way.
black hole mass is clearly one of the most important physical parameters of a quasar. 

theory of accretion disc winds has long histroy, but really gained holdin the late 1990s, with a sueries of publications my Murray and collaborators (see refs p128). There is partially inoised material around accretion disk, including tripally ionsied carbon. UV photons from acretion disk excite electrons in the atoms, and as areuslt will exert a pressure on the atom itself - radiation line driving.  result is that a wind can be blown from the accretion disk. process is very efficient as long as atoms are not complelet inoised. if this happens there are no electons left in the atoms to produce ratioation pressure from spectral lines. This, the wind is senstive to the ratio of the UV to X-ray continuum. 

In the mid-80's it was determined from CCD imaging of some low redshift QSOs that the QSOs were located in the centers of galaxies and hence are just a more luminous form of Active Galactic Nuclei like Seyfert Galaxies .

M-sigma seems to argue that black holes konew what is going on in the bulge adnd/or the bulge knows about the black hole. 
That is, there is a causal connection between gravity o very small scales with gravity on much large scale. 
In order for MBH and Bbulge to grown in sync, suggested that energetic feedback from the black hole regulates the rate and duration of both star frmation in the bulge and the growth of the black hole itself (31, 38, 70)
Origin of the relationship still isn't clear, but how this relation changes with rdshift probably tells us something about how black holes and galaxie form and evolve over cosmic times. 

sphere of influence - sphere within which the gravity of the black hole will dominate the dynamics of the stars to a greater etent than the other mass components (stars, gas)

Interest in quasars has exploded with the discovery of the M-sigma relation. 

Within a decade of their discovery, the standard model for quasars was in place  - supermassive black hole, accretion disc, and jet. 

Accretion disc winds provide a unifying model for all emission and absorption features - e.g. elvis 2000

Beyond some radius, this accretion disc is cool enoughfor dust accreitn ito it from the host interstellar mdedum not to be destroyed. if this dusty gas were ejected from the disk, it would be efficiently accelerated by the central continuum source because of the high cross section of dust. This would create a dusty wind from the disk. 
(Elitzur \& Ivezic 2001)

The central location of quasars in their host galaxies and the fact that they can produce a large amount of energy imply that quasars can play a dominant role in determining the physical conditions of matter, not liny in the quasar vicinity but also on galactic and even intergalactic scales (e.g. 13, 31, 67, ..) 
BALs etc indiate that quasar continuum radiation affects the quasars immediate environment.  In addition, the tight correlation between the mass M of the central H in a galactic nucelus and the velocity dispersion sigma of the galaxys bulge or spheroid, can be explained by feedback between quasares and the infalling material from large distances.  The feedback can quench both BH accretion and star formation in te galaxy when the BH reaches a certain mass. Quasars could provide such feedback precisely because they are very powerful sources of energy and momentum (references given). At present, the physical processes involved are not well understood. We don't know if quasar disk winds are strong and persistent enoujgh to deliver energy over large radii. 

In recent years it has become clear that BHs are ubiquitous in nearby galaxies. 
Mass correlated with properties of the surrounding galaxy bulge. 
The cosmic history of BH growth also appears to trace cosmic star formatoin rate, establishing anoter connection between BH and galaxy properties. 
These discoveries have changed our view of AGNs.
Accretion now appears to be a common phase in the evolution of normal galaxies. 
This has led to a renewed interest in AGN properties, especially regarding the host galaxies. 

When and how these relations set in and what are the physical processes respnobsible are stil  open questions. 
Probing the BH-host galaxy at high redshift is challenging. 
The only was to measure MBH is is to use the virial relation. 

The presence of massive BHs in the center of most galaxies (precuted by Soltan many yeaars before and promoted by Rees) changed our vision of galaxy evolution. 

Until the mid-1980s, quasars involved a few hundred distant peculiar sources.
Since then, the number of quasars has grown exponentially.

More to the story than unification which paints a static picture and ignores the diversity among unobscured quasars. e.g. changing look quasars. Evolution not included in unifcation scheme. 


If wuasar activity is a stage in the life of massive galaxies (lasting perhaps 10^8 years) then most galaxies today should harbour an extingusihed black hole, as propsed inthe late 1960s (Lynden Bell 1969). The quest for exhausted quasars became serious in the 1980s. In the early 1990s HST observations resolved ga motions in the nucelous of M 87 where a black hole was found with mass similar to quasars at z=2. 

Observational evidence supports a role for gravitation (virial broadening of low-ionisiation lines) and radiation pressure (driving the outflow in high-ionisation part) Eddington ratio controls balance of gravitationa ad radiative forces and could determine strcutre of BLR (pop A/B)

The radiaiet energy released may have a dramatic effect on surroundings. Little doubt about this. But can mechanical and radiative output effect the entire bulge? This question has arisin in the last 10-15 years. 

Recent claims that BH might drive large scale molecular outflows. 

M-sigma was unknown for the first 25 years of quasar research. The importance inolves the expectation of coeval evolution for bulges and SMBHs through feedback. (footnote: although there are claims that M-sigma is a selection effect. Need to resolve the sphere of influence and maybe only most massive have been measured.)

Mention PCA/ICA stuff is quest of HR diagram for quasars - clearly very important. 