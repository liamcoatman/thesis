%*******************************************************
% Abstract
%*******************************************************
%\renewcommand{\abstractname}{Abstract}
\pdfbookmark[1]{Abstract}{Abstract}
\begingroup
\let\clearpage\relax
\let\cleardoublepage\relax
\let\cleardoublepage\relax

\chapter*{Abstract}

The near-infrared band provides a wealth of information on the properties of quasars and their environments at redshifts $2 \lesssim z \lesssim 4$. 
Quasar activity and cosmic star formation both peak during this epoch, and powerful quasar-driven outflows are thought to quench star formation and shut down black hole (BH) accretion. 
We have constructed a large near-infrared spectroscopic catalogue of high redshift quasars to investigate quasar-driven outflows and their relation to fundamental properties of quasars. 

At high redshift, $z \gtrsim 2$, quasar BH masses are derived using the velocity-width of the \ion{C}{IV} broad emission line, based on the assumption that the observed velocity-widths arise from virial-induced motions.  
\ion{C}{IV} has long been known to exhibit significant displacements to the blue and these `blueshifts' almost certainly signal the presence of strong outflows.
As a consequence, single-epoch virial BH mass estimates derived from \ion{C}{IV} velocity-widths are known to be systematically biased compared to masses from the hydrogen Balmer lines.
By combining our near-infrared spectroscopic catalogue (covering the Balmer lines) with optical spectroscopy from SDSS (covering \ion{C}{IV}), we have quantified the bias in \ion{C}{IV} BH masses as a function of the \ion{C}{IV} blueshift. 
\ion{C}{IV} BH masses are shown to be over-estimated by almost an order of magnitude at the most extreme blueshifts.
Using the monotonically increasing relationship between the \ion{C}{IV} blueshift and the mass ratio BH(\ion{C}{IV})/BH(\hans) we derive an empirical correction to all \ion{C}{IV} BH-masses.
The correction depends only on the \ion{C}{IV} line properties and therefore enables the derivation of un-biased virial BH mass estimates for the majority of high-luminosity, high-redshift, spectroscopically confirmed quasars in the literature. 

[\ion{O}{III}] emission traces ionised gas extended over kiloparsec scales in the quasar host galaxies. 
The prevalence of high-velocity blueshifted emission suggests that outflows are very common in luminous quasars at redshifts $z\sim2$. 
There is a strong anti-correlation between the [\ion{O}{III}] EQW and the \ion{C}{IV} blueshift, suggesting that quasar-driven winds are capable of sweeping away gas extended over kilo-parsec scales in the host galaxies. The [\ion{O}{III}] blueshift is correlated with the \ion{C}{IV} blueshift, which could indicate a connection between gas kinematics on sub-parsec and kilo-parsec scales. 
We estimate the kinetic power of the outflows traced by [\ion{O}{III}] to be up to a few percent of the quasar bolometric luminosity, which is similar to the efficiencies required in recent quasar-feedback models. 

Using data from a number of recent wide-field photometric surveys, we build a parametric spectral energy distribution model that is able to reproduce the median optical-infrared colours of tens of thousands of SDSS AGN at redshifts $1 < z < 3$. 
In individual objects, we find significant variation in the near-infrared region dominated by emission from hot dust.   
We find that the hot dust abundance is strongly correlated with the strength of outflows in the quasar BLR and consider the implications of this result in the context of accretion disc wind models.


\vfill

\endgroup			

\vfill