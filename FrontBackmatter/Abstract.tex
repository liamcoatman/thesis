%*******************************************************
% Abstract
%*******************************************************
%\renewcommand{\abstractname}{Abstract}
\pdfbookmark[1]{Abstract}{Abstract}
\begingroup
\let\clearpage\relax
\let\cleardoublepage\relax
\let\cleardoublepage\relax

\chapter*{Abstract}

Supermassive black holes (BHs) and their host-galaxies are thought to evolve in tandem, with the energy output from the rapidly-accreting BH regulating star formation and the growth of the BH itself. 
The goal of better understanding this process has led to much work focussing on the properties of quasars and active galactic nuclei (AGN) at relatively high redshifts, $z\gtrsim 2$, when cosmic star formation and BH accretion both peaked. 
At these redshifts, however, ground-based statistical studies of the quasar population generally have no access to the rest-frame optical spectral region, which is needed to measure \hbns-based BH masses and narrow line region outflow properties. 
The cornerstone of this thesis has been a new near-infrared spectroscopic catalogue providing rest-frame optical data on $434$ luminous quasars at redshifts $1.5 \lesssim z \lesssim 4$.

At high redshift, $z \gtrsim 2$, quasar BH masses are derived using the velocity-width of the \ion{C}{IV} broad emission-line, based on the assumption that the observed velocity-widths arise from virial-induced motions.  
However, \ion{C}{IV} exhibits significant asymmetric structure which suggests that the associated gas is not tracing virial motions. 
By combining near-infrared spectroscopic data (covering the hydrogen Balmer lines) with optical spectroscopy from the Sloan Digital Sky Survey (covering \ion{C}{IV}), we have quantified the bias in \ion{C}{IV} BH masses as a function of the \ion{C}{IV} blueshift. 
\ion{C}{IV} BH masses are shown to be over-estimated by almost an order of magnitude at the most extreme blueshifts.
Using the monotonically increasing relationship between the \ion{C}{IV} blueshift and the mass ratio BH(\ion{C}{IV})/BH(\hans) we derive an empirical correction to all \ion{C}{IV} BH-masses.
The correction depends only on the \ion{C}{IV} line properties and therefore enables the derivation of un-biased virial BH mass estimates for the majority of high-luminosity, high-redshift, spectroscopically confirmed quasars in the literature. 

Quasars driving powerful outflows over galactic scales is a central tenet of galaxy evolution models involving `quasar feedback' and significant resources have been devoted to searching for observational evidence of this phenomenon.  
We have used [\ion{O}{III}] emission to probe ionised gas extended over kilo-parsec scales in luminous $z\gtrsim2$ quasars.
Broad [\ion{O}{III}] velocity-widths and asymmetric structure indicate that strong outflows are prevalent in this population.  
We estimate the kinetic power of the outflows to be up to a few percent of the quasar bolometric luminosity, which is similar to the efficiencies required in recent quasar-feedback models. 
[\ion{O}{III}] emission is very weak in quasars with large \ion{C}{IV} blueshifts, suggesting that quasar-driven winds are capable of sweeping away gas extended over kilo-parsec scales in the host galaxies. 

Using data from a number of recent wide-field photometric surveys, we have built a parametric spectral energy distribution model that is able to reproduce the median optical to infrared colours of tens of thousands of AGN at redshifts $1 < z < 3$. 
In individual objects, we find significant variation in the near-infrared spectral energy distribution, which is dominated by emission from hot dust. 
We find that the hot dust abundance is strongly correlated with the strength of outflows in the quasar broad line region, suggesting that the hot dust may be in a wind emerging from the outer edges of the accretion disc. 

\vfill

\endgroup			

\vfill