\documentclass[a4paper,11pt]{article}

\usepackage[utf8]{inputenc}
\usepackage{amssymb,amsmath}
\usepackage{wrapfig}
\usepackage{graphicx,float}
\usepackage{caption}
\usepackage[margin=1in, includehead, headheight=15pt]{geometry}
\usepackage{booktabs}
\usepackage{fancyhdr}
\usepackage{enumitem}
\usepackage{multicol}
\usepackage{natbib}
\usepackage{paralist}
\usepackage[greek,english]{babel}
\usepackage[font={small}]{caption}
\usepackage[small]{titlesec} % smaller section titles 

\newcommand\ion[2]{\text{#1\,\textsc{\lowercase{#2}}}}
\newcommand{\fixbib}[1]{}

\newcommand{\kms}{\,km\,s$^{-1}$} % kilometres per second
\newcommand{\ergs}{\,erg\,s$^{-1}$} % ergs per second

\newcommand{\ha}{H$\textrm{\greektext a}$~}
\newcommand{\hb}{H$\textrm{\greektext b}$~}
\newcommand{\hans}{H$\textrm{\greektext a}$}
\newcommand{\hbns}{H$\textrm{\greektext b}$}

\def \ll {$\lambda\lambda$}
\def \l {$\lambda$}

\makeatletter
\newcommand{\mybullet}{%
    \ifnum\value{enumi}=1
    \else
        \textbullet
    \fi
}
\makeatother


\renewenvironment{thebibliography}[1]{\let\par\relax%
  \section*{\refname}\inparaenum[\mybullet]}{\endinparaenum}
\let\oldbibitem\bibitem
\renewcommand{\bibitem}{\item \oldbibitem}

\include{JournalAbbr} 

\newenvironment{Figure}
  {\par\medskip\noindent\minipage{\linewidth}}
  {\endminipage\par\medskip}

\renewcommand{\arraystretch}{1.2}

\pagestyle{empty} 

\pagestyle{fancy}
\fancyhead{}
\lhead{Liam Coatman}
\rhead{Quasar-driven outflows}

\begin{document}
\pagenumbering{gobble}
%\maketitle
\vspace*{-3cm}
\thispagestyle{plain}
\begin{center}
\textbf{\Large{A near-infrared view of luminous quasars: black hole masses, outflows and hot dust}}\\
\vspace{16pt}
\textbf{\large{Liam Coatman - Summary of thesis}}
\vspace{6pt}
\end{center}

Emission-lines provide a wealth of information on the properties of active galactic nuclei (AGN) and quasars and their environments. 
The rest-frame optical region in particular includes a number of strong emission features, including the Balmer lines and the [\ion{O}{III}] doublet, that are used to measure black hole (BH) masses, accretion rates, systemic redshifts and outflow properties. 
At $z\sim2$, rest-frame optical lines are redshifted to near-infrared wavelengths, and so near-infrared spectroscopy is essential for a complete understanding of quasars during the peak epoch of galaxy formation ($2 \lesssim z \lesssim 4$). 
In Chapter~2, we describe the construction of a near-infrared spectroscopic catalogue containing $462$ redshift $1.5 < z < 4$ quasars. 
This is the largest sample of its kind, and has facilitated the investigations described in Chapters~3 and 4.   

At high redshift, $z \gtrsim 2$, quasar BH masses are normally derived using the velocity-width of the \ion{C}{IV} broad emission line, based on the assumption that the observed velocity-widths arise from virial-induced motions.  
\ion{C}{IV} has long been known to exhibit significant displacements to the blue and these `blueshifts' almost certainly signal the presence of strong outflows.
As a consequence, single-epoch virial BH mass estimates derived from \ion{C}{IV} velocity-widths are known to be systematically biased compared to masses from the hydrogen Balmer lines.
By combining our near-infrared spectroscopic catalogue (covering the Balmer lines) with optical spectroscopy from SDSS (covering \ion{C}{IV}), we have quantified the bias in \ion{C}{IV} BH masses as a function of the \ion{C}{IV} blueshift. 
\ion{C}{IV} BH masses are shown to be over-estimated by almost an order of magnitude at the most extreme blueshifts.
Using the monotonically increasing relationship between the \ion{C}{IV} blueshift and the mass ratio BH(\ion{C}{IV})/BH(\hans) we derive an empirical correction to all \ion{C}{IV} BH-masses.
The correction depends only on the \ion{C}{IV} line properties and therefore enables the derivation of un-biased virial BH mass estimates for the majority of high-luminosity, high-redshift, spectroscopically confirmed quasars in the literature. 

In Chapter~4, we use our near-infrared spectroscopic catalogue to analyse the [\ion{O}{III}] emission properties of luminous quasars. 
Compared to $z \lesssim 1$ SDSS AGN, we find [\ion{O}{III}] to be significantly broader and have stronger blue-asymmetries, suggesting that AGN efficiency in driving galaxy-wide outflows increases with luminosity. 
We find the [\ion{O}{III}] equivalent width to be strongly anti-correlated with the \ion{C}{IV} blueshift, suggesting that the BLR outflows are breaking out into the interstellar medium of the host-galaxies and sweeping away the NLR gas. 
The blueshifting of [\ion{O}{III}] and \ion{C}{IV} are correlated, indicating a possible connection between gas kinematics on parsec- and kilo-parsec scales. 
We estimate the kinetic power of the outflows traced by [\ion{O}{III}] to be up to a few percent of the quasar bolometric luminosity, which is similar to the efficiencies required in quasar-feedback models. 

AGN emit strongly over many decades of the electromagnetic spectrum, and so a complete understanding of their properties critically depends on the availability of multi-wavelength data spanning the full spectral energy distribution (SED). 
Progress has been made in recent years using data from new sensitive, wide-field photometric surveys, including SDSS (ultra-violet/optical), UKIDSS (near-infrared) and WISE (mid-infrared). 
In Chapter~5, we build a simple parametric SED model that is able to reproduce the median optical-infrared colours of tens of thousands of SDSS AGN at redshifts $1 < z < 3$. 
In individual objects, we find significant variation in the near-infrared region of the SED which is dominated by emission from hot dust.   
We find that the hot dust abundance is strongly correlated with the strength of outflows in the quasar BLR (parametrised using the \ion{C}{IV} blueshift) and consider the implications of this result in the context of accretion disc wind models.
We demonstrate that apparent correlations between the hot dust abundance and BH mass result from systematic biases in the \ion{C}{IV}-based BH masses and disappear when new, unbiased BH mass estimates (derived in Chapter~3) are adopted. 

\end{document}
